\section{认证加密:定义}\label{sec:9-1}

首先,对于一个密码 $\mathcal{E}$,我们定义什么叫做 $\mathcal{E}$ 提供认证加密。事实上,它必须满足两个属性。第一,$\mathcal{E}$ 必须是 CPA 安全的。第二,$\mathcal{E}$ 必须提供密文完整性,其定义如下。密文完整性是一个新的属性,它抓住这样一个事实,即 $\mathcal{E}$ 应当具备与 MAC 相似的属性。令 $\mathcal{E}=(E,D)$ 是一个定义在 $(\mathcal{K},\mathcal{M},\mathcal{C})$ 上的密码。我们使用下面的攻击游戏来定义密文完整性,如图 \ref{fig:9-2} 所示。该游戏与 MAC 攻击游戏 \ref{game:6-1} 相似。

\begin{game}[密文完整性]\label{game:9-1}
对于一个定义在 $(\mathcal{K},\mathcal{M},\mathcal{C})$ 上的给定密码 $\mathcal{E}=(E,D)$ 和一个给定对手 $\mathcal{A}$,攻击游戏运行如下:
\begin{itemize}
	\item 挑战者随机选择一个 $k\overset{\rm R}\leftarrow\mathcal{K}$。
	\item $\mathcal{A}$ 向挑战者发起多次查询。对于 $i=1,2,\dots$,第 $i$ 次查询包含一条消息 $m_i\in\mathcal{M}$。挑战者计算 $c_i\overset{\rm R}\leftarrow E(k, m_i)$,并将 $c_i$ 发送给 $\mathcal{A}$。
	\item 最后,$\mathcal{A}$ 输出一个候选密文 $c\in\mathcal{C}$,该文本不在 $\mathcal{A}$ 所得到的密文集合中,即:
	\[
	c\notin\{c_1,c_2,\dots\}
	\]
\end{itemize}
如果 $c$ 是用 $k$ 加密得到的一条有效密文,即 $D(k,c)\neq\mathsf{reject}$,我们就称 $\mathcal{A}$ 赢得游戏。我们将 $\mathcal{A}$ 相对于 $\mathcal{E}$ 的优势定义为 $\mathrm{CI}\mathsf{adv}[\mathcal{A},\mathcal{E}]$,即 $\mathcal{A}$ 赢得游戏的概率。最后,如果 $\mathcal{A}$ 最多发起 $Q$ 次加密查询,我们就称 $\mathcal{A}$ 是一个 \textbf{$Q$ 次查询对手 ($Q$-query adversary)}。
\end{game}

\begin{definition}\label{def:9-1}
如果对于每个有效对手 $\mathcal{A}$,$\mathrm{CI}\mathsf{adv}[\mathcal{A},\mathcal{E}]$ 的值都可忽略不计,我们就称密码 $\mathcal{E}=(E,D)$ 提供\textbf{密文完整性 (ciphertext integrity, CI)}。
\end{definition}

CPA 安全性和密文完整性是认证加密所必须的属性。这在下面的定义中得到了体现。

\begin{figure}
  \centering
  

\tikzset{every picture/.style={line width=0.75pt}} %set default line width to 0.75pt        

\begin{tikzpicture}[x=0.75pt,y=0.75pt,yscale=-1,xscale=1]
%uncomment if require: \path (0,212); %set diagram left start at 0, and has height of 212

%Shape: Rectangle [id:dp3703164431038082] 
\draw  [fill={rgb, 255:red, 255; green, 255; blue, 255 }  ,fill opacity=1 ][line width=1.2] [general shadow={fill=black,shadow xshift=2.25pt,shadow yshift=-2.25pt}] (0,0.5) -- (70,0.5) -- (70,20.5) -- (0,20.5) -- cycle ;
%Straight Lines [id:da2288818424042205] 
\draw    (35,20) -- (35,45.5) ;
\draw [shift={(35,48.5)}, rotate = 270] [fill={rgb, 255:red, 0; green, 0; blue, 0 }  ][line width=0.08]  [draw opacity=0] (7.14,-3.43) -- (0,0) -- (7.14,3.43) -- cycle    ;
%Shape: Rectangle [id:dp38434692876948917] 
\draw  [fill={rgb, 255:red, 255; green, 255; blue, 255 }  ,fill opacity=1 ][line width=1.2] [general shadow={fill=black,shadow xshift=2.25pt,shadow yshift=-2.25pt}] (90,0) -- (160,0) -- (160,20) -- (90,20) -- cycle ;
%Straight Lines [id:da5188910523623944] 
\draw    (125,20) -- (125,45.5) ;
\draw [shift={(125,48.5)}, rotate = 270] [fill={rgb, 255:red, 0; green, 0; blue, 0 }  ][line width=0.08]  [draw opacity=0] (7.14,-3.43) -- (0,0) -- (7.14,3.43) -- cycle    ;
%Shape: Rectangle [id:dp3329124039509572] 
\draw  [fill={rgb, 255:red, 255; green, 255; blue, 255 }  ,fill opacity=1 ][line width=1.2] [general shadow={fill=black,shadow xshift=2.25pt,shadow yshift=-2.25pt}] (180,0) -- (250,0) -- (250,20) -- (180,20) -- cycle ;
%Straight Lines [id:da9982393046561384] 
\draw    (215,20.5) -- (215,46) ;
\draw [shift={(215,49)}, rotate = 270] [fill={rgb, 255:red, 0; green, 0; blue, 0 }  ][line width=0.08]  [draw opacity=0] (7.14,-3.43) -- (0,0) -- (7.14,3.43) -- cycle    ;
%Shape: Rectangle [id:dp5817171685461355] 
\draw  [fill={rgb, 255:red, 255; green, 255; blue, 255 }  ,fill opacity=1 ][line width=1.2] [general shadow={fill=black,shadow xshift=2.25pt,shadow yshift=-2.25pt}] (310,0) -- (380,0) -- (380,20) -- (310,20) -- cycle ;
%Straight Lines [id:da43116016737929086] 
\draw    (345,20) -- (345,45.5) ;
\draw [shift={(345,48.5)}, rotate = 270] [fill={rgb, 255:red, 0; green, 0; blue, 0 }  ][line width=0.08]  [draw opacity=0] (7.14,-3.43) -- (0,0) -- (7.14,3.43) -- cycle    ;
%Straight Lines [id:da13149857378263796] 
\draw    (35,90) -- (35,105) -- (180.08,139.31) ;
\draw [shift={(183,140)}, rotate = 193.31] [fill={rgb, 255:red, 0; green, 0; blue, 0 }  ][line width=0.08]  [draw opacity=0] (7.14,-3.43) -- (0,0) -- (7.14,3.43) -- cycle    ;
%Straight Lines [id:da8567076780252318] 
\draw    (125,90) -- (125,105) -- (185.29,133.71) ;
\draw [shift={(188,135)}, rotate = 205.46] [fill={rgb, 255:red, 0; green, 0; blue, 0 }  ][line width=0.08]  [draw opacity=0] (7.14,-3.43) -- (0,0) -- (7.14,3.43) -- cycle    ;
%Straight Lines [id:da4177163318233448] 
\draw    (215,90) -- (215,105) -- (193.83,132.62) ;
\draw [shift={(192,135)}, rotate = 307.48] [fill={rgb, 255:red, 0; green, 0; blue, 0 }  ][line width=0.08]  [draw opacity=0] (7.14,-3.43) -- (0,0) -- (7.14,3.43) -- cycle    ;
%Straight Lines [id:da08439238562392437] 
\draw    (345,90) -- (345,105) -- (199.92,139.31) ;
\draw [shift={(197,140)}, rotate = 346.69] [fill={rgb, 255:red, 0; green, 0; blue, 0 }  ][line width=0.08]  [draw opacity=0] (7.14,-3.43) -- (0,0) -- (7.14,3.43) -- cycle    ;
%Straight Lines [id:da12566654354732099] 
\draw    (190,145) -- (190,177) ;
\draw [shift={(190,180)}, rotate = 270] [fill={rgb, 255:red, 0; green, 0; blue, 0 }  ][line width=0.08]  [draw opacity=0] (7.14,-3.43) -- (0,0) -- (7.14,3.43) -- cycle    ;
%Shape: Trapezoid [id:dp0005402873648003848] 
\draw  [fill={rgb, 255:red, 255; green, 255; blue, 255 }  ,fill opacity=1 ][line width=1.2] [general shadow={fill=black,shadow xshift=2.25pt,shadow yshift=-2.25pt}] (70,50) -- (58,90) -- (12,90) -- (0,50) -- cycle ;
%Shape: Trapezoid [id:dp2231918842135976] 
\draw  [fill={rgb, 255:red, 255; green, 255; blue, 255 }  ,fill opacity=1 ][line width=1.2] [general shadow={fill=black,shadow xshift=2.25pt,shadow yshift=-2.25pt}] (160,50) -- (148,90) -- (102,90) -- (90,50) -- cycle ;
%Shape: Trapezoid [id:dp8197321821978614] 
\draw  [fill={rgb, 255:red, 255; green, 255; blue, 255 }  ,fill opacity=1 ][line width=1.2] [general shadow={fill=black,shadow xshift=2.25pt,shadow yshift=-2.25pt}] (250,50) -- (238,90) -- (192,90) -- (180,50) -- cycle ;
%Shape: Trapezoid [id:dp6599035304294065] 
\draw  [fill={rgb, 255:red, 255; green, 255; blue, 255 }  ,fill opacity=1 ][line width=1.2] [general shadow={fill=black,shadow xshift=2.25pt,shadow yshift=-2.25pt}] (380,50) -- (368,90) -- (322,90) -- (310,50) -- cycle ;

% Text Node
\draw (35,70) node  [font=\small]  {$F( k ,\cdot )$};
% Text Node
\draw (35,10.5) node  [font=\small]  {$( a_{1} ,\ 1)$};
% Text Node
\draw (125,10) node  [font=\small]  {$( a_{2} ,\ 2)$};
% Text Node
\draw (215,10) node  [font=\small]  {$( a_{3} ,\ 3)$};
% Text Node
\draw (345,10) node  [font=\small]  {$( a_{v} ,\ v)$};
% Text Node
\draw (190,140) node  [font=\large]  {$\oplus $};
% Text Node
\draw (125,70) node  [font=\small]  {$F( k ,\cdot )$};
% Text Node
\draw (215,70) node  [font=\small]  {$F( k ,\cdot )$};
% Text Node
\draw (345,70) node  [font=\small]  {$F( k ,\cdot )$};
% Text Node
\draw (192,180) node [anchor=west] [inner sep=0.75pt]  [font=\small]  {$F^{\oplus }( k ,m)$};
% Text Node
\draw (280,10) node    {$\cdots $};
% Text Node
\draw (280,70) node    {$\cdots $};


\end{tikzpicture}
  \caption{密文完整性攻击游戏(攻击游戏 \ref{game:9-1})}
  \label{fig:9-2}
\end{figure}

\begin{definition}\label{def:9-2}
对于一个密码 $\mathcal{E}=(E,D)$,如果 $\mathcal{E}$ (1) 在选择明文攻击下是语义安全的,以及 (2) 能够提供密文完整性,我们就称 $\mathcal{E}$ 能够提供\textbf{认证加密},或者简单地称它是 \textbf{AE 安全}的。
\end{definition}

为什么定义 \ref{def:9-2} 是正确的?特别地,为什么我们要求\emph{密文}完整性,而不是其他的什么\emph{明文}完整性的概念(这可能看起来更自然)?在 \ref{sec:9-2} 节中,我们会描述一类非常阴险的攻击,称为\emph{选择密文攻击},我们将看到,我们对 AE 安全性的定义足以(事实上是必要的)防止这种攻击。在 \ref{sec:9-3} 节中,我们将对该定义给出一个更高层次的理由。

\subsection{一次性认证加密}\label{subsec:9-1-1}

在实践中,我们经常使用对称密钥对单一消息进行加密。该密钥不会被再次使用。例如,在发送加密电子邮件时,我们经常选择一个临时密钥,并用这个临时密钥来加密电子邮件的正文。随后,这个临时密钥也会被加密,然后附在电子邮件的首部中被一起发送出去。每发送一封电子邮件,都会生成一个新的临时密钥。

在这种设置下,我们可以使用一次性的加密方案,比如流密码。所选择的密码必须是语义安全的,但不一定需要是 CPA 安全的。同样地,该密码提供一次性的密文完整性就足够了,这是一个比密文完整性更弱的概念。特别是,我们可以修改攻击游戏 \ref{game:9-1},使得对手只能获得单独的某条消息 $m$ 的加密。

\begin{definition}\label{def:9-3}
如果对于每个有效的单次查询对手 $\mathcal{A}$,$\mathrm{CI}\mathsf{adv}[\mathcal{A},\mathcal{E}]$ 的值都可忽略不计,我们就称密码 $\mathcal{E}=(E,D)$ 提供\textbf{一次性密文完整性 (one-time ciphertext integrity)}。
\end{definition}

\begin{definition}\label{def:9-4}
对于一个密码 $\mathcal{E}=(E,D)$,如果 $\mathcal{E}$ 是语义安全的,并且能够提供一次性密文完整性,我们就称 $\mathcal{E}$ 能够提供\textbf{一次性认证加密(one-time authenticated encryption)},或者称它是 \textbf{1AE 安全}的。
\end{definition}

在那些对称密钥只会被使用一次的应用中,1AE 安全性就足够了。我们将会表明,对于图 \ref{fig:9-1} 中的先加密后 MAC 构造,如果所使用的是语义安全的密码和一次性 MAC,它就能够提供一次性认证加密。用一次性 MAC 替代 MAC 能够显著提高效率。