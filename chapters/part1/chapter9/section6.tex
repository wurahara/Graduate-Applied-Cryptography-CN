\section{另一个变体:带有关联数据的 CCA 安全密码}\label{sec:9-6}

在 \ref{sec:9-5} 节中,我们为密码引入了两个新的特征:nonce 和关联数据。我们可以考虑两种变化:包含 nonce 但没有关联数据的密码,以及包含关联数据但没有 nonce 的密码。我们还可以考察这些变体的其他安全属性,比如 CCA 安全性。详细地考察所有变体显然是枯燥乏味的。尽管如此,我们还是考察一个在后文中很重要的变体,即带有关联数据的 CCA 安全密码(但不包含 nonce)。

为了定义这个概念,我们首先定义带有关联数据的密码,或者 AD 密码的概念,该密码中不包含 nonce。对于这样的一个密码 $\mathcal{E}=(E,D)$,加密算法可能是概率性的,其工作原理如下:
\[
c\overset{\rm R}\leftarrow E(k,m,d)
\]
其中 $c\in\mathcal{C}$ 是密文,$k\in\mathcal{K}$ 是密钥,$m\in\mathcal{M}$ 是消息,$d\in\mathcal{D}$ 是关联数据。解密的语法是:
\[
D(k,c,d)
\]
它输出一条消息 $m$ 或 $\mathsf{reject}$。我们称 AD 密码定义在 $(\mathcal{K},\mathcal{M},\mathcal{D},\mathcal{C})$ 上。和之前一样,我们要求,只要两者被赋予相同的关联数据,由 $E$ 产生的密文就一定能被 $D$ 正确地解密,即:
\[
\Pr\big[D\big(k,\;E(k,m,d),\;d\big)=m\big]=1
\]

\begin{definition}[带有关联数据的 CCA 和 1CCA 安全性]\label{def:9-10}
普通密码的 CCA 安全性定义可以很自然地被迁移到 AD 密码上。我们可以按如下方式修改攻击游戏 \ref{game:9-2}。对于加密查询,除了一对消息 $(m_{i0},m_{i1})$ 外,对手还要提交关联数据 $d_i$,而挑战者会计算 $c_i\overset{\rm R}\leftarrow E(k,m_{ib},d_i)$。对于解密查询,除了密文 $\hat{c}$ 外,对手也要提交关联数据 $\hat{d}$,挑战者计算 $\hat{m}\leftarrow D(k,\hat{c},\hat{d})$。限制在于,数对 $(\hat{c},\hat{d})$ 不能与任何一个之前的加密查询 $(c_1,d_1),\,(c_2,d_2),\dots$ 相等。我们将对手 $\mathcal{A}$ 在这个游戏中的优势定义为 $\mathrm{CCA}_\mathrm{ad}\mathsf{adv}[\mathcal{A},\mathcal{E}]$。如果对于所有的有效对手 $\mathcal{A}$,上述优势都可忽略不计,我们就称该密码是 \textbf{CCA 安全的}。如果我们像定义 \ref{def:9-6} 那样将对手限制为只能发起单次加密查询,它的优势就被定义为 $\mathrm{1CCA}_\mathrm{ad}\mathsf{adv}[\mathcal{A},\mathcal{E}]$。如果对于所有的有效对手 $\mathcal{A}$,这个优势都可忽略不计,我们就称该密码是 \textbf{1CCA 安全的}。
\end{definition}

\begin{snote}[通用先加密后 MAC 组合。]
在后面的应用中,我们主要使用 1CCA 安全性这个概念,所以简单起见,我们现在聚焦于对这个概念的考察。我们可以将一个语义安全密码 $(E,D)$ 和一个一次性 MAC $(S,V)$ 组合起来,以构建一个 1CCA 安全的 AD 密码 $\mathcal{E}=(E_\mathrm{EtM},D_\mathrm{EtM})$,方法如下:

\vspace*{10pt}

\hspace*{20pt} $E_\mathrm{EtM}\big((k_\mathrm{e},k_\mathrm{m}),\;m,d\big)$
				\quad $:=$ \quad
				$c\leftarrow E(k_\mathrm{e},m)$,
				\quad
				$t\leftarrow S\big(k_\mathrm{m},\;(c,d)\big)$\\
\hspace*{175.5pt} 输出 $(c,t)$

\vspace*{5pt}

\hspace*{7pt} $D_\mathrm{EtM}\big((k_\mathrm{e},k_\mathrm{m}),\;(c,t),d\big)$
				\quad $:=$ \quad
				如果 $V\big(k_\mathrm{m},\;(c,d),\;t\big)=\mathsf{reject}$,则输出 $\mathsf{reject}$\\
\hspace*{175.5pt} 否则输出 $D(k_\mathrm{e},c,d)$

\vspace*{10pt}

\noindent
EtM 系统定义在 $(\mathcal{K}_\mathrm{e}\times\mathcal{K}_\mathrm{m},\mathcal{M},\mathcal{D},\mathcal{C}\times\mathcal{T})$ 上。
\end{snote}

\begin{theorem}\label{theo:9-5}
令 $\mathcal{E}=(E,D)$ 是一个语义安全密码,$\mathcal{I}=(S,V)$ 是一个一次性安全的 MAC。则 $\mathcal{E}_\mathrm{EtM}$ 是一个 1CCA 安全的 AD 密码。
\end{theorem}

定理 \ref{theo:9-5} 的证明是非常直接的,我们将其作为一个练习留给读者。

我们观察到,在语义安全密码 $\mathcal{E}=(E,D)$ 的大多数常见实现中,加密算法 $E$ 都是确定性的。同样,在一次性安全 MAC $\mathcal{I}=(S,V)$ 的大多数常见实现中,签名算法也都是确定性的。因此,对于这样的实现,所产生的 1CCA 安全 AD 密码也会包含一个确定性的加密算法。