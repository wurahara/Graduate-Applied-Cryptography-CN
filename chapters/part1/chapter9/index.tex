\chapter{认证加密}\label{chap:9}

本章,我们终于来到了对称加密的最高点。在这里,我们将构建能够同时确保数据机密性和完整性的系统,即使面对的是那些能够同时与发送方和接收方进行恶意交互的,极具攻击性的攻击者。我们将这样的系统称为\textbf{认证加密(authenticated encryption)}。或者简单地称它们是 AE 安全的。本章将会收束我们对对称加密的讨论,还会展示如何在现实世界中正确地进行安全的加密。

回顾一下,当我们在第\ref{chap:5}章中讨论 CPA 安全性时,我们强调,CPA 安全性并不提供任何完整性保证。攻击者可以篡改 CPA 安全密码的输出,而不被解密者发现。在本章中,我们将会介绍许多现实世界中的设置,其中,未被发现的密文篡改会损害消息的机密性和完整性。因此,单靠 CPA 安全性几乎无法满足任何应用的实际需求。相对地,应用程序总是应该使用认证加密来确保消息的机密性和完整性。我们强调,即使一个系统只要求机密性,CPA 安全性也是远远不够的。

在本章中,我们将讨论认证加密的概念,并构建几个 AE 系统。有两种构建 AE 系统的一般范式。第一种被称为\textbf{通用组合(generic composition)},它可以将 CPA 安全的密码与安全的 MAC 相结合。有许多方法可以结合这两个密码学原语,但并非所有的组合都是安全的。下面,我们简单考虑两个例子。

令 $(E,D)$ 是一个密码,$(S,V)$ 是一个 MAC。令 $k_\mathrm{enc}$ 是一个密码密钥,$k_\mathrm{mac}$ 是一个 MAC 密钥。我们立即就可以想到两种将加密与完整性相结合的方案,如图 \ref{fig:9-1} 所示,它们运行如下:
\begin{description}
	\item [先加密后 MAC] 使用 $c\overset{\rm R}\leftarrow E(k_\mathrm{enc},m)$ 加密消息,然后用 $\mathrm{tag}\overset{\rm R}\leftarrow S(k_\mathrm{mac},c)$ 计算密文的 MAC;结果是密文-标签对 $(c,\mathrm{tag})$。这种方法在 TLS 1.2 协议和后续版本中,以及在 IPsec 协议和广泛使用的 NIST GCM 标准中都得到了支持(见 \ref{sec:9-7} 节)。
	\item [先 MAC 后加密] 使用 $\mathrm{tag}\overset{\rm R}\leftarrow S(k_\mathrm{mac},m)$ 计算消息的 MAC,然后使用 $c\overset{\rm R}\leftarrow E\big(k_\mathrm{enc},(m,\mathrm{tag})\big)$ 加密消息-标签对;结果是密文 $c$。这种方法被用在了 TLS 的旧版本(例如 SSL 3.0 和后续的 TLS 1.0 版本)和 802.11i WiFi 加密协议中。
\end{description}
事实证明,对于任意 CPA 安全的密码和安全的 MAC 的组合,都只有第一种方法是安全的。其直觉是,密文上的 MAC 可以防止对密文的任何篡改。我们将表明,第二种方法可能是不安全的——MAC 和密码之间的交互可能会出现问题,导致所产生的系统不是 AE 安全的。这使得一些已经被广泛部署的系统反复遭到攻击。

\begin{figure}
	\centering
	\tikzset{every picture/.style={line width=0.75pt}}

\begin{tikzpicture}[x=0.75pt,y=0.75pt,yscale=-1,xscale=1]

\draw [line width=1.2]  (0,0) -- (140,0) -- (140,20) -- (0,20) -- cycle ;
\draw  [fill={rgb, 255:red, 155; green, 155; blue, 155 }  ,fill opacity=0.5 ][line width=1.2] (0,50) -- (140,50) -- (140,80) -- (0,80) -- cycle ;
\draw  [fill={rgb, 255:red, 155; green, 155; blue, 155 }  ,fill opacity=0.5 ][line width=1.2] (0,120) -- (140,120) -- (140,140) -- (0,140) -- cycle ;
\draw [line width=1.2] (140,120) -- (190,120) -- (190,140) -- (140,140) -- cycle ;

\draw    (0,20) -- (0,120) ;
\draw    (140,0) -- (140,80) ;
\draw    (140,80) -- (190,120) ;

\draw  [line width=1.2] (280,0.67) -- (420,0.67) -- (420,20.67) -- (280,20.67) -- cycle ;
\draw  [fill={rgb, 255:red, 155; green, 155; blue, 155 }  ,fill opacity=0.5 ][line width=1.2] (280,110) -- (470,110) -- (470,140) -- (280,140) -- cycle ;
\draw  [line width=1.2] (280,50) -- (420,50) -- (420,70) -- (280,70) -- cycle ;
\draw  [line width=1.2] (420,50) -- (470,50) -- (470,70) -- (420,70) -- cycle ;

\draw    (280,0.67) -- (280,140) ;
\draw    (420,20.67) -- (470,50) ;
\draw    (470,70) -- (470,110) ;

\draw (70,10) node    {$m$};
\draw (70,130) node    {$c$};
\draw (165,130) node    {$\mathrm{tag}$};
\draw (70,65) node    {$c\leftarrow E(k_{\mathrm{enc}},m)$};
\draw (80,100) node    {$\mathrm{tag}\leftarrow S( k_{\mathrm{mac}},c)$};
\draw (350,10.67) node   {$m$};
\draw (350,60) node    {$m$};
\draw (445,60) node    {$\mathrm{tag}$};
\draw (375,125) node    {$c\leftarrow E\big(k_{\mathrm{enc}} ,(m,\mathrm{tag})\big)$};
\draw (360,35) node    {$\mathrm{tag}\leftarrow S(k_{\mathrm{mac}} ,m)$};
\draw (95,170) node   [align=left] {先加密后 MAC};
\draw (375,170) node   [align=left] {先 MAC 后加密};

\end{tikzpicture}
	\caption{两种组合加密和 MAC 的方法}
	\label{fig:9-1}
\end{figure}

构建认证加密的第二种范式是直接利用分组密码或 PRF,而不需要先构建一个独立的密码或 MAC。这种方法有时被称为\textbf{集成方案(integrated schemes)}。OCB 加密模式是这一范式的一个典型案例(见练习 \ref{exer:9-17})。其他的例子包括 IAPM、XCBC、CCFB 等方案。

\begin{snote}[认证加密标准。]
像 OpenSSL 这样的密码学库通常会提供一个 CPA 安全加密的接口(比如带有随机 IV 的计数器模式),以及一个单独用来计算消息的 MAC 的接口。在过去,开发者需要自己正确地组合这两种密码学原语来实现认证加密。每个系统的做法都不尽相同,而且,并非所有在实践中使用的实例都是安全的。

最近,一些安全认证加密标准被陆续提出。一种被称为 Galois 计数模式 (Galois Counter Mode, GCM) 的流行方法使用先加密后 MAC 的方法来组合随机计数器模式加密和 Carter-Wegman MAC(见 \ref{sec:9-7} 节)。我们将在本章的稍后部分研究这种结构的细节及其安全性。我们鼓励开发者使用由底层密码学库提供的认证加密模式,而不要自己去实现它们。
\end{snote}


\section{认证加密的定义}\label{sec:9-1}
\section{认证加密的引申义}\label{sec:9-2}

在构建 AE 安全的系统之前,让我们先再讨论一下定义 \ref{def:9-1},看看它到底意味着什么。考虑一个发送者 Alice 和一个接收者 Bob,他们共享一个密钥 $k$。Alice 通过公网向 Bob 发送了一连串的消息。每条消息都用一个 AE 安全的密码 $\mathcal{E}=(E,D)$ 加密,使用的密钥就是 $k$。

首先,考虑一个窃听对手 $\mathcal{A}$。由于 $\mathcal{E}$ 是 CPA 安全的,这并不能使得 $\mathcal{A}$ 获取任何新的关于 Alice 发送给 Bob 的消息的信息。

现在,考虑一个更具侵略性的对手 $\mathcal{A}$,它试图让 Bob 收到一条并不由 Alice 发出的消息。我们声称这不可能发生。为了了解原因,不妨考虑下面这个只有一条消息的例子:Alice 向 Bob 发送一条 $m$ 的加密消息,但密文 $c$ 在中途被 $\mathcal{A}$ 截获。对手的目标是创造某个 $\hat{c}$,使得 $\hat{m}:=D(k,\hat{c})\neq\mathsf{reject}$,同时 $\hat{m}\neq m$。这样的 $\hat{c}$ 就能够欺骗 Bob,使他认为 Alice 发送的是 $\hat{m}$,而不是 $m$。但这样一来,$\mathcal{A}$ 也能够就 $\mathcal{E}$ 赢得攻击游戏 \ref{game:9-1},而这与 $\mathcal{E}$ 的密文完整性相矛盾。因此,$\mathcal{A}$ 不可能修改 $c$ 而不被发现。更一般地说,将该论证应用于多条消息的场合,我们就能表明,$\mathcal{A}$ 不能使 Bob 接受任何不由 Alice 发出的消息。这里,更一般的结论是,\emph{密文完整性}就能够导出\emph{消息完整性}。

\subsection{选择密文攻击:一个例子}\label{subsec:9-2-1}

我们现在考虑一种更具侵略性的攻击类型,称为\textbf{选择密文攻击(chosen ciphertext attack)}。正如我们将要看到的,即使面对如此强大的攻击,AE 安全的密码也能为消息提供机密性和完整性。

为了说明选择密文攻击,假设 Alice 向 Bob 发送了一封电子邮件。简单起见,我们假设每封电子邮件都以字符 \texttt{To:} 为开头,后面紧跟着收件人的邮箱地址。因此,发送给 Bob 的邮件会以 \texttt{To:bob@mail.com} 作为开头,而发送给 Mel 的邮件则会从 \texttt{To:mel@mail.com} 开始。邮件服务器会解密收到的每一封邮件,并将其发送到收件人的收件箱:以 \texttt{To:bob@mail.com} 为开头的邮件会被送入 Bob 的收件箱,而以 \texttt{To:mel@mail.com} 为开头的邮件会被送入 Mel 的收件箱。

这个故事中的攻击者 Mel 想要阅读 Alice 发给 Bob 的邮件。对 Mel 来说不幸的是,Alice 很小心,她用一个只有 Alice 和邮件服务器知道的密钥加密了邮件。当邮件服务器收到密文 $c$ 时,它会解密密文,并将得到的明文发送到 Bob 的收件箱。因此,Mel 是无法阅读这封邮件的。

然而,我们表明,如果 Alice 用 CPA 安全的密码,比如随机化计数器模式或随机化 CBC 模式来加密邮件,Mel就能很容易地获得邮件内容。方法如下:Mel 在密文 $c$ 到达邮件服务器之前截获它,并对其进行修改,以获取一个新的密文 $\hat{c}$,并使 $\hat{c}$ 的密文以 \texttt{To:mel@mail.com} 为开头,但其余部分与原始消息相同。此后,Mel 再将 $\hat{c}$ 转发给邮件服务器。当邮件服务器收到 $\hat{c}$ 时,它会解密这条密文,并(错误地)将明文发送给 Mel 的收件箱,使得 Mel 可以轻松读取它。

为了成功发起这种攻击,Mel 必须首先解决以下问题:给定某条消息 $(u\,\Vert\,m)$ 的加密 $c$,其中 $u$ 是一个固定的已知前缀(在我们的例子中,$u:=\texttt{To:bob@mail.com}$),生成一条密文 $\hat{c}$,它的解密是 $(v\,\Vert\,m)$,其中,$v$ 是另一个前缀(在我们的例子中,$v:=\texttt{To:mel@mail.com}$)。

我们表明,如果加密方案是随机化计数器模式或随机化 CBC 模式,Mel 就可以轻松地解决这个问题。简单起见,我们假设 $u$ 和 $v$ 都是二进制序列,其长度与底层分组密码的分组大小相同。与之前一样,我们记 $c[0]$ 和 $c[1]$ 分别是 $c$ 的第一个和第二个分组,其中 $c[0]$ 就是随机 IV。Mel 构建 $\hat{c}$ 的方法如下:
\begin{itemize}
	\item 随机化计数器模式:$\hat{c}$ 与 $c$ 基本相同,只是 $\hat{c}[1]:=c[1]\oplus u\oplus v$。
	\item 随机化 CBC 模式:$\hat{c}$ 与 $c$ 基本相同,只是 $\hat{c}[0]:=c[0]\oplus u\oplus v$。
\end{itemize}
不难看出,在这两种情况下,对 $\hat{c}$ 的解密都是从前缀 $v$ 开始的(见 \ref{subsec:3-3-2} 小节)。现在,Mel 就能够获得 $\hat{c}$ 的解密,并且明文读取秘密消息 $m$。

刚才发生了什么?我们已经证明了,这两种加密模式都是 CPA 安全的,但是上面的介绍展示了破解它们的方法。这就是一个选择密文攻击的例子——通过查询对 $\hat{c}$ 的解密,Mel 就能够推断出对 $c$ 的解密。这种攻击再次展示了攻击者如何利用一个密码的\emph{易被控制性(malleability)}——我们曾经在 \ref{subsec:3-3-2} 小节中展示过一个基于易被控制性的攻击方式。

正如我们刚刚看到的,当攻击者可以解密某些密文时,即便他不能直接解密他感兴趣的密文,一个 CPA 安全的系统也会变得完全不安全。换句话说,密文完整性的缺失就会完全损害机密性——就算明文完整性不是明确的安全要求,情况也是如此。

我们非正式地论证,如果 Alice 使用的是一个 AE 安全的密码 $\mathcal{E}=(E,D)$,上述攻击就无法成功。假设 Mel 截获了一条密文 $c:=E(k,m)$。他试图创建另一条密文 $\hat{c}$,它满足 (1) $\hat{m}:=D(k,\hat{c})$ 从前缀 $v$ 开始,以及 (2) 对手可以从 $\hat{m}$ 恢复 $m$,特别地,$\hat{m}\neq\mathsf{reject}$。密文完整性——以及随之而来的 AE 安全性——意味着,攻击者无法创建这样的 $\hat{c}$。事实上,攻击者无法创建任何新的有效密文,因此,一个 AE 安全的密码就可以挫败这种攻击。

在下一小节中,我们将正式定义选择密文攻击。我们还将表明,如果一个密码是 AE 安全的,那么即使面对这种类型的攻击,它也是安全的。
 
\subsection{选择密文攻击:定义}\label{subsec:9-2-2}

在这一小节中,我们将正式定义选择密文攻击。在这样的攻击中,对手拥有选择明文攻击中攻击者所拥有的所有权力,除此之外,它还可以获取由它选取的密文的解密——但受到一个限制。回顾一下,在选择明文攻击中,对手向挑战者发起了一连串的加密查询,并获取了一些密文作为应答。我们施加的限制是,对手不得要求解密这些密文中的任意一条。虽然这样的限制对于使攻击游戏有意义来说是有必要的,但它看起来可能有点不直观:如果对手可以解密由它选择的密文,为什么它不能解密最重要的密文?我们将在后面(\ref{sec:9-3} 节)更深层次地解释这个定义背后的直觉。我们将在下面(\ref{subsec:9-2-3} 小节)表明,如果一个密码是 AE 安全的,它对选择密文攻击也是安全的。

下面是正式的攻击游戏:

\begin{game}[CCA 安全性]\label{game:9-2}
对于一个定义在 $(\mathcal{K},\mathcal{M},\mathcal{C})$ 上的给定密码 $\mathcal{E}=(E,D)$ 和一个给定对手 $\mathcal{A}$,我们定义两个实验。对于 $b=0,1$,我们定义:

\vspace*{5pt}

\noindent\textbf{实验 $b$:}
\begin{itemize}
	\item 挑战者随机选取 $k\overset{\rm R}\leftarrow\mathcal{K}$。
	\item $\mathcal{A}$ 向挑战者发起一连串查询。每个查询都属于下面两种类型中的一个:
	\begin{itemize}
		\item \emph{加密查询}:对于 $i=1,2,\dots$,第 $i$ 个加密查询由一对消息 $(m_{i0},m_{i1})\in\mathcal{M}^2$ 组成。挑战者计算 $c_i\overset{\rm R}\leftarrow E(k,m_{ib})$,并将 $c_i$ 发送给 $\mathcal{A}$。
		\item \emph{解密查询}:对于 $j=1,2,\dots$,第 $j$ 个解密查询由一条密文 $\hat{c}_j\in\mathcal{C}$ 组成,该密文不是对之前任意一次加密查询的应答,即:
			\[
			\hat{c}_j\notin\{c_1,c_2,\dots\}
			\]
		挑战者计算 $\hat{m}_j\leftarrow D(k,\hat{c}_j)$,并将 $\hat{m}_j$ 发送给 $\mathcal{A}$。
	\end{itemize}
	\item 在游戏结束时,对手输出一个比特 $\hat{b}\in\{0,1\}$。
\end{itemize}
令 $W_b$ 是 $\mathcal{A}$ 在实验 $b$ 中输出 $1$ 的事件。我们将 $\mathcal{A}$ 相对于 $\mathcal{E}$ 的\textbf{优势}定义为:
\[
\mathrm{CCA}\mathsf{adv}[\mathcal{A},\mathcal{E}]
:=
\big\lvert
\Pr[W_0]-\Pr[W_1]
\big\rvert
\]
\end{game}

我们强调,在上述攻击游戏中,加密查询和解密查询可以以任意次序交错进行。

\begin{definition}[CCA 安全性]\label{def:9-5}
如果对于所有的有效对手 $\mathcal{A}$,$\mathrm{CCA}\mathsf{adv}[\mathcal{A},\mathcal{E}]$ 的值都可忽略不计,我们就称密码 $\mathcal{E}$ \textbf{对选择密文攻击是语义安全的(semantically secure against chosen ciphertext attack)},简称为 \textbf{CCA 安全的 (CCA-secure)}。
\end{definition}

在某些情况下,每条消息都需要生成一个新的密钥,因此,一个特定的密钥 $k$ 只会被用于加密一条消息。当攻击者试图愚弄用户用一个 $k$ 来解密多条密文时,系统也想要对选择密文攻击保持安全性。对于这些情况,我们定义针对下述对手的安全性,它只能发起\emph{一次}加密查询,但是可以发起多次解密查询。

\begin{definition}[1CCA 安全性]\label{def:9-6}
在攻击游戏 \ref{game:9-2} 中,如果对手 $\mathcal{A}$ 被限制只能发起一次加密查询,我们就将其优势表示为 $\mathrm{1CCA}\mathsf{adv}[\mathcal{A},\mathcal{E}]$。如果对于所有的有效对手 $\mathcal{A}$,$\mathrm{1CCA}\mathsf{adv}[\mathcal{A},\mathcal{E}]$ 的值都可忽略不计,我们就称密码 $\mathcal{E}$ \textbf{对选择密文攻击是一次性语义安全的(one-time semantically secure against chosen ciphertext attack)},简称为 \textbf{1CCA 安全的 (CCA-secure)}。
\end{definition}

正如 \ref{subsec:2-2-5} 小节所讨论的,攻击游戏 \ref{game:9-2} 也可以被重构为一个``比特猜测"游戏,此时,挑战者不再拥有两个独立的实验,而是会随机选择一个 $b\in\{0,1\}$,然后针对对手 $\mathcal{A}$ 运行实验 $b$。在这个游戏中,我们定义 $\mathcal{A}$ 的\emph{比特猜测优势} $\mathrm{CCA}\mathsf{adv}^*[\mathcal{A},\mathcal{E}]$(以及 $\mathrm{1CCA}\mathsf{adv}^*[\mathcal{A},\mathcal{E}]$)为 $\lvert\Pr[\hat{b}=b]-1/2\lvert$。\ref{subsec:2-2-5} 小节中的一般结论(即式 \ref{eq:2-11})在此也同样适用:
\begin{equation}\label{eq:9-1}
\mathrm{CCA}\mathsf{adv}[\mathcal{A},\mathcal{E}]
=
2\cdot
\mathrm{CCA}\mathsf{adv}^*[\mathcal{A},\mathcal{E}]
\end{equation}
同样地,对于被限制只能进行单次加密查询的对手,我们有:
\begin{equation}\label{eq:9-2}
\mathrm{1CCA}\mathsf{adv}[\mathcal{A},\mathcal{E}]
=
2\cdot
\mathrm{1CCA}\mathsf{adv}^*[\mathcal{A},\mathcal{E}]
\end{equation}

% TODO: here

\subsection{认证加密意味着选择密文安全性}\label{subsec:9-2-3}

我们现在表明,每个AE安全的系统也是CCA安全的。同样地,每个1AE安全的系统都是1CCA安全的。

\begin{theorem}\label{theo:9-1}
让E = (E,D) 是一个密码。如果E是AE安全的,那么它就是CCA安全的。如果E是1AE安全的,那么它就是1CCA安全的。

特别是,假设A是E的一个CCA-反面,它最多只能进行Qe次加密查询和Qd次解密查询。那么,存在一个CPA-adversary Bcpa和一个CI-adversary Bci,其中Bcpa和Bci是围绕A的基本包装,从而

CCAadv[A, E] ≤ CPAadv[Bcpa, E] + 2Qd - CIadv[Bci, E]。

此外,Bcpa和Bci都最多进行Qe次加密查询。
\end{theorem}

在证明这个定理之前,我们要指出一个反证:如果一个密码是CCA安全的,并且提供明文完整性,那么它一定是AE安全的。在练习9.15中要求你证明这一点。这两个结果共同为AE安全是不安全网络上通用通信的正确安全概念这一说法提供了有力支持。我们还注意到,有可能建立一个不提供密文(或明文)完整性的CCA安全密码--见练习9.12的例子。

\begin{proof}[证明思路]
一个CCA对手A发出加密和允许解密的查询。我们首先论证,对所有这些解密查询的响应必须是拒绝。为了了解原因,观察一下,如果对手曾经发出一个有效的解密查询ci,其解密不是拒绝,那么这个ci可以用来赢得密码文本完整性游戏。因此,由于A的所有解密查询都被拒绝了,对手通过发出解密查询没有学到任何东西,还不如把它们丢掉。在去除解密查询后,我们最终得到一个标准的CPA游戏。由于E是CPA安全的,对手无法赢得这个游戏。我们的结论是,A在赢得CCA游戏中的优势可以忽略不计。
\end{proof}

\begin{proof}
假设A是一个高效的CCA对手,如攻击游戏9.2中那样攻击E,它最多进行Qe次加密查询和Qd次解密查询。我们想证明CCAadv[A,E]是可以忽略的,假设E是AE安全的。我们将使用CCA和CPA攻击游戏的比特猜测版本,并表明

CCAadv∗[A, E] ≤ CPAadv∗[Bcpa, E] + Qd - CIadv[Bci, E]。(9.4)

为有效对手Bcpa和Bci。然后(9.3)由(9.4)以及(9.1)和(5.4)得出。此外,我们将看到,对手Bcpa最多只能进行Qe次加密查询;因此,如果E是1AE安全的,它也是1CCA安全的。

让我们把游戏0定义为攻击游戏9.2的比特猜测版本。在这个游戏中,挑战者被称为游戏0,其工作方式如下。

b←R {0,1}//A在收到第1个加密查询时,将尝试猜测b k←R K
在收到A的第i个加密查询(mi0,mi1)时,做:发送ci←R E(k,mb)给A
在收到A的第j个解密查询cˆj时,做:发送D(k,cˆj)给A
(1)

最终,对手输出一个猜测ˆb∈{0,1}。如果b = ˆb,我们就说A赢得了游戏。用W0表示这一事件。根据定义,比特猜测的优势是

CCAadv∗[A, E] = |Pr[W0] - 1/2|。(9.5)

游戏1。现在我们把挑战者的第(1)行修改如下。

(1) 发送拒绝信息给A

我们认为,A无法将这个挑战者与原挑战者区分开来。设Z是指在游戏1中,A发出一个解密查询cˆj,使D(k, cˆj ) ̸=拒绝的事件。显然,只要Z不发生,游戏0和1的进程是相同的。因此,根据差异定理(即定理4.7),可以得出|Pr[W0] - Pr[W1]| ≤ Pr[Z]。

使用类似于定理6.1证明中使用的 "猜测策略",我们可以使用A来建立一个CI对抗者Bci,以至少Pr[Z]/Qd的概率赢得CI攻击游戏。请注意,在游戏1中,根本没有使用解密算法。对手Bci的策略是简单地猜测一个随机数ω∈{1, . . . , Qd},然后扮演A的挑战者的角色。

- 当A进行加密查询时,Bci将其转发给自己的挑战者,并将响应返回给A。
- 当A进行解密查询cˆj时,Bci只是简单地将拒绝发送给A,但如果j = ω,Bci会输出cˆj并停止。

不难看出,CIadv[Bci,E]≥Pr[Z]/Qd,所以

|Pr[W0]-Pr[W1]|≤Pr[Z]≤Qd-CIadv[Bci, E]。(9.6)

最后的还原。由于在博弈1中所有的解密查询都被拒绝,这本质上是一个CPA攻击游戏。更确切地说,我们可以构建一个CPA对手Bcpa,它扮演着A的挑战者的角色,如下所示。

- 当A进行加密查询时,Bcpa将其转发给自己的挑战者,并将响应返回给A。
- 当A进行解密查询时,Bcpa只是向A发送拒绝。

在游戏结束时,Bcpa只需输出A所输出的位ˆb。显然。

|Pr[W1] - 1/2| = CPAadv∗[Bcpa, E] 。

将方程(9.5)-(9.7)放在一起,可以得到(9.4),这就证明了该定理。
\end{proof}
\section{作为抽象接口的加密}
\section{基于通用组合的认证加密密码}\label{sec:9-4}

在本节,我们试图通过组合一个 CPA 安全的密码和一个安全的 MAC 组合来构建认证加密。我们将会表明,先加密后 MAC 范式总是 AE 安全的,但是先 MAC 后加密范式却不是这样。

\subsection{先加密后MAC}\label{subsec:9-4-1}

令 $\mathcal{E}=(E,D)$ 是一个定义在 $(\mathcal{K}_\mathrm{e},\mathcal{M},\mathcal{C})$ 上的密码,$\mathcal{I}=(S,V)$ 是一个定义在 $(\mathcal{K}_\mathrm{m},\mathcal{C},\mathcal{T})$ 上的 MAC。先加密后 MAC 系统 $\mathcal{E}_\mathrm{EtM}=(E_\mathrm{EtM},D_\mathrm{EtM})$,简称为 $\mathrm{EtM}$,定义如下:

\vspace*{10pt}

\hspace*{20pt} $E_\mathrm{EtM}\big((k_\mathrm{e},k_\mathrm{m}),\;m\big)$
				\quad $:=$ \quad
				$c\overset{\rm R}\leftarrow E(k_\mathrm{e},m)$,
				\quad
				$t\overset{\rm R}\leftarrow S(k_\mathrm{m},c)$\\
\hspace*{165.5pt} 输出 $(c,t)$

\vspace*{5pt}

\hspace*{7pt} $D_\mathrm{EtM}\big((k_\mathrm{e},k_\mathrm{m}),\;(c,t)\big)$
				\quad $:=$ \quad
				如果 $V(k_\mathrm{m},c,t)=\mathsf{reject}$,则输出 $\mathsf{reject}$\\
\hspace*{165.5pt} 否则输出 $D(k_\mathrm{e},c)$

\vspace*{10pt}

\noindent
我们称 $\mathrm{EtM}$ 系统定义在 $(\mathcal{K}_\mathrm{e}\times\mathcal{K}_\mathrm{m},\;\mathcal{M},\;\mathcal{C}\times\mathcal{T})$ 上。下面的定理将表明,$\mathcal{E}_\mathrm{EtM}$ 能够提供认证加密。

\begin{theorem}\label{theo:9-2}
令 $\mathcal{E}=(E,D)$ 是一个密码,$\mathcal{I}=(S,V)$ 是一个 MAC 系统。如果 $\mathcal{E}$ 是 CPA 安全的,且 $\mathcal{I}$ 是一个安全的 MAC 系统,则 $\mathcal{E}_\mathrm{EtM}$ 就是 AE 安全的。此外,如果 $\mathcal{E}$ 是语义安全的,且 $\mathcal{I}$ 是一个一次性安全的 MAC 系统,则 $\mathcal{E}_\mathrm{EtM}$ 就是 1AE 安全的。
\begin{quote}
特别地,对于每个像攻击游戏 \ref{game:9-1} 中那样攻击 $\mathcal{E}_\mathrm{EtM}$ 的密文完整性对手 $\mathcal{A}_\mathrm{ci}$,都存在一个像攻击游戏 \ref{game:6-1} 中那样攻击 $\mathcal{I}$ 的 MAC 对手 $\mathcal{B}_\mathrm{mac}$,其中 $\mathcal{B}_\mathrm{mac}$ 是一个围绕 $\mathcal{A}_\mathrm{ci}$ 的基本包装器,由其发起的签名查询不多于由 $\mathcal{A}_\mathrm{ci}$ 发起的加密查询,满足:
\end{quote}
\[
\mathrm{CI}\mathsf{adv}[\mathcal{A}_\mathrm{ci},\mathcal{E}_\mathrm{EtM}]
=
\mathrm{MAC}\mathsf{adv}[\mathcal{B}_\mathrm{mac},\mathcal{I}]
\]
\begin{quote}
对于每个像攻击游戏 \ref{game:5-2} 中那样攻击 $\mathcal{E}_\mathrm{EtM}$ 的 CPA 对手 $\mathcal{A}_\mathrm{cpa}$,都存在一个像攻击游戏 \ref{game:5-2} 中那样攻击 $\mathcal{E}$ 的 CPA 对手 $\mathcal{B}_\mathrm{cpa}$,其中 $\mathcal{B}_\mathrm{cpa}$ 是一个围绕 $\mathcal{A}_\mathrm{cpa}$ 的基本包装器,由其发起的加密查询不多于由 $\mathcal{A}_\mathrm{cpa}$ 发起的加密查询,满足:
\end{quote}
\[
\mathrm{CPA}\mathsf{adv}[\mathcal{A}_\mathrm{cpa},\mathcal{E}_\mathrm{EtM}]
=
\mathrm{CPA}\mathsf{adv}[\mathcal{B}_\mathrm{cpa},\mathcal{E}]
\]
\end{theorem}

\begin{proof}
我们先证明 $\mathcal{E}_\mathrm{EtM}$ 能够提供密文完整性。该证明可以通过一个直接的归约实现。假设 $\mathcal{A}_\mathrm{ci}$ 是一个攻击 $\mathcal{E}_\mathrm{EtM}$ 的密文完整性对手。我们构建一个攻击 $\mathcal{I}$ 的 MAC 对手 $\mathcal{B}_\mathrm{mac}$。

对手 $\mathcal{B}_\mathrm{mac}$ 在 $\mathcal{I}$ 的 MAC 攻击游戏中扮演对手的角色,它与一个 MAC 挑战者 $\mathbf{C}_\mathrm{mac}$ 交互,后者在交互开始时随机选取一个 $k_\mathrm{m}\overset{\rm R}\leftarrow\mathcal{K}_\mathrm{m}$。然后,对手 $\mathcal{B}_\mathrm{mac}$ 模拟 $\mathcal{A}_\mathrm{ci}$ 的 $\mathcal{E}_\mathrm{EtM}$ 密文完整性挑战者,运行如下:

\vspace*{10pt}

\hspace*{5pt} 随机选取 $k_\mathrm{e}\overset{\rm R}\leftarrow\mathcal{K}_\mathrm{e}$\\
\hspace*{26pt} 当从 $\mathcal{A}_\mathrm{ci}$ 处收到一个查询 $m_i\in\mathcal{M}$ 时:\\
\hspace*{50pt} 令 $c_i\overset{\rm R}\leftarrow E(k_\mathrm{e},m_i)$\\
\hspace*{50pt} 在 $c_i$ 处查询 $\mathbf{C}_\mathrm{mac}$,并获得一个应答 $t_i\overset{\rm R}\leftarrow S(k_\mathrm{m},c_i)$\\
\hspace*{50pt} 将 $(c_i,t_i)$ 发送给 $\mathcal{A}_\mathrm{ci}$
\hspace*{10pt} // \quad \emph{于是有} $(c_i,t_i)=E_\mathrm{EtM}\big((k_\mathrm{e},k_\mathrm{m}),m_i\big)$\\
\hspace*{26pt} 最后,$\mathcal{A}_\mathrm{ci}$ 输出一条密文 $(c,t)\in\mathcal{C}\times\mathcal{T}$\\
\hspace*{26pt} 输出该消息-标签对 $(c,t)$

\vspace*{10pt}

\noindent
应该明确的是,$\mathcal{B}_\mathrm{mac}$ 就像在一个真正的密文完整性攻击游戏中一样对 $\mathcal{A}_\mathrm{ci}$ 的查询给出了应答。因此,对手 $\mathcal{A}_\mathrm{ci}$ 能以 $\mathrm{CI}\mathsf{adv}[\mathcal{A}_\mathrm{ci},\mathcal{E}_\mathrm{EtM}]$ 的概率输出一条能让它赢得攻击游戏 \ref{game:9-1} 的密文 $(c,t)$,满足 $(c,t)\notin\{(c_1,t_1),\dots\}$ 且 $V(k_\mathrm{m},c,t)=\mathsf{accept}$。由此可知,$(c,t)$ 是一个能让 $\mathcal{B}_\mathrm{mac}$ 赢得 MAC 攻击游戏的消息-标签对。因此,我们可得 $\mathrm{CI}\mathsf{adv}[\mathcal{A}_\mathrm{ci},\mathcal{E}_\mathrm{EtM}]=\mathrm{MAC}\mathsf{adv}[\mathcal{B}_\mathrm{mac},\mathcal{I}]$,正如定理所要求的。

剩下的工作就是证明,如果 $\mathcal{E}$ 是 CPA 安全的,则 $\mathcal{E}_\mathrm{EtM}$ 也是安全的。这就等于是说,密文中包含的那个使用密钥 $k_\mathrm{m}$ 计算出来的(因而根本不涉及加密密钥 $k_\mathrm{e}$)标签不会在攻击者破坏 $\mathcal{E}_\mathrm{EtM}$ 的 CPA 安全性时提供帮助。这部分的证明非常简单,我们将其留作练习(可参见练习 \ref{exer:5-20})。
\end{proof}

回顾一下我们在第\ref{chap:6}章中给出的安全的 MAC 的定义,我们要求,给定一个消息-标签对 $(c,t)$,攻击者无法制造一个新的标签 $t\neq t'$,同时使得 $(c,t')$ 也是一个合法的消息-标签对。在当时,这样的要求似乎有些奇怪:如果攻击者已经有了一个 $c$ 的有效标签,我们为什么还要关心它能否为 $c$ 找到另一个标签?现在,我们看到,如果攻击者能为 $c$ 找到另一个有效标签 $t'$,他就可以破坏 $\mathrm{EtM}$ 的密文完整性。攻击者可以使用 $\mathrm{EtM}$ 的密文 $(c,t)$ 来构建另一条有效密文 $(c,t')$,并赢得密文完整性游戏。我们对安全的 MAC 的定义能够确保,攻击者无法修改 $\mathrm{EtM}$ 的密文而不被发现。

\subsubsection{实现先加密后 MAC 时的常见错误}\label{subsubsec:9-4-1-1}

在实现先加密后 MAC 时,一个常见的错误是为密码和 MAC 选用相同的密钥,即设置 $k_\mathrm{e}=k_\mathrm{m}$。由此产生的系统无法提供认证加密,而且可能是不安全的,如练习 \ref{exer:9-8} 中所展示的那样。在定理 \ref{theo:9-2} 的证明中,我们利用了这样一个事实,即两个密钥 $k_\mathrm{e}$ 和 $k_\mathrm{m}$ 的选择是相互独立的。

另一种常见的错误是只对密文的一部分应用 MAC 签名算法。我们看一个例子。假设底层的 CPA 安全密码 $\mathcal{E}=(E,D)$ 是由随机化 CBC 模式(见 \ref{subsec:5-4-3} 小节)构建的,那么消息 $m$ 的加密就是 $(r,c)\overset{\rm R}\leftarrow E(k,m)$,这里的 $r$ 是一个随机 IV。当实现先加密后 MAC $\mathcal{E}_\mathrm{EtM}=(E_\mathrm{EtM},D_\mathrm{EtM})$ 时,加密算法被错误地定义为:
\[
E_\mathrm{EtM}\big((k_\mathrm{e},k_\mathrm{m}),\;m\big)
:=
\big\{
(r,c)\overset{\rm R}\leftarrow E(k_\mathrm{e},m),
\;
t\overset{\rm R}\leftarrow S(k_\mathrm{m},c),
\;
\text{输出}\,(r,c,t)
\big\}
\]
这里,$E(k_\mathrm{e},m)$ 输出了密文 $(r,c)$,但 MAC 签名算法只被应用到了 $c$ 上,而 IV 没有受到 MAC 的保护。这个错误会完全破坏密文完整性:给定一条密文 $(r,c,t)$,攻击者可以创建另一条有效密文 $(r',c,t)$,而 $r'\neq r$。解密算法无法检测到这种对 IV 的修改,更不会输出 $\mathsf{reject}$。相反,解密算法会输出 $D\big(k_\mathrm{e},(r',c)\big)$。由于 $(r',c,t)$ 是一条有效密文,对手就赢得了密文完整性游戏。更糟糕的是,假如 $(r,c,t)$ 是对一条消息 $m$ 的加密,那么对于任意的 $\Delta$,将 $(r,c,t)$ 改为 $(r\oplus\Delta,c,t)$ 都会使得 CBC 解密算法输出一条消息 $m'$,而 $m'[0]=m[0]\oplus\Delta$。这意味着,攻击者可以将 $m$ 的第一个分组中的头部信息改为它自己所选择的任何值。ISO 19772 认证加密标准的一个早期版本正是犯了这个错误 \cite{namprempre2014reconsidering}。类似地,在 2013 年,有人发现苹果公司的 iOS 系统中为数据加密而建立的 \texttt{RNCryptor} 设施使用了一个错误的先加密后 MAC,其中的 HMAC 没有被应用到加密 IV 上 \cite{napier2013rncryptor}。

在实现中,另一个需要注意的隐患是,在整条消息的完整性标签都被验证完成之前,系统不应当输出任何明文数据。\ref{sec:9-9} 节会介绍这方面的一个例子。

\subsection{先 MAC 后加密一般是不安全的:SSL 上的填充预言机攻击}\label{subsec:9-4-2}

接下来,我们考虑由一个 CPA 安全密码和一个安全的 MAC 的构成的先 MAC 后加密通用组合。我们将表明,这种构造不一定是 AE 安全的,而且可能会导致许多现实世界中的问题。

为了准确定义先 MAC 后加密范式,令 $\mathcal{I}=(S,V)$ 是一个定义在 $(\mathcal{K}_\mathrm{m},\mathcal{M},\mathcal{T})$ 上的 MAC,$E=(E,D)$ 是一个定义在 $(\mathcal{K}_\mathrm{e},\mathcal{M}\times\mathcal{T},\mathcal{C})$ 上的密码。先 MAC 后加密系统 $\mathcal{E}_\mathrm{MtE}=(E_\mathrm{MtE},D_\mathrm{MtE})$,或简称 $\mathrm{MtE}$,定义如下:

\vspace*{10pt}

\hspace*{20pt} $E_\mathrm{MtE}\big((k_\mathrm{e},k_\mathrm{m}),\;m\big)$
				\quad $:=$ \quad
				$t\overset{\rm R}\leftarrow E(k_\mathrm{m},m)$,
				\quad
				$c\overset{\rm R}\leftarrow S\big(k_\mathrm{e},\,(m,t)\big)$\\
\hspace*{165.5pt} 输出 $c$

\vspace*{5pt}

\hspace*{24pt} $D_\mathrm{MtE}\big((k_\mathrm{e},k_\mathrm{m}),\;c\big)$
				\quad $:=$ \quad
				$(m,t)\leftarrow D(k_\mathrm{e},c)$\\
\hspace*{165.5pt} 如果 $V(k_\mathrm{m},m,t)=\mathsf{reject}$,则输出 $\mathsf{reject}$\\
\hspace*{165.5pt} 否则输出 $D(k_\mathrm{e},c)$

\vspace*{10pt}

\noindent
我们称 $\mathrm{MtE}$ 系统定义在 $(\mathcal{K}_\mathrm{e}\times\mathcal{K}_\mathrm{m},\;\mathcal{M},\;\mathcal{C})$。

\begin{snote}[一种被彻底破解的 $\mathbf{MtE}$ 密码。]
我们表明,就算 $\mathcal{E}$ 是一个 CPA 安全的密码,$\mathcal{I}$ 是一个安全的 MAC,$\mathrm{MtE}$ 也不一定是 AE 安全的。事实上,对于广泛使用的密码和 MAC 来说,$\mathrm{MtE}$ 很可能是不安全的,而这在事实上已经导致了很多针对已部署系统的重大攻击。

考虑一下用于保护 WWW 流量的 SSL 3.0 协议,该协议已经被使用了二十多年(但在现代浏览器中被禁用)。SSL 3.0 使用 $\mathrm{MtE}$ 来组合随机化 CBC 模式加密和安全 MAC。我们曾在第\ref{chap:5}章中表明,随机化 CBC 模式加密是 CPA 安全的,尽管如此,这种组合仍然被彻底破解了:攻击者可以使用选择密文攻击有效地解密所有流量。这就导致了一种针对 SSL 3.0 的破坏性攻击,称为 \textbf{POODLE} \cite{moller2014poodle}。

我们假设 CBC 中所使用的底层分组密码运行在 $16$ 字节分组上,就像 AES 那样。回顾一下,CBC 模式加密将其输入填充到分组长度的整数倍,而在 SSL 3.0 中,具体的做法如下:如果需要一个长度为 $p>0$ 字节的填充序列,该方案就用一个长为 $p-1$ 字节的任意序列来填充消息,并且增加再额外增加一个字节,该字节的值就是 $(p-1)$。如果消息长度已经是分组长度($16$ 字节)的整数倍,SSL 3.0 就会添加一个长为 $16$ 字节的假分组,其最后一个字节被置为 $15$,而其前 $15$ 字节是任意内容。在解密过程中,算法会读取最后一个字节,并移除其值相应的那么多字节,以正确地移除填充。

具体来说,将 $\mathrm{MtE}$ 应用于随机化 CBC 模式加密和安全 MAC 得到的密码 $\mathcal{E}_\mathrm{MtE}=(E_\mathrm{MtE},D_\mathrm{MtE})$ 工作如下:
\begin{itemize}
	\item $E_\mathrm{MtE}\big((k_\mathrm{e},k_\mathrm{m}),\,m\big)$:首先,使用 MAC 签名算法为 $m$ 计算一个定长标签 $t\overset{\rm R}\leftarrow E(k_\mathrm{m},m)$。然后,用随机化 CBC 加密对 $m\,\Vert\,t$ 进行加密:对信息进行填充,然后使用密钥 $k_\mathrm{e}$ 和一个随机的 IV 在 CBC 模式下进行加密。因此,下面的数据会被加密以生成密文 $c$:
	\begin{equation}\label{eq:9-8}
		\boxed{\qquad\qquad\text{消息}\; m \qquad\qquad}\boxed{\quad\text{标签}\; t \quad}\boxed{\quad\text{填充}\; p \quad}
	\end{equation}
	
	请注意,标签 $t$ 并不保护填充的完整性。我们将利用这一点,用选择密文攻击来打破 CPA 安全性。
	\item $D_\mathrm{MtE}\big((k_\mathrm{e},k_\mathrm{m}),\,c\big)$:运行 CBC 解密以获得式 \ref{eq:9-8} 中的明文数据。然后,读取式 \ref{eq:9-8} 中最后一字节,并从数据中移除与其值相等长度的字节(即,如果最后一字节的值是 $3$,就移除该字节,再额外移除 $3$ 个字节),以完全移除填充 $p$。最后验证 MAC 标签,如果有效,就返回剩余的字节作为消息,否则就输出 $\mathsf{reject}$。
\end{itemize}
SSL 3.0 和 TLS 1.0 都使用了随机化 CBC 加密的一种有缺陷的变体,我们曾在练习 \ref{exer:5-13} 中讨论过这个问题,但它与我们这里的讨论无关。在这里,我们假设所使用的随机化 CBC 加密的实现是正确的。
\end{snote}

\begin{snote}[选择密文攻击。]
我们展示一种针对 $\mathcal{E}_\mathrm{MtE}$ 系统的选择密文攻击,它能让对手解密由其挑选的任何密文。考虑到这种攻击的存在,即使底层密码是 CPA 安全的,$\mathcal{E}_\mathrm{MtE}$ 也不一定是 AE 安全的。接下来,我们用 $(E,D)$ 表示用在 CBC 加密中的底层分组密码,它作用于 $16$ 字节的数据分组。

假设对手截获了一条对应于未知消息 $m$ 的有效密文 $c:=E_\mathrm{MtE}\big((k_\mathrm{e},k_\mathrm{m}),\,m)$。$m$ 的长度满足如下条件,即当 MAC 标签 $t$ 被添加到 $m$ 之后,$(m\,\Vert\,t)$ 的长度是 $16$ 字节的整数倍。这就意味着,在 CBC 加密的过程中,有一个完整的 $16$ 字节填充分组被添加到了消息后,且这个填充分组的最后一个字节的值是 $15$。因此,密文看起来就像下面这样:
\[
	c\quad = \quad
	\underbrace{\boxed{\quad c[0]\quad}}_{\text{IV}}\!
	\underbrace{\boxed{\quad c[1]\quad}\;\;\;\cdots\quad}_{m\,\text{的加密}}
	\underbrace{\quad\boxed{\; c[\ell-1]\;}}_{\text{加密标签}}\!
	\underbrace{\boxed{\quad c[\ell]\quad}}_{\text{加密填充}}
\]

我们首先证明,对手能够学到一些关于 $m[0]$($m$ 的第一个 $16$ 字节分组)的知识。这将打破 $\mathcal{E}_\mathrm{MtE}$ 的语义安全性。攻击者用 $c[1]$ 替换 $c$ 的最后一个分组,以准备一个选择密文查询 $\hat{c}$。也就是说:
\begin{equation}\label{eq:9-9}
\hat{c}\quad := \quad
\boxed{\quad c[0]\quad}
\boxed{\quad c[1]\quad}
\quad\cdots\quad
\boxed{\; c[\ell-1]\;}\!
\underbrace{\boxed{\quad c[1]\quad}}_{\text{加密填充?}}
\end{equation}
根据 CBC 解密的定义,解密 $\hat{c}$ 的最后一个分组可以得到 $16$ 字节的明文分组:
\[
v:=D\left(k_\mathrm{e},c[1]\right)\oplus c[\ell-1]=m[0]\oplus c[0]\oplus c[\ell-1]
\]
如果 $v$ 最后一字节的值是 $15$,那么在解密过程中,最后一个分组会被视为一个填充分组而被删除。剩下的序列是一个有效的消息-标签对,能够被正确地解密。如果 $v$ 最后一字节的值不是 $15$,那么对解密查询的应答很有可能就是 $\mathsf{reject}$。

换言之,如果对 $\hat{c}$ 的解密查询的应答不是 $\mathsf{reject}$,攻击者就能知道,$m[0]$ 的最后一个字节就等于 $u:=15\oplus c[0]\oplus c[\ell-1]$ 的最后一个字节。否则,攻击者也能知道 $m[0]$ 的最后一个字节不等于 $u$ 的最后一个字节。这就直接打破了 $\mathcal{E}_\mathrm{MtE}$ 的语义安全性:攻击者能够获得一些关于明文 $m$ 的知识。

读者可以用选择密文攻击中的对手(就像在攻击游戏 \ref{game:9-2} 中那样)重述上述攻击,我们将它留作一个启发性的练习。只要通过单次明文查询和单次密文查询,对手就能以 $1/256$ 的优势赢得游戏。这就证明 $\mathcal{E}_\mathrm{MtE}$ 是不安全的。

现在,假设攻击者使用另一个 IV 获得了对 $m$ 的另一个加密 $c'$。攻击者可以用密文 $c$ 和 $c'$ 来构造四个有用的选择密文查询:它可以用 $c[1]$ 或 $c'[1]$ 替换 $c$ 或 $c'$ 的最后一个分组。攻击者发出这四个密文查询,就可以了解到 $m[0]$ 的最后一字节是否等于以下四个值:
\[
15\oplus c[0]\oplus c[\ell-1],\qquad
15\oplus c[0]\oplus c'[\ell-1],\qquad
15\oplus c'[0]\oplus c[\ell-1],\qquad
15\oplus c'[0]\oplus c'[\ell-1]
\]
中某一个的最后一字节。如果这四个值各不相同,他们就能给攻击者提供四次学习 $m[0]$ 的最后一字节的机会。使用对消息 $m$ 的新加密多次重复这一过程,攻击者很快就能确定 $m[0]$ 的最后一字节。每次选择密文查询都能以 $1/256$ 的概率确定该字节。因此,平均来说,只要进行 $256$ 次选择密文查询,攻击者就能知道 $m[0]$ 的最后一字节的确切值。所以,攻击者可以打破语义安全性,更具体地说,它可以恢复明文的一个字节。接下来,假设攻击者能够请求对 $m$ 右移一比特后明文的加密,得到一条密文 $c_1$。将 $c1[1]$ 插入上一阶段密文(即对未移位的 $m$ 的加密)的最后一个分组,然后发出选择密文查询,攻击者就能揭示 $m[0]$ 的倒数第二个字节。对 $m$ 的每个字节重复该过程,就能够揭示 $m$ 的全部信息。下面我们表明,这能够导出一种针对 SSL 3.0 的真实攻击。
\end{snote}

\begin{snote}[彻底攻破 SSL 3.0。]
选择密文攻击似乎只在理论上可行,但事实上,它们经常被转化为极具破坏性的现实世界攻击。考虑一个网络浏览器和一个名为 \texttt{bank.com} 的受害网络服务器。在两者之间交换的信息使用 SSL 3.0 加密。浏览器和服务器之间共享一个被称为 cookie 的秘密,浏览器在它每一个发往 \texttt{bank.com} 的请求中都嵌入了这个 cookie。抽象地讲,浏览器发往 \texttt{bank.com} 的请求看起来就像:
\[
\boxed{\;\text{GET}\;\texttt{path}\quad\text{cookie:}\;\texttt{cookie}\;}
\]
其中,\texttt{path} 指浏览器向 \texttt{bank.com} 请求的资源的标识符。浏览器只会在它向 \texttt{bank.com} 发出的请求中插入该 cookie。

攻击者的目标是恢复秘密的 cookie。首先,它让浏览器访问 \texttt{attacker.com},在那里,它会向浏览器发送一个 JavaScript 程序。这个程序会迫使浏览器请求 \texttt{bank.com} 的资源 \texttt{/AA}。之所以请求这个路径,是为了确保消息和 MAC 的长度是分组长度($16$ 字节)的倍数,这是攻击所需要的。因此,浏览器向 \texttt{bank.com} 发送以下请求:
\begin{equation}\label{eq:9-10}
\boxed{\;\text{GET}\;\texttt{/AA}\quad\text{cookie:}\;\texttt{cookie}\;}
\end{equation}
该请求会被 SSL 3.0 加密。攻击者可以截获这个加密请求 $c$,并对 MtE 发起选择密文攻击,以了解 cookie 的一个字节。也就是说,攻击者会像式 \ref{eq:9-9} 那样准备好一个 $\hat{c}$,将 $\hat{c}$ 发送给 \texttt{bank.com},并查看\texttt{bank.com} 是否应答一个 SSL 报错信息。如果没有产生报错信息,攻击者就能知道 cookie 的一个字节。JavaScript 程序可以迫使浏览器重复发出式 \ref{eq:9-10} 中的请求,以给攻击者提供它所需的新鲜密文,直到最终揭示 cookie 的一个字节。

一旦对手知道了 cookie 的一个字节,它就可以让 JavaScript 程序向 \texttt{bank.com} 发出请求:
\[
\boxed{\;\text{GET}\;\texttt{/AAA}\quad\text{cookie:}\;\texttt{cookie}\;}
\]
以将 cookie 右移一个字节。这就为攻击者提供了一新的密文分组,不妨将其称作 $c_1[2]$,其中 cookie 被右移了一个字节。重新向服务器发送上一阶段的请求,但现在最后一个分组被替换为 $c_1[2]$,直到揭示 cookie 的第二个字节。对 cookie 的每一个字节重复这一过程,最终就能够揭示整个 cookie。

实际上,浏览器中的 JavaScript 程序能够为攻击者发动选择明文攻击提供充足的工具。而拦截网络中的数据包,对其进行修改并观察服务器的响应,就能为攻击者发动选择密文攻击提供充分的信息。这两者的结合就能完全打破 SSL 3.0 的 MtE 加密。

一个小细节是,每当 \texttt{bank.com} 应答一个 SSL 报错信息,SSL 会话都会关闭。但这并不构成问题:每当浏览器中的 JavaScript 程序向 \texttt{bank.com} 发起一个新的请求,都会自动启动一个新的 SSL 会话。因此,每个选择密文查询都是在不同的会话密钥下加密的,但这对攻击来说没有任何区别:每个查询都会检验 cookie 的一个字节是否等于一个已知的随机字节。只要有足够的查询,攻击者就能了解整个 cookie。
\end{snote}

\subsection{其他填充预言机攻击}\label{subsec:9-4-3}

TLS 1.0 是 SSL 3.0 的一个更新版本。它在填充中添加新的结构(见 \ref{subsec:5-4-4} 小节),以此来防御上一小节中的攻击:当填充 $p$ 个字节时,填充中所有字节的内容都会被置为 $p-1$。此外,在解密过程中,解密者需要检查所有填充字节的值是否都是正确的,如果不然就拒绝密文。这使得攻击者难以发动上一小节中介绍的攻击。当然,我们的目标只是想要表明 MtE 一般来说是不安全的,而 SSL 3.0 已经充分说明了这一点。

\begin{snote}[一种填充预言机计时攻击。]
尽管 TLS 1.0 加入了新的防御措施,但是对 MtE 解密的简陋实现仍然可能遭受攻击。假设这种实现是这样工作的:首先,它使用 CBC 解密所收到的密文;然后,它检查填充结构是否有效,如果不然,它就拒绝该密文;反之,如果填充是有效的,它就检查完整性标签,如果标签也是有效的,它就返回明文。在这个实现中,只有填充结构有效时,完整性标签才会被检查。这意味着,如果有一条密文包含无效的填充结构,而另一条密文包含有效的填充,但其标签是无效的,那么前者会比后者更快被拒绝。攻击者可以测量服务器应答一次选择密文查询所需的时间,如果很快就返回了一条 TLS 错误消息,它就能知道填充结构是无效的。否则,它至少能了解到这个填充是有效的。

这种计时信道被称为\textbf{填充预言机边信道(padding oracle side-channel)}。就像我们在 SSL 3.0 中所做的那样,我们也可以根据这种行为设计一个选择密文攻击,来完全解密一个秘密 cookie,读者可以将其当作一种很好的练习。为了了解如何实现这种攻击,不妨假设一个攻击者截获了一条加密 TLS 1.0 记录 $c$。令 $m$ 是 $c$ 的解密。假设攻击者想要检验 $m[2]$ 的最后一字节是否等于某个固定值 $b$。攻击者创建一个新的密文分组 $\hat{c}[1]:=c[1]\oplus B$,并将包含三个分组的记录 $\hat{c}=(c[0],\hat{c}[1],c[2])$ 发送给服务器。在对 $\hat{c}$ 进行 CBC 解密后,最后一个明文分组将是:
\[
\hat{m}[2]
:=\hat{c}[1]\oplus D\big(k,c[2]\big)
=m[2]\oplus B
\]
如果 $m[2]$ 的最后一字节等于 $b$,$\hat{m}[2]$ 的最后一比特就是 $0$,而这是一个有效的填充。服务器会尝试验证完整性标签,而这就会导致响应缓慢。如果 $m[2]$ 的最后一字节不等于 $b$,$\hat{m}[2]$ 就不以 $0$ 结尾,并且这很可能是一个无效的填充,而服务器很快就会给出应答。通过测量响应时间,攻击者就可以了解到 $m[2]$ 的最后一字节是否等于 $b$。就像我们对 SSL 3.0 所做的那样,用多个选择密文查询来重复这一过程,就可以揭示整个秘密 cookie。

有一种和用在 TLS 1.0 中的攻击类似,但更加复杂的 MtE 填充预言机计时攻击,被称作 Lucky13 \cite{al2013lucky}。想要在实现 TLS 1.0 解密时针对 Lucky13 攻击隐藏计时信息是相当有挑战性的。
\end{snote}

\begin{snote}[信息性报错消息。]
更糟糕的是,TLS 1.0 规范 \cite{dierks1999rfc2246} 中规定,当收到的密文因 MAC 验证错误而被拒绝时,服务器应发送一种特定类型(被称作 \texttt{bad\_record\_mac})的报错消息;而当密文因填充分组无效而被拒绝时,服务器应发送另一种类型(被称作 \texttt{decryption\_failed})的报错消息。理论上,这就能告诉攻击者,一条密文被拒绝到底是因为填充分组无效,还是因为完整性标签被损坏。这就可以使上述选择密文攻击成为可能,而且不需要借助于计时信息。唯一幸运的是,报错消息是加密的,攻击者无法看到错误代码。

尽管如此,这里仍然有一个重要的教训:当解密失败时,系统决不应该解释原因。应该发送一个通用的``\texttt{decryption\_failed}"代码,而不提供任何其他信息。这个问题在 TLS 1.1 中得到了承认和解决。此外,当解密失败时,无论失败的原因是什么,正确的实现总是应该花费相同的时间提供应答。
\end{snote}

\subsection{安全的先 MAC 后加密实例}\label{subsec:9-4-4}

\begin{theorem}\label{theo:9-3}
	
\end{theorem}

\subsection{是先加密后MAC还是先MAC后加密?}\label{subsec:9-4-5}
\section{带有关联数据的基于nonce的认证加密}\label{sec:9-5}

在这一节中,我们将扩展认证加密的语法来符合其常用的形式。首先,正如我们对加密和 MAC 所做的那样,我们会定义基于 nonce 的认证加密,它能使得加解密算法具有确定性,但是会将唯一的 nonce 作为输入。这种方法不但可以缩短密文,也能提高安全性。

其次,为了扩展加密算法,我们会为加密算法额外提供一条输入消息,称为\textbf{关联数据(associated data)},其完整性受到密文的保护,但机密性则不然。很多设置都会需要关联数据。例如,当在网络协议中加密数据包时,认证加密能够保护数据包主体,但数据包的首部必须以透明方式传输,以便网络能够将数据包传送到其预定的目的地。尽管如此,我们还是要确保数据包首部的完整性。此时,首部就可以作为关联数据被输入到加密算法中。

支持关联数据的密码被称为 \textbf{AD 密码}。基于 nonce 的 AD 密码的语法如下:
\[
c=E(k,m,d,{\scriptstyle\mathpzc{N}})
\]
其中,$c\in\mathcal{C}$ 是密文,$k\in\mathcal{K}$ 是密钥,$m\in\mathcal{M}$ 是消息,$d\in\mathcal{D}$ 是关联数据,${\scriptstyle\mathpzc{N}}\in\mathpzc{N}$ 是 nonce。此外,我们要求加密算法 $E$ 是确定性的。类似地,解密的语法变成了:
\[
D(k,c,d,{\scriptstyle\mathpzc{N}})
\]
它输出一条消息 $m$ 或者 $\mathsf{reject}$。我们称基于 nonce 的 AD 密码定义在 $(\mathcal{K},\mathcal{M},\mathcal{D},\mathcal{C},\mathpzc{N})$ 上。和之前一样,我们要求,\emph{只要被赋予相同的 nonce 和关联数据},由 $E$ 产生的密文就一定能被 $D$ 正确地解密。也就是说,对于所有的密钥 $k$,所有的消息 $m$,所有的关联数据 $d$,以及所有的 nonce 值 ${\scriptstyle\mathpzc{N}}\in\mathpzc{N}$,我们都有:
\[
D\big(k,\,E(k,m,d,{\scriptstyle\mathpzc{N}}),\,d,\,{\scriptstyle\mathpzc{N}}\big)=m
\]

如果输入到加密算法中的消息 $m$ 是一条空消息,那么密码 $(E,D)$ 本质上就变成了用于关联数据 $d$ 的 MAC 系统。

\begin{snote}[CPA 安全性。]
对于一个基于 nonce 的 AD 密码,假设在加密过程中\emph{没有 nonce 会被多次使用},如果该密码不会向窃听者泄露任何有用的信息,我们就称它是 CPA 安全的。基于 nonce 的 AD 密码的 CPA 安全性定义与基于 nonce 的密码的 CPA 安全性定义(见 \ref{sec:5-5} 节)基本一致。唯一的区别在加密查询方面。对于 $b=0,1$,在实验 $b$ 中,加密查询会按如下方式处理:
\begin{quote}
第 $i$ 次加密查询的内容是一对长度相同的消息 $m_{i0},m_{i1}\in\mathcal{M}$,关联数据 $d_i\in\mathcal{D}$,以及一个唯一的 nonce ${\scriptstyle\mathpzc{N}}_i\in\mathpzc{N}\setminus\{{\scriptstyle\mathpzc{N}}_1,\dots,{\scriptstyle\mathpzc{N}}_{i-1}\}$。

挑战者计算 $c_i\leftarrow E(k,m_{ib},d_i,{\scriptstyle\mathpzc{N}}_i)$,并将 $ci$ 发送给对手。
\end{quote}
除此之外,其他部分都与 \ref{sec:5-5} 节中的定义相同。请注意,关联数据 $d_i$ 是在对手的控制之下的,nonce 值 ${\scriptstyle\mathpzc{N}}_i$ 也是如此,只需要保证其唯一性即可。对于 $b=0,1$,令 $W_b$ 是对手 $\mathcal{A}$ 在实验 $b$ 中输出 $1$ 的事件。我们将 $\mathcal{A}$ 相对于 $\mathcal{E}$ 的优势定义为:
\[
\mathrm{nCPA}_\mathrm{ad}\mathsf{adv}[\mathcal{A},\mathcal{E}]
:=
\big\lvert
\Pr[W_0]-\Pr[W_1]
\big\rvert
\]
\end{snote}

\begin{definition}[CPA 安全性]\label{def:9-7}
如果对于所有的有效对手 $\mathcal{A}$,$\mathrm{nCPA}_\mathrm{ad}\mathsf{adv}[\mathcal{A},\mathcal{E}]$ 的值都可忽略不计,我们就称基于 nonce 的 AD 密码\textbf{对选择明文攻击是语义安全的(semantically secure against chosen plaintext attack)},简称为是 \textbf{CPA 安全的 (CPA-secure)}。
\end{definition}

\begin{snote}[密文完整性。]
如果一个攻击者能够在密钥 $k$ 下请求由它所选择的消息、关联数据和 nonce 的加密结果,但仍然无法输出一个能够被解密算法接受的新三元组 $(c,d,{\scriptstyle\mathpzc{N}})$,我们就称基于 nonce 的 AD 密码能够提供密文完整性。尽管如此,对手决不能使用曾经使用过的 nonce 发起加密查询。

更确切地说,我们按如下方式修改密文完整性游戏(攻击游戏 \ref{game:9-1}):
\end{snote}

\begin{game}[密文完整性]\label{game:9-3}
对于一个定义在 $(\mathcal{K},\mathcal{M},\mathcal{D},\mathcal{C},\mathpzc{N})$ 上的给定 AD 密码 $\mathcal{E}=(E,D)$ 和一个给定对手 $\mathcal{A}$,攻击游戏运行如下:
\begin{itemize}
	\item 挑战者随机选取一个 $k\overset{\rm R}\leftarrow\mathcal{K}$。
	\item $\mathcal{A}$ 对挑战者发起一系列查询。对于 $i=1,2,\dots$,第 $i$ 次查询包含一条消息 $m_i\in\mathcal{M}$,关联数据 $d_i\in\mathcal{D}$,以及一个先前未被使用过的 nonce ${\scriptstyle\mathpzc{N}}_i\in\mathpzc{N}\setminus\{{\scriptstyle\mathpzc{N}}_1,\dots,{\scriptstyle\mathpzc{N}}_{i-1}\}$。挑战者计算 $c_i\leftarrow E(k,m_i,d_i,{\scriptstyle\mathpzc{N}}_i)$,并将 $ci$ 交给 $\mathcal{A}$。
	\item 最终,$\mathcal{A}$ 输出一个候选三元组 $(c,d,{\scriptstyle\mathpzc{N}})$,其中 $c\in\mathcal{C}$,$d\in\mathcal{D}$,${\scriptstyle\mathpzc{N}}\in\mathpzc{N}$,且该三元组与它先前收到的所有三元组都不同,即:
		\[
		(c,d,{\scriptstyle\mathpzc{N}})\notin
		\big\{
			(c_1,d_1,{\scriptstyle\mathpzc{N}}_1),\;
			(c_2,d_2,{\scriptstyle\mathpzc{N}}_2),\;
			\cdots
		\big\}
		\]
\end{itemize}
如果 $D(k,c,d,{\scriptstyle\mathpzc{N}})\neq\mathsf{reject}$,我们就称 $\mathcal{A}$ 赢得游戏。我们将 $\mathcal{A}$ 相对于 $\mathcal{E}$ 的优势定义为 $\mathrm{nCI}_\mathrm{ad}\mathsf{adv}[\mathcal{A},\mathcal{E}]$,即 $\mathcal{A}$ 赢得游戏的概率。
\end{game}

\begin{definition}\label{def:9-8}
如果对于所有的有效对手 $\mathcal{A}$,$\mathrm{nCI}_\mathrm{ad}\mathsf{adv}[\mathcal{A},\mathcal{E}]$ 的值都可忽略不计,我们就称基于 nonce 的 AD 密码 $\mathcal{E}=(E,D)$ 具有\textbf{密文完整性}。
\end{definition}

\begin{snote}[认证加密。]
现在,我们可以为 AD 密码定义基于 nonce 的认证加密。我们将其称作\textbf{基于 nonce 的 AEAD 密码(nonce-based AEAD cipher)},其中,AEAD 指的是\textbf{带有关联数据的认证加密 (authenticated encryption with associated data)}。
\end{snote}

\begin{definition}\label{def:9-9}
如果基于 nonce 的 AD 密码 $\mathcal{E}=(E,D)$ 是 CPA 安全的,并且具有密文完整性,我们就称 $\mathcal{E}$ 提供认证加密,或将其简称为\textbf{基于 nonce 的 AEAD 密码(nonce-based AEAD cipher)}。
\end{definition}

\begin{snote}[通用先加密后 MAC 组合。]
我们可以将一个基于 nonce 的 CPA 安全密码 $(E,D)$(如 \ref{sec:5-5} 节)和一个基于 nonce 的安全 MAC $(S,V)$(如 \ref{sec:7-5} 节)组合起来,以构建一个基于 nonce 的 AEAD 密码 $\mathcal{E}=(E_\mathrm{EtM},D_\mathrm{EtM})$,方法如下:

\vspace*{10pt}

\hspace*{20pt} $E_\mathrm{EtM}\big((k_\mathrm{e},k_\mathrm{m}),\;m,d,{\scriptstyle\mathpzc{N}}\big)$
				\quad $:=$ \quad
				$c\leftarrow E(k_\mathrm{e},m,{\scriptstyle\mathpzc{N}})$,
				\quad
				$t\leftarrow S\big(k_\mathrm{m},\;(c,d),\;{\scriptstyle\mathpzc{N}}\big)$\\
\hspace*{187pt} 输出 $(c,t)$

\vspace*{5pt}

\hspace*{7pt} $D_\mathrm{EtM}\big((k_\mathrm{e},k_\mathrm{m}),\;(c,t),d,{\scriptstyle\mathpzc{N}}\big)$
				\quad $:=$ \quad
				如果 $V\big(k_\mathrm{m},\;(c,d),\;t,\;{\scriptstyle\mathpzc{N}}\big)=\mathsf{reject}$,则输出 $\mathsf{reject}$\\
\hspace*{187pt} 否则输出 $D(k_\mathrm{e},c,d,{\scriptstyle\mathpzc{N}})$

\vspace*{10pt}

\noindent
EtM 系统定义在 $(\mathcal{K}_\mathrm{e}\times\mathcal{K}_\mathrm{m},\mathcal{M},\mathcal{D},\mathcal{C}\times\mathcal{T},\mathpzc{N})$上。下面的定理将表明,$\mathcal{E}_\mathrm{EtM}$ 是一个安全的 AEAD 密码。
\end{snote}


\begin{theorem}\label{theo:9-4}
令 $\mathcal{E}=(E,D)$ 是一个基于 nonce 的密码,$\mathcal{I}=(S,V)$ 是一个基于 nonce 的 MAC 系统。假设 $\mathcal{E}$ 是 CPA 安全的,$\mathcal{I}$ 是一个安全的 MAC,则 $\mathcal{E}_\mathrm{EtM}$ 是一个基于 nonce 的 AEAD 密码。
\end{theorem}

定理 \ref{theo:9-4} 的证明与定理 \ref{theo:9-2} 基本相同。
\section{另一个变体:包含相关数据的CCA安全密码}\label{sec:9-6}
\section{案例研究:Galois计数器模式(GCM)}\label{sec:9-7}
\section{TLS 1.3 记录协议}
\section{针对 SSH 中非原子性解密的一种攻击}\label{sec:9-9}
\section{案例研究:802.11b WEP,一个千疮百孔的系统}\label{sec:9-10}

于 1999 年获批的 IEEE 802.11b 标准为短距离无线通信 (WiFi) 提供了一个协议。其安全性由 802.11b 数据帧封装的有线等效加密 (Wired Equivalent Privacy, WEP) 协议提供。WEP 的设计目标是提供能够媲美有线网络的数据隐私性。然而,WEP 在这方面彻头彻尾地失败了,并且它为我们提供了一个极好的案例,展示了一个孱弱的设计是如何导致灾难性的后果的。

当启用 WEP 时,无线网络中的所有成员都会共享一个长期密钥 $k$。该标准支持 $40$ 或 $128$ 比特密钥。$40$ 比特的版本符合标准起草时美国的出口限制。我们将使用以下符号来描述 WEP:
\begin{itemize}
	\item WEP 加密使用 RC4 流密码。我们用 $\mathrm{RC4}(s)$ 表示当给定种子 $s$ 时,RC4 所产生的伪随机序列。
	\item 我们用 $CRC(m)$ 表示消息 $m\in\{0,1\}^*$ 的 $32$ 比特 CRC 校验和。CRC 的实现细节与我们这里的讨论无关,读者只需将 CRC 视为从任意比特序列映射到 $\{0,1\}^{32}$ 的某个特定函数。
\end{itemize}

令 $m$ 是一个 802.11b 明文帧。$m$ 的前几个比特编码了 $m$ 的长度。为了加密一个 802.11b 帧 $m$,发送者选择一个 $24$ 比特的 IV,并计算:
\[
\begin{aligned}
& c\leftarrow\big(m\,\Vert\,\mathrm{CRC}(m)\big)\oplus\mathrm{RC4}(\mathrm{IV}\,\Vert\,k)\\
& c_\mathrm{full}\leftarrow(\mathrm{IV},\,c)
\end{aligned}
\]
WEP 的加密过程如图 \ref{fig:9-4} 所示。接收者首先计算 $c\oplus\mathrm{RC4}(\mathrm{IV}\,\Vert\,k)$ 来解密得到数对 $(m,s)$。如果 $s=\mathrm{CRC}(m)$,接收方就接受该帧,否则就拒绝它。

\begin{snote}[攻击 1:IV 碰撞。]
WEP 的设计者明白,流密码的密钥不应该被重复使用。因此,他们使用一个 $24$ 比特的 IV 来派生出每一帧的密钥 $k_\mathrm{f}:=\mathrm{IV}\,\Vert\,k$。然而,该标准并没有规定如何选择 IV,而许多实现都做得不够好。我们说,每当一个无线基站碰巧发送了两个帧,例如第 $i$ 帧和第 $j$ 帧,而它们都是由相同的 IV 加密得到的,IV 碰撞就会发生。由于 IV 是以透明方式发送的,所以窃听者很容易发现 IV 碰撞。此外,一旦 IV 碰撞发生,攻击者就可以使用 \ref{subsec:3-3-1} 小节中讨论的两次性密码本攻击来解密 $i$ 和 $j$ 这两个帧。

那么,IV 碰撞发生的概率到底有多大?根据生日悖论,如果为每一帧选择一个随机 IV,那么预期每隔 $\sqrt{2^{24}}=2^{12}=4096$ 帧,就会发生一次 IV 碰撞。由于每帧最长是 $1156$ 字节,那么平均每传输大约 4 MB 的数据,就会发生一次碰撞。

除此之外,我们也可以使用一个计数器来生成 IV。该实现会在发送 $2^{24}$ 帧后耗尽整个 IV 空间,对于一个满负荷工作的无线接入点来说,这只需要大约一天时间。更糟糕的是,一些使用计数器方法的无线网卡在开机时会将计数器重置为 $0$。因此,这些网卡会经常重复使用较小的 IV,而这会让流量极易遭受两次性密码本攻击。
\end{snote}

\begin{figure}
  \centering
  \tikzset{every picture/.style={line width=0.75pt}}

\begin{tikzpicture}[x=0.75pt,y=0.75pt,yscale=-1,xscale=1]


\draw   (70,0) -- (390,0) -- (390,40) -- (70,40) -- cycle ;
\draw   (390,0) -- (470,0) -- (470,40) -- (390,40) -- cycle ;
\draw   (70,55) -- (470,55) -- (470,95) -- (70,95) -- cycle ;
\draw   (20,125) -- (65,125) -- (65,165) -- (20,165) -- cycle ;
\draw  [fill={rgb, 255:red, 155; green, 155; blue, 155 }  ,fill opacity=0.5 ] (70,125) -- (470,125) -- (470,165) -- (70,165) -- cycle ;

\draw [line width=2.5]    (60,110) -- (480,110) ;

\draw   (40,48) .. controls (40,44.69) and (42.69,42) .. (46,42) .. controls (49.31,42) and (52,44.69) .. (52,48) .. controls (52,51.31) and (49.31,54) .. (46,54) .. controls (42.69,54) and (40,51.31) .. (40,48) -- cycle ;
\draw   (40,48) -- (52,48) ;
\draw   (46,42) -- (46,54) ;

\draw (220,20) node   [align=left] {明文载荷 $m$};
\draw (270,145) node   [align=left] {加密帧};
\draw (42.5,145) node   [align=left] {IV};
\draw (430,20) node    {$\mathrm{CRC}(m)$};
\draw (270,75) node    {$\mathrm{RC4}(\mathrm{IV}\,\Vert\,k)$};


\end{tikzpicture}
  \caption{WEP 加密}
  \label{fig:9-4}
\end{figure}

\begin{snote}[攻击 2:相关密钥。]
对 WEP 加密的一种更具破坏性的攻击来自对相关 RC4 密钥的使用。在第\ref{chap:3}章中,我们曾经解释,必须为每条加密消息选择一个新的、\emph{随机的}流密码密钥。然而,WEP 所使用的密钥 $1\,\Vert\,k$,$2\,\Vert\,k$,$\dots$ 都是密切相关的——它们都有相同的后缀 $k$。RC4 从来就不是为这种目的而设计的,事实上,它在这些情况下是完全不安全的。Fluhrer、Mantin 和 Shamir 表明,在发送了大约一百万个 WEP 帧后,窃听者就可以恢复整个长期密钥 $k$ \cite{fluhrer2001weaknesses}。该攻击由 Stubblefield、Ioannidis 和 Rubin 实现 \cite{stubblefield2004key},现在可被用于各种黑客工具,如 \textsc{WepCrack} 和 \textsc{AirSnort}。

应该使用一个 PRF 来生成每个帧的帧密钥,比如说将第 $i$ 帧的密钥设置为 $k_i:=F(k,\mathrm{IV})$——所产生的密钥和随机独立的密钥是无法区分的。当然,虽然这种方法确实可以防止相关密钥问题,但无法解决上面讨论的 IV 碰撞问题,以及接下来讨论的可塑性问题。
\end{snote}

\begin{snote}[攻击 3:易被控制性。]
回顾一下,WEP 试图使用 CRC 校验来为认证加密提供完整性。在某种意义上,WEP 使用的是先 MAC 后加密的方法,但它使用的是 CRC,而不是 MAC。我们表明,尽管包含加密步骤,但这种构造其实完全无法提供密文完整性。

针对它的攻击利用了 CRC 的线性特性。也就是说,对于某条消息 $m$,给定 $\mathrm{CRC}(m)$,我们很容易为任何的 $\Delta$ 计算出 $\mathrm{CRC}(m\oplus\delta)$。更确切地说,存在一个公共函数 $L$,对于任何的 $m$ 和 $\Delta\in\{0,1\}^\ell$,我们都有:
\[
\mathrm{CRC}(m\oplus\Delta)=\mathrm{CRC}(m)\oplus L(\Delta)
\]
这一属性使攻击者可以对 WEP 密文进行任意的修改,而不会被接收方发现。令 $c$ 是一条 WEP 密文,即:
\[
c=\big(m,\mathrm{CRC}(m)\big)\oplus\mathrm{RC4}(\mathrm{IV}\,\Vert\,k)
\]
因此,$c'$ 可以完全无误地被解密为 $m\oplus\Delta$。我们可以发现,给定对 $m$ 的加密,攻击者就可以为它所选择的任何一个 $\Delta$ 创建一个对 $m\oplus\Delta$ 的有效加密。我们已经在 \ref{subsec:3-3-2} 小节解释过,这很有可能会导致严重的攻击。
\end{snote}

\begin{snote}[攻击4:选择的密码文本攻击。]
该协议还容易受到一种被称为\textbf{断续攻击 (chop-chop)}的选择密文攻击,这种攻击可以让攻击者解密由其选择的加密帧。我们会在练习 \ref{exer:9-5} 中描述这种攻击的一个简单版本。
\end{snote}

\vspace*{-10pt}

\begin{snote}[攻击5: 拒绝服务。]
我们简单提一下,802.11b 曾遭受过许多严重的拒绝服务 (Denial of Service, DoS) 攻击。比如说,在 802.11b 中,一旦客户端结束了网络调用,无线客户端就会向无线基站发送一条``断联"的消息。这是为了让无线基站能够释放分配给该客户端的内存资源。不幸的是,``断联"信息不会经过任何认证,也就是说,任何人都可以代表其他人发送断联消息。一旦断联,受害者将需要几秒钟才能和基站重新建立连接。因此,只要每隔几秒钟就向基站发送一次``断联"消息,攻击者就可以阻止任何由其选定的计算机连接到无线网络中。一些 802.11b 工具,比如 \texttt{Void11},就实现了这种攻击。
\end{snote}

\begin{snote}[802.11i。]
在 802.11b WEP 协议失败后,一个名为 802.11i 的新标准在 2004 年得到批准。802.11i 使用一种称作 CCM 的先加密后 MAC 模式来提供认证加密。具体地说,CCM 使用(原生)CBC-MAC 进行 MAC,并使用计数器模式进行加密。802.11i 使用 AES 作为底层的 PRF 来实现这两者。后来,CCM 被 NIST 采纳为联邦标准 \cite{dworkin2004sp}。
\end{snote}
\section{案例研究:IPsec}
\section{一个有趣的应用:隐私信息检索}\label{sec:9-12}

待写。
\section{笔记}\label{sec:9-13}

对文献的引用有待补充。
\section{练习}\label{sec:9-14}

\begin{exercise}\label{exer:9-1}
\end{exercise}

\begin{exercise}\label{exer:9-2}
\end{exercise}

\begin{exercise}\label{exer:9-3}
\end{exercise}

\begin{exercise}\label{exer:9-4}
\end{exercise}

\begin{exercise}\label{exer:9-5}
\end{exercise}

\begin{exercise}\label{exer:9-6}
\end{exercise}

\begin{exercise}\label{exer:9-7}
\end{exercise}

\begin{exercise}\label{exer:9-8}
\end{exercise}

\begin{exercise}\label{exer:9-9}
\end{exercise}

\begin{exercise}\label{exer:9-10}
\end{exercise}

\begin{exercise}\label{exer:9-11}
\end{exercise}

\begin{exercise}\label{exer:9-12}
\end{exercise}

\begin{exercise}\label{exer:9-13}
\end{exercise}

\begin{exercise}\label{exer:9-14}
\end{exercise}

\begin{exercise}\label{exer:9-15}
\end{exercise}

\begin{exercise}\label{exer:9-16}
\end{exercise}

\begin{exercise}\label{exer:9-17}
\end{exercise}

\begin{exercise}\label{exer:9-18}
\end{exercise}

\begin{exercise}\label{exer:9-19}
\end{exercise}
