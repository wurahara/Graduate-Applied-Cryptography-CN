\chapter{消息完整性}\label{chap:6}

在前几章中,我们重点讨论了针对窃听对手的安全性。窃听攻击的对手有能力窃听传输的消息,但不能在途中篡改这些消息。我们表明,选择明文安全是抵御这种攻击所需的自然安全属性。

在本章中,我们将注意力转向主动对手。我们从\emph{消息完整性}的基本问题开始。Bob 从 Alice 那里收到一条消息 $m$,并想确认这条消息在传输过程中没有遭到篡改。我们将会设计一个机制,它能让 Alice 为消息 $m$ 计算一个简短的消息完整性标签 $t$,并将这个数对 $(m,t)$ 发送给 Bob,如图 \ref{fig:6-1} 所示。Bob 在收到数对后会检查标签 $t$,如果检查未能通过,他就拒绝该消息。反之,如果验证通过,Bob 就能确信消息在传输途中并未遭到篡改。

需要强调的是,本章中的消息本身不需要是秘密的。和前几章不同,在本章中,我们的目标并不是隐藏信息。相反,我们只关注消息的完整性。在第\ref{chap:9}章中,我们还会讨论同时保证消息的机密性和完整性这一更普遍的问题。有许多应用需要保证消息的完整性,但并不要求机密性。我们举两个例子。

\begin{example}\label{exmp:6-1}
考虑一下通过互联网传递金融新闻或股票报价的问题。虽然新闻本身是公共信息,但至关重要的一点是,必须要确保没有第三方在消息发送给用户的过程中篡改数据。在这里,消息的机密性是无关紧要的,但完整性却非常关键。我们构造想要确保,如果用户 Bob 拒绝所有带有无效的消息完整性标签的消息,那么攻击者就不能注入看起来合法的修改内容。有一点需要注意的是,攻击者仍然可以改变新闻报道到达 Bob 的顺序。例如,Bob 可能在看到 $1$ 号报告之前看到 $2$ 号报告。在某些情况下,这可能导致用户采取错误的行动。为了防止这种情况,新闻服务可能需要在每份报告中包含一个序列号,以便用户的机器可以缓冲报告,并确保用户总是能以正确的顺序看到新闻。
\end{example}

在本章中,我们只关注试图修改数据的攻击。我们不考虑拒绝服务 (Denial of Service, DoS) 攻击,即攻击者迟滞或阻止新闻到达用户。防御 DoS 攻击的常用方法是确保从发送方到接收方的网络中存在冗余路径,这样攻击者就很难封锁所有路径。我们在本章中不会讨论这些问题。

\begin{example}\label{exmp:6-2}
考虑一个在磁盘上存储数据的应用程序,如文字处理机或者邮件客户端。尽管应用程序的代码并不是机密的(甚至在公共领域中),但其完整性至关重要。在运行该程序之前,用户希望确保存储在磁盘上的代码没有被病毒篡改。为此,在第一次安装程序时,用户为代码计算出一个信息完整性标签,并将该标签与程序一起存储在磁盘上。之后,在每次启动应用程序之前,用户需要验证这个消息完整性标签。如果该标签仍然是有效的,用户就可以确信,自从该标签最初被生成以来,代码没有被修改过。显然,病毒可以同时覆盖应用程序的代码和完整性标签。尽管如此,我们的构造将确保没有病毒可以欺骗用户运行未经认证的代码。就像我们在例 \ref{exmp:6-1} 中所述的那样,攻击者可以交换两个认证程序,当用户启动程序 $A$ 时,它就运行程序 $B$。对此的标准防御方法是在可执行文件中包含程序名称。这样,当一个应用程序被启动时,系统可以向用户显示一个经过验证的应用程序名称。
\end{example}

那么,问题就在于,如何设计一个安全的消息完整性机制?我们首先需要论证以下基本原则:
\begin{quote}
在两个通信方之间提供消息的完整性需要发送方有一个对手不知道的密钥。
\end{quote}
没有密钥,想要确保消息完整性就是不可能的,因为对手拥有足够的信息来计算它选择的任意消息的完整性标签——它知道消息完整性算法是如何工作的,这就足以用来计算标签了。因此,所有的密码学消息完整性机制都需要一个对手不知道的密钥。在本章中,我们将假设发送方和接收方都将共享秘钥;但在后面的章节中,我们还将进一步放宽这一假设。

需要强调的是,不是为安全性而设计的通信协议中经常使用\emph{无密钥}的完整性机制。例如,以太网协议使用 CRC32 作为其消息完整性算法。这个算法是公开的,它将输出的 $32$ 比特标签嵌入到每个以太网帧中。TCP 协议使用一个无密钥的 $16$ 比特校验和,它会被嵌入到每个数据包中。我们强调,这些无密钥的完整性机制是为了检测\emph{随机的}传输错误而设计的,并不能针对恶意的篡改。上一段的论证表明,对手可以很容易地攻破这些机制,并产生看起来合法的消息。例如,在以太网的场景中,对手完全知道 CRC32 算法的工作原理,这样它就可以计算出任意消息的有效标签。然后,它就可以篡改以太网流量而不被发现。

\begin{figure}
  \centering
  \includegraphics[width=0.75\linewidth]{figures/chapter6/fig1.png}
  \caption{添加到消息上的短的消息完整性标签}
  \label{fig:6-1}
\end{figure}

\section{消息认证码的定义}
\section{MAC验证查询不会帮助攻击者}
\section{使用 PRF 构建 MAC}
\section{用于长消息的无前缀PRF}

\subsection{CBC无前缀安全PRF}

\subsection{级联无前缀安全PRF}\label{subsec:6-4-2}
\section{从无前缀安全PRF到完全安全PRF(方法1):加密PRF}
\section{从无前缀安全PRF到完全安全PRF(方法2):无前缀编码}\label{sec:6-6}

将无前缀安全的 PRF 转换为安全 PRF 的另一种方法是对 PRF 的输入进行编码,使得没有任何一条编码后的输入是另一条的前缀。我们定义以下术语:
\begin{itemize}
	\item 如果 $S\subseteq\mathcal{X}^{\leq\ell}$ 中的任何元素都不是其他元素的前缀,我们就称 $S$ 是一个\textbf{无前缀集 (prefix-free set)}。例如,如果 $(x_1,x_2,x_3)$ 属于一个无前缀集 $S$,那么 $x_1$ 和 $(x_1,x_2)$ 都不在 $S$ 中。
	\item 令 $\mathcal{X}^{\leq\ell}_{_{>0}}$ 表示$\mathcal{X}$ 上所有长度不超过 $\ell$ 的非空序列的集合。如果映射 $pf$ 是一个单射(即一对一函数),且 $pf$ 的像是一个无前缀集,我们就称 $pf:\mathcal{M}\to\mathcal{X}^{\leq\ell}_{_{>0}}$ 是一个\textbf{无前缀编码 (prefix-free encoding)}。
\end{itemize}

令 $PF$ 是一个定义在 $(\mathcal{K},\mathcal{X}^{\leq\ell},\mathcal{Y})$ 上的无前缀安全 PRF,且 $pf:\mathcal{M}\to\mathcal{X}^{\leq\ell}_{_{>0}}$ 是一个无前缀编码。定义派生的 PRF $F$ 为:
\[
F(k,m):=PF(k,pf(m))
\]
那么 $F$ 定义在 $(\mathcal{K},\mathcal{M},\mathcal{Y})$ 上。于是,我们就可以得到下面的定理。

\begin{theorem}\label{theo:6-8}
如果 $PF$ 是一个无前缀安全的 PRF,$pf$ 是一个无前缀编码,那么 $F$ 是一个安全的 PRF。
\end{theorem}

\subsection{无前缀编码}

为了使用定理 \ref{theo:6-8} 构造 PRF,我们介绍两种无前缀编码 $pf:\mathcal{M}\to\mathcal{X}^{\leq\ell}$。我们假设 $\mathcal{X}=\{0,1\}^n$,其中 $n$ 是某个整数。

\begin{snote}[方法 1:前置长度。]
设 $\mathcal{M}:=\mathcal{X}^{\leq\ell-1}$,并令 $m=(a_1,\dots,a_v)\in\mathcal{M}$。定义:
\[
pf(m):=(\langle v\rangle,a_1,\dots,a_v)\quad\in\mathcal{X}^{\leq\ell}_{_{>0}}
\]
其中 $\langle v\rangle\in\mathcal{X}$ 是 $v$ 的二进制表示,即 $m$ 的长度。我们假设 $\ell<2^n$,那么消息长度可以被编码为 $n$ 比特的二进制序列。

我们下面论证 $pf$ 是一个无前缀编码。首先,$pf$ 显然是一个单射。为了说明 $pf$ 的像是一个无前缀集,令 $pf(x)$ 和 $pf(y)$ 是 $pf$ 的像中的两个元素。如果 $pf(x)$ 和 $pf(y)$ 包含相同数量的分组,那么它们中的任何一个都不可能是另一个的真前缀。如果 $pf(x)$ 和 $pf(y)$ 包含不同数量的分组,它们就必须在第一个分组上就有所不同,此时它们也同样都不可能是另一个的真前缀。因此,$pf$ 是一个无前缀编码。

这种无前缀编码在实践中并不常用,因为这样所产生的 MAC 不是一个流式 MAC:使用这种 MAC 的应用程序必须提前把消息的长度承诺给 MAC。正如我们之前所说,这对于流媒体应用来说极不实用,因为在这种应用中,数据包的长度可能无法提前预知。
\end{snote}

\begin{snote}[方法 2:停止比特。]
设 $\mathcal{\bar{X}}:=\{0,1\}^{n-1}$,并令 $\mathcal{M}=\mathcal{\bar{X}}^{\leq\ell}_{_{>0}}$。对于 $m=(a_1,\dots,a_v)\in\mathcal{M}$,定义:
\[
pf(m):=\big((a_1||0),\,(a_2||0),\,\dots,\,(a_{v-1}||0),\,(a_v||1)\big)\quad\in\mathcal{X}^{\leq\ell}_{_{>0}}
\]
显然,$pf$ 是一个单射。为了说明 $pf$ 的像是一个无前缀集,令 $pf(x)$ 和 $pf(y)$ 是 $pf$ 的像中的两个元素。设 $v$ 是 $pf(x)$ 中分组的数量。如果 $pf(y)$ 包含 $v$ 个或更少的分组,那么 $pf(x)$ 必然不是 $pf(y)$ 的真前缀。如果 $pf(y)$ 包含多于 $v$ 个分组,那么 $pf(y)$ 中的 $v$ 号分组以 $0$ 结束,而 $pf(x)$ 中的 $v$ 号分组以 $1$ 结束。因此,$pf(x)$ 和 $pf(y)$ 在 $v$ 号分组上有所不同,所以 $pf(x)$ 不是 $pf(y)$ 的真前缀。

这种无前缀编码所产生的 MAC 是一个流式 MAC。然而,这种编码使 MAC 的长度增加了 $v$ 比特。当使用 CBC 或级联计算一个长消息的 MAC 时,这种编码将导致对底层 PRF(比如 AES)的额外计算。相对地,\ref{sec:6-5} 节中介绍的加密 PRF 方法只增加了一次对底层 PRF 的计算。例如,使用 ECBC-AES 和 $pf$ 对一兆字节 ($2^{20}$ 字节) 的消息计算 MAC,除了加密 PRF 方法所需要的,还需要对 AES 进行额外的 $511$ 次评估。在实践中,情况甚至可能更糟。由于计算机更喜欢字节对齐的数据,我们很可能需要在每个分组中附加一整个字节,而不是仅仅一个比特。那么,使用 ECBC-AES 和 $pf$ 对一兆字节的消息计算 MAC,将导致比加密 PRF 方法多出 $4096$ 次 AES 的计算,大约多 $6\%$ 的开销。
\end{snote}
\section{从无前缀安全 PRF 到完全安全 PRF(方法 3):CMAC}\label{sec:6-7}

上一节中介绍的两种无前缀编码方法都有问题。第一种方法产生的 MAC 是非流式的,而第二种方法对于长的消息需要对底层的 PRF 进行更多的计算。我们可以将无前缀编码进行随机化处理来进一步地改进设计。我们下面建立一个流式且安全的 PRF,除了底层的无前缀安全 PRF 之外,该设计不会引入任何额外的开销。图 \ref{fig:6-6} 展示了这样的 MAC,它比从加密 PRF 或者确定性编码得到的 MAC 都要优秀。这种方法被用于 NIST 的一种 MAC 标准,称为 CMAC,我们将在 \ref{sec:6-10} 节中介绍它。

\begin{figure}
  \centering
  \subfigure[应用于 CBC 的 \textit{rpf}]{
    \tikzset{every picture/.style={line width=0.75pt}}

\begin{tikzpicture}[x=0.75pt,y=0.75pt,yscale=-1,xscale=1]

\draw  [fill={rgb, 255:red, 255; green, 255; blue, 255 }  ,fill opacity=1 ][line width=1.2] [general shadow={fill=black,shadow xshift=2.25pt,shadow yshift=-2.25pt}] (10,90) -- (60,90) -- (60,140) -- (10,140) -- cycle ;
\draw  [fill={rgb, 255:red, 255; green, 255; blue, 255 }  ,fill opacity=1 ][line width=1.2] [general shadow={fill=black,shadow xshift=2.25pt,shadow yshift=-2.25pt}] (0,0) -- (70,0) -- (70,20) -- (0,20) -- cycle ;
\draw  [fill={rgb, 255:red, 255; green, 255; blue, 255 }  ,fill opacity=1 ][line width=1.2] [general shadow={fill=black,shadow xshift=2.25pt,shadow yshift=-2.25pt}] (100,0) -- (170,0) -- (170,20) -- (100,20) -- cycle ;
\draw  [fill={rgb, 255:red, 255; green, 255; blue, 255 }  ,fill opacity=1 ][line width=1.2] [general shadow={fill=black,shadow xshift=2.25pt,shadow yshift=-2.25pt}] (200,0) -- (270,0) -- (270,20) -- (200,20) -- cycle ;
\draw  [fill={rgb, 255:red, 255; green, 255; blue, 255 }  ,fill opacity=1 ][line width=1.2] [general shadow={fill=black,shadow xshift=2.25pt,shadow yshift=-2.25pt}] (330,0) -- (400,0) -- (400,20) -- (330,20) -- cycle ;
\draw  [fill={rgb, 255:red, 255; green, 255; blue, 255 }  ,fill opacity=1 ][line width=1.2] [general shadow={fill=black,shadow xshift=2.25pt,shadow yshift=-2.25pt}] (110,90) -- (160,90) -- (160,140) -- (110,140) -- cycle ;
\draw  [fill={rgb, 255:red, 255; green, 255; blue, 255 }  ,fill opacity=1 ][line width=1.2] [general shadow={fill=black,shadow xshift=2.25pt,shadow yshift=-2.25pt}] (210,90) -- (260,90) -- (260,140) -- (210,140) -- cycle ;
\draw  [fill={rgb, 255:red, 255; green, 255; blue, 255 }  ,fill opacity=1 ][line width=1.2] [general shadow={fill=black,shadow xshift=2.25pt,shadow yshift=-2.25pt}] (340,90) -- (390,90) -- (390,140) -- (340,140) -- cycle ;

\draw  [->]  (35,20) -- (35,90) ;
\draw  [->]  (330,55) -- (360,55) ;
\draw  [->]  (135,20) -- (135,50) ;
\draw  [->]  (135,60) -- (135,90) ;
\draw  [->]  (235,20) -- (235,50) ;
\draw  [->]  (365,20) -- (365,50) ;
\draw  [->]  (235,60) -- (235,90) ;
\draw  [->]  (365,60) -- (365,90) ;
\draw    (260,115) -- (285,115) ;
\draw  [->]  (390,115) -- (440,115) ;
\draw  [<-]  (370,55) -- (420,55) ;
\draw  [->]  (60,115) -- (85,115) -- (85,55) -- (130,55) ;
\draw  [->]  (160,115) -- (185,115) -- (185,55) -- (230,55) ;
\draw    [dash pattern={on 0.84pt off 2.51pt}]  (285,115) -- (310,115) -- (310,55) -- (330,55) ;


\draw (35,115) node    {$F( k,\cdot )$};
\draw (135,115) node    {$F( k,\cdot )$};
\draw (235,115) node    {$F( k,\cdot )$};
\draw (365,115) node    {$F( k,\cdot )$};
\draw (135,55) node  [font=\large]  {$\oplus $};
\draw (235,55) node  [font=\large]  {$\oplus $};
\draw (365,55) node  [font=\large]  {$\oplus $};
\draw (300,10) node    {$\cdots $};
\draw (35,10) node    {$a_{1}$};
\draw (135,10) node    {$a_{2}$};
\draw (235,10) node    {$a_{3}$};
\draw (365,10) node    {$a_{\ell }$};
\draw (440,111.6) node [anchor=south] [inner sep=0.75pt]    {$tag$};
\draw (418,51.6) node [anchor=south east] [inner sep=0.75pt]    {$k_{1}$};


\end{tikzpicture}
  }
  
  \,
  
  \,
  
  \subfigure[应用于级联的 \textit{rpf}]{
    \tikzset{every picture/.style={line width=0.75pt}}

\begin{tikzpicture}[x=0.75pt,y=0.75pt,yscale=-1,xscale=1]

\draw  [fill={rgb, 255:red, 255; green, 255; blue, 255 }  ,fill opacity=1 ][line width=1.2] [general shadow={fill=black,shadow xshift=2.25pt,shadow yshift=-2.25pt}] (50,90) -- (100,90) -- (100,140) -- (50,140) -- cycle ;
\draw  [fill={rgb, 255:red, 255; green, 255; blue, 255 }  ,fill opacity=1 ][line width=1.2] [general shadow={fill=black,shadow xshift=2.25pt,shadow yshift=-2.25pt}] (40,0) -- (110,0) -- (110,20) -- (40,20) -- cycle ;
\draw  [fill={rgb, 255:red, 255; green, 255; blue, 255 }  ,fill opacity=1 ][line width=1.2] [general shadow={fill=black,shadow xshift=2.25pt,shadow yshift=-2.25pt}] (140,0) -- (210,0) -- (210,20) -- (140,20) -- cycle ;
\draw  [fill={rgb, 255:red, 255; green, 255; blue, 255 }  ,fill opacity=1 ][line width=1.2] [general shadow={fill=black,shadow xshift=2.25pt,shadow yshift=-2.25pt}] (240,0) -- (310,0) -- (310,20) -- (240,20) -- cycle ;
\draw  [fill={rgb, 255:red, 255; green, 255; blue, 255 }  ,fill opacity=1 ][line width=1.2] [general shadow={fill=black,shadow xshift=2.25pt,shadow yshift=-2.25pt}] (370,0) -- (440,0) -- (440,20) -- (370,20) -- cycle ;
\draw  [fill={rgb, 255:red, 255; green, 255; blue, 255 }  ,fill opacity=1 ][line width=1.2] [general shadow={fill=black,shadow xshift=2.25pt,shadow yshift=-2.25pt}] (150,90) -- (200,90) -- (200,140) -- (150,140) -- cycle ;
\draw  [fill={rgb, 255:red, 255; green, 255; blue, 255 }  ,fill opacity=1 ][line width=1.2] [general shadow={fill=black,shadow xshift=2.25pt,shadow yshift=-2.25pt}] (250,90) -- (300,90) -- (300,140) -- (250,140) -- cycle ;
\draw  [fill={rgb, 255:red, 255; green, 255; blue, 255 }  ,fill opacity=1 ][line width=1.2] [general shadow={fill=black,shadow xshift=2.25pt,shadow yshift=-2.25pt}] (380,90) -- (430,90) -- (430,140) -- (380,140) -- cycle ;

\draw  [->]  (75,20) -- (75,89) ;
\draw  [->]  (175,20) -- (175,89) ;
\draw  [->]  (275,20) -- (275,89) ;
\draw  [->]  (405,20) -- (405,49) ;
\draw  [->]  (405,60) -- (405,89) ;

\draw  [->]  (0,115) -- (49,115) ;
\draw  [->]  (100,115) -- (149,115) ;
\draw  [->]  (200,115) -- (249,115) ;
\draw  [->]  (355,115) -- (379,115) ;
\draw  [->]  (430,115) -- (480,115) ;
\draw        (300,115) -- (325,115) ;

\draw  [<-]  (411,55) -- (460,55) ;

\draw   (55,115) -- (50,118.5) -- (50,111.5) -- cycle ;
\draw   (155,115) -- (150,118.5) -- (150,111.5) -- cycle ;
\draw   (255,115) -- (250,118.5) -- (250,111.5) -- cycle ;
\draw   (385,115) -- (380,118.5) -- (380,111.5) -- cycle ;

\draw  [dash pattern={on 0.84pt off 2.51pt}]  (325,115) -- (355,115) ;

\draw (75,115) node    {$F$};
\draw (175,115) node    {$F$};
\draw (275,115) node    {$F$};
\draw (405,115) node    {$F$};
\draw (340,10) node    {$\cdots $};
\draw (75,10) node    {$a_{1}$};
\draw (175,10) node    {$a_{2}$};
\draw (275,10) node    {$a_{3}$};
\draw (405,10) node    {$a_{\ell }$};
\draw (480,111.6) node [anchor=south] [inner sep=0.75pt]    {$\mathit{tag}$};
\draw (2,111.6) node [anchor=south west] [inner sep=0.75pt]    {$k$};
\draw (405,55) node  [font=\large]  {$\oplus $};
\draw (458,51.6) node [anchor=south east] [inner sep=0.75pt]    {$k_{1}$};


\end{tikzpicture}
  }
  \caption{使用随机无前缀编码的安全 PRF}
  \label{fig:6-6}
\end{figure}

首先,为了叙述方便,我们引入一些符号:

\begin{definition}\label{def:6-5}
对于两个序列 $x,y\in\mathcal{X}^{\leq\ell}$,如果 $x$ 是 $y$ 的一个前缀,或者 $y$ 是 $x$ 的一个前缀,我们就记 $x\sim y$。
\end{definition}

\begin{definition}\label{def:6-6}
令 $\epsilon$ 是一个实数,满足 $0\leq\epsilon\leq1$。一个\textbf{随机性 $\epsilon$-无前缀 (randomized $\epsilon$-prefix-free)} 编码是一个函数 $rpf:\mathcal{K}\times\mathcal{M}\to\mathcal{X}^{\leq\ell}_{_{>0}}$,使得对于任意 $m_0,m_1\in\mathcal{M}$,$m_0\neq m_1$,我们都有:
\[
\Pr\big[rpf(k,m_0)\sim rpf(k,m_1)\big]\leq\epsilon
\]
这里的概率取决于从 $\mathcal{K}$ 中随机选取的 $k$。
\end{definition}

\noindent
请注意,$rpf(k,\cdot)$ 的像不需要是一个无前缀集合。然而,在不知道 $k$ 的情况下,很难找到 $m_0,m_1\in\mathcal{M}$ 使得 $rpf(k,m_0)$ 是 $rpf(k,m_1)$ 的真前缀(或者反之)。函数 $rpf(k,\cdot)$ 甚至不需要是一个单射。

\begin{snote}[一个简单的 $rpf$。]
令 $\mathcal{K}:=\mathcal{X}$,$\mathcal{M}:=\mathcal{X}^{\leq\ell}_{_{>0}}$。定义:
\[
rpf(k,(a_1,\dots,a_v))=(a_1,\dots,a_{v-1},(a_v\oplus k))\in\mathcal{X}^{\leq\ell}_{_{>0}}
\]
不难发现 $rpf$ 是一个随机性 $({1}/{|\mathcal{X}|})$-无前缀编码。令 $m_0,m_1\in\mathcal{M}$,并且 $m_0\neq m_1$。假设 $|m_0|=|m_1|$。那么很明显,对于所有 $k$ 的选择,$rpf(k,m_0)$ 和 $rpf(k,m_1)$ 都是长度相同的不同序列,所以两者都不可能是对方的前缀。接下来,假设 $|m_0|<|m_1|$。如果 $v:=|rpf(k,m_0)|$,那么很显然,当且仅当:
\[
m_0[v-1]\oplus k=m_1[v-1]
\]
时,$rpf(k,m_0)$ 才是 $rpf(k,m_1)$ 的一个真前缀。然而,在随机选择 $k$ 的情况下,上式成立的概率是 ${1}/{|\mathcal{X}|}$,这符合要求。最后,对于 $|m_0|>|m_1|$ 的情况,可以用一个对称的论证来处理。
\end{snote}

\begin{snote}[使用 $rpf$。]
令 $PF$ 是一个定义在 $(\mathcal{K},\mathcal{X}^{\leq\ell},\mathcal{Y})$ 上的无前缀安全 PRF,且 $rpf:\mathcal{K}_1\times\mathcal{M}\to\mathcal{X}^{\leq\ell}_{_{>0}}$ 是一个随机化的无前缀编码。定义派生出的 PRF $F$ 为:
\begin{equation}\label{eq:6-21}
F\big((k,k_1),\,m\big):=PF\big(k,\,rpf(k_1,m)\big)
\end{equation}
那么,$F$ 就定义在 $(\mathcal{K}\times\mathcal{K}_1,\mathcal{M},\mathcal{Y})$ 上。我们可以得到以下定理,它与定理 \ref{theo:6-8} 类似。
\end{snote}

\begin{theorem}\label{theo:6-9}
如果 $PF$ 是一个无前缀安全的 PRF,$\epsilon$ 可忽略不计,并且 $rpf$ 是一个随机性 $\epsilon$-无前缀编码,则式 \ref{eq:6-21} 中定义的 $F$ 是一个安全的 PRF。
\begin{quote}
特别地,对于每个按照攻击游戏 \ref{game:4-2} 攻击 $F$ 的 PRF 对手 $\mathcal{A}$,假设它最多能够向其挑战者发起 $Q$ 次查询,则必然存在两个按照攻击游戏 \ref{game:4-2} 攻击 $PF$ 的无前缀 PRF 对手 $\mathcal{B}_1$ 和 $\mathcal{B}_2$,使得 $\mathcal{B}_1$ 和 $\mathcal{B}_2$ 都是围绕 $\mathcal{A}$ 的基本包装器,满足:
\end{quote}
\begin{equation}\label{eq:6-22}
{\rm PRF}\mathsf{adv}[\mathcal{A},F]\leq
{\rm PRF^{pf}}\mathsf{adv}[\mathcal{B}_1,PF]+{\rm PRF^{pf}}\mathsf{adv}[\mathcal{B}_2,PF]+{Q^2\epsilon}/{2}
\end{equation}
\end{theorem}

\begin{proof}[证明思路]
如果对手对于 $F$ 的输入集能够产生对于 $PF$ 的无前缀输入集,那么对手看到的就只是一些看起来很随机的输出。此外,如果对手看到的是随机的输出,它就没有得到关于 $rpf$ 的密钥 $k_1$ 的任何有效信息,这就确保 $PF$ 的输入集确实(以压倒性的概率)是无前缀的。不幸的是,上述思路会导致一个循环论证。我们将在下面的证明中介绍打破这种循环论证的方法。
\end{proof}

\begin{proof}
不失一般性,我们假设 $\mathcal{A}$ 从未发出过两次相同的查询。我们将证明组织成由三个游戏构成的一个序列。对于 $j=0,1,2$,我们令 $W_j$ 是 $\mathcal{A}$ 在游戏 $j$ 结束时输出 $1$ 的事件。

\vspace{5pt}

\noindent\textbf{游戏$\mathbf{0}$}。
在 PRF 攻击游戏 \ref{game:4-2} 的实验 $0$ 中,挑战者就 $F$ 进行如下操作:

\vspace{5pt}

\hspace*{5pt} 选取 $k\overset{\rm R}\leftarrow\mathcal{K}$,$k_1\overset{\rm R}\leftarrow\mathcal{K}_1$\\
\hspace*{26pt} 当从 $\mathcal{A}$ 处收到签名查询 $m_i\in\mathcal{M}$ ($i=1,2,\dots$) 时:\\
\hspace*{50pt} 令 $x_i\leftarrow rpf(k_1,m_i)\in\mathcal{X}^{\leq\ell}_{_{>0}}$\\
\hspace*{50pt} 令 $y_i\leftarrow PF(k,x_i)$\\
\hspace*{50pt} 将 $y_i$ 发送给 $\mathcal{A}$

\vspace{5pt}

\noindent\textbf{游戏$\mathbf{1}$}。
我们修改游戏 $0$ 中的挑战者,以确保对 $PF$ 的所有查询都是无前缀的。回顾一下符号 $x\sim y$,它表示 $x$ 是 $y$ 的前缀,或者 $y$ 是 $x$ 的前缀。

\vspace{5pt}

\hspace*{5pt} 选取 $k\overset{\rm R}\leftarrow\mathcal{K}$,$k_1\overset{\rm R}\leftarrow\mathcal{K}_1$,$r_1,\dots,r_Q\overset{\rm R}\leftarrow\mathcal{Y}$\\
\hspace*{26pt} 当从 $\mathcal{A}$ 处收到签名查询 $m_i\in\mathcal{M}$ ($i=1,2,\dots$) 时:\\
\hspace*{50pt} 令 $x_i\leftarrow rpf(k_1,m_i)\in\mathcal{X}^{\leq\ell}_{_{>0}}$\\
\hspace*{1pt} ($1$)
\hspace*{28.5pt} 如果存在某个 $j<i$ 使得 $x_i\sim x_j$:\\
\hspace*{75pt} 则令 $y_i\leftarrow r_i$\\
\hspace*{1pt} ($2$)
\hspace*{53.5pt} 否则令 $y_i\leftarrow PF(k,x_i)$\\ % 21.5pt deducted
\hspace*{50pt} 将 $y_i$ 发送给 $\mathcal{A}$

\vspace{5pt}

\noindent
令 $Z_1$ 为在游戏 $1$ 中的某个时刻,行 ($1$) 中的条件成立的事件。显然,在事件 $Z_1$ 发生之前,游戏 $1$ 和游戏 $2$ 的进程是完全相同的;特别地,当且仅当 $W_1\land\bar{Z}_1$ 发生时,$W_0\land\bar{Z}_1$ 才会发生。因此,基于差分引理(定理 \ref{theo:4-7}),我们得到:
\begin{equation}\label{eq:6-23}
\big\lvert\Pr[W_1]-\Pr[W_0]\big\rvert\leq\Pr[Z_1]
\end{equation}
不幸的是,在这一点上,我们还不能完全约束 $\Pr[Z_1]$。当分析进行到这一阶段时,我们还不能说明对 $PF$ 的评估在行 ($2$) 不会泄漏关于 $k_1$ 的信息。这些信息可能可以帮助 $\mathcal{A}$ 促使事件 $Z_1$ 发生。这就是我们上面提到的循环论证的问题。为了客服这一问题,我们把对 $Z_1$ 的分析推迟到下一个游戏。

\vspace{5pt}

\noindent\textbf{游戏$\mathbf{2}$}。
现在,与往常一样,我们打出``PRF牌",用一个真随机函数取代函数 $PF(k,\cdot)$。这是合理的,因为从构造上来看,在游戏 $1$ 中,$PF(k,\cdot)$ 的输入集是无前缀的。为了实现这一修改,我们可以简单地将标有 ($2$) 的一行替换为:

\vspace{5pt}

\hspace*{-19pt} ($2$)
\hspace*{53.5pt} 否则令 $y_i\leftarrow PF(k,x_i)$

\vspace{5pt}

\noindent
做了这样的修改后,我们可以看到,无论行 ($1$) 中的条件是否成立,$y_i$ 都会被分配一个随机值 $r_i$。

现在,令 $Z_2$ 表示在游戏 $2$ 中的某个时刻,行 ($1$) 中的条件成立的事件。不难得到,对于有效的无前缀 PRF 对手 $\mathcal{B}_1$ 和 $\mathcal{B}_2$,有:
\begin{equation}\label{eq:6-24}
\big\lvert\Pr[Z_1]-\Pr[Z_2]\big\rvert\leq{\rm PRF^{pf}}\mathsf{adv}[\mathcal{B}_1,F]
\end{equation}
和:
\begin{equation}\label{eq:6-25}
\big\lvert\Pr[W_1]-\Pr[W_2]\big\rvert\leq{\rm PRF^{pf}}\mathsf{adv}[\mathcal{B}_2,F]
\end{equation}
成立。这两个对手基本上是一样的,只是当行 ($1$) 中的条件成立时,$\mathcal{B}_1$ 就会输出 $1$,而 $\mathcal{B}_2$ 则会输出 $\mathcal{A}$ 所输出的任何结果。

此外,在游戏 $2$ 中,$k_1$ 的值显然与 $\mathcal{A}$ 的查询无关,因此利用 $rpf$ 的 $\epsilon$-无前缀属性,根据联合约束,我们可以得到:
\begin{equation}\label{eq:6-26}
\Pr[Z_2]\leq{Q^2\epsilon}/{2}
\end{equation}

最后,对 $\mathcal{A}$ 来说,游戏 $2$ 完美地模拟了 ${\rm Funs}[\mathcal{M},\mathcal{Y}]$ 中的一个随机函数。因此,游戏 $2$ 与就 $F$ 的攻击游戏 \ref{game:4-2} 中的实验 $1$ 是相同的,因此:
\begin{equation}\label{eq:6-27}
\big\lvert\Pr[W_0]-\Pr[W_2]\big\rvert={\rm PRF}\mathsf{adv}[\mathcal{A},F]
\end{equation}
现在,结合式 \ref{eq:6-23},\ref{eq:6-24},\ref{eq:6-25},\ref{eq:6-26} 和 \ref{eq:6-27},该定理得证。
\end{proof}
\section{将按分组PRF转化为按比特PRF}\label{sec:6-8}

到目前为止,我们为 $\mathcal{X}^{\leq\ell}$ 上的变长输入构建了一些 PRF。通常 $\mathcal{X}=\{0,1\}^n$,其中 $n$ 是底层 PRF 的分组大小,CBC 或者级联构造都基于该参数构建(例如,对于 AES,$n=128$)。迄今为止,我们所介绍的所有 MAC 都被设计用来验证长度为 $n$ 比特的倍数的消息。

在本节中,我们将展示如何将这些 PRF 转换为针对任意长度消息的 PRF。也就是说,给定一个针对 $\mathcal{X}^{\leq\ell}$ 上的消息的 PRF,我们想要构建一个针对 $\{0,1\}^{\leq n\ell}$ 上的消息的 PRF。

令 $F$ 是一个接受 $\mathcal{X}^{\leq l+1}$ 中输入的 PRF。令 $inj:\{0,1\}^{\leq n\ell}\to\mathcal{X}^{\leq\ell+1}$ 是一个单射(即一对一)函数。定义派生出的 PRF $F_{\rm bit}$ 为:
\[
F_{\rm bit}(k,x):=F(k,\,inj(x))
\]
于是,我们可以得到下面的一个三段式定理。

\begin{theorem}\label{theo:6-10}
如果 $F$ 是一个定义在 $(\mathcal{K},\mathcal{X}^{\leq\ell+1},\mathcal{Y})$ 上的安全的 PRF,则 $F_{\rm bit}$ 是一个定义在 $(\mathcal{K},\{0,1\}^{\leq n\ell},\mathcal{Y})$ 上的安全的 PRF。
\end{theorem}

\begin{figure}
  \centering
  

\tikzset{every picture/.style={line width=0.75pt}} %set default line width to 0.75pt        

\begin{tikzpicture}[x=0.75pt,y=0.75pt,yscale=-1,xscale=1]
%uncomment if require: \path (0,102); %set diagram left start at 0, and has height of 102

%Shape: Rectangle [id:dp6664083677098316] 
\draw  [line width=1.2]  (70,0) -- (150,0) -- (150,20) -- (70,20) -- cycle ;
%Shape: Rectangle [id:dp9556880837925956] 
\draw  [line width=1.2]  (150,0) -- (180,0) -- (180,20) -- (150,20) -- cycle ;
%Shape: Rectangle [id:dp9642427275639509] 
\draw  [line width=1.2]  (70,60) -- (150,60) -- (150,80) -- (70,80) -- cycle ;
%Shape: Rectangle [id:dp38266360352065854] 
\draw  [line width=1.2]  (150,60) -- (230,60) -- (230,80) -- (150,80) -- cycle ;
%Shape: Rectangle [id:dp5384874063810199] 
\draw  [line width=1.2]  (280,60) -- (360,60) -- (360,80) -- (280,80) -- cycle ;
%Shape: Rectangle [id:dp12198869664334477] 
\draw  [line width=1.2]  (375,60) -- (455,60) -- (455,80) -- (375,80) -- cycle ;
%Shape: Rectangle [id:dp07210885232506747] 
\draw  [line width=1.2]  (470,60) -- (550,60) -- (550,80) -- (470,80) -- cycle ;
%Shape: Rectangle [id:dp30140895463818707] 
\draw  [line width=1.2]  (280,0) -- (360,0) -- (360,20) -- (280,20) -- cycle ;
%Shape: Rectangle [id:dp3582436073878503] 
\draw  [line width=1.2]  (375,0) -- (405,0) -- (405,20) -- (375,20) -- cycle ;
%Shape: Rectangle [id:dp9390521369303302] 
\draw  [line width=1.2]  (405,0) -- (455,0) -- (455,20) -- (405,20) -- cycle ;
%Straight Lines [id:da5225518130167728] 
\draw  [->]  (240,70) -- (268,70) ;
%Straight Lines [id:da4183424986140616] 
\draw  [->] (240,10) -- (268,10) ;

% Text Node
\draw (110,10) node    {$a_{1}$};
% Text Node
\draw (165,10) node    {$a_{2}$};
% Text Node
\draw (320,10) node    {$a_{1}$};
% Text Node
\draw (390,10) node    {$a_{2}$};
% Text Node
\draw (320,70) node    {$a_{1}$};
% Text Node
\draw (110,70) node    {$a_{1}$};
% Text Node
\draw (190,70) node    {$a_{2}$};
% Text Node
\draw (415,70) node    {$a_{2}$};
% Text Node
\draw (430,10) node    {$1000$};
% Text Node
\draw (510,70) node    {$1000000$};
% Text Node
\draw (2,10) node [anchor=west] [inner sep=0.75pt]   [align=left] {情况 $1$:};
% Text Node
\draw (2,70) node [anchor=west] [inner sep=0.75pt]   [align=left] {情况 $2$:};


\end{tikzpicture}
  \caption{一个单射函数 $inj:\{0,1\}^{\leq n\ell}\to\mathcal{X}^{\leq\ell+1}$}
  \label{fig:6-7}
\end{figure}

\begin{snote}[一个单射函数。]
对于 $\mathcal{X}:=\{0,1\}^n$,一个从 $\{0,1\}^{\leq n\ell}$ 到 $\mathcal{X}^{\leq\ell+1}$ 的单射 $inj$ 的标准例子如下。如果输入消息的长度不是 $n$ 的倍数,那么 $inj$ 会添加 $100\dots00$ 来填充消息,使得它的长度成为 $n$ 的倍数,图 \ref{fig:6-7} 以图像的形式直观地刻画了这一点。更准确地说,该函数的工作原理如下:

\vspace{5pt}

\hspace*{5pt} 输入:$m\in\{0,1\}^{\leq nl}$

\vspace{3pt}

\hspace*{5pt} 令 $u\leftarrow|m|\;\mathrm{mod}\;n$,$m'\leftarrow m\,\Vert\,1\,\Vert\,0^{n-u-1}$\\
\hspace*{26pt} 输出 $m'$ 作为 $n$ 比特消息分组的一个序列

\vspace{5pt}

\noindent
为了说明 $inj$ 是一个单射,我们先表明它是可逆的。给定 $y\leftarrow inj(m)$,从右到左扫描 $y$,并删除所有 $0$ 和初次见到的 $1$,那么剩下的序列就是 $m$。

\vspace{5pt}

一个常见的错误就是用一个全 $0$ 的填充序列将给定的消息填充成分组长度的倍数。这样的映射并不是单射,并且会产生一个不安全的 MAC:对于任何长度不是分组长度的整数倍的消息 $m$,其 MAC 也是 $m\,\Vert\,0$ 的一个有效 MAC。因此,这样的 MAC 容易受到存在性伪造攻击。
\end{snote}

\begin{snote}[单射函数必会扩充。]
当我们把一个 $n$ 比特的单分组消息输入 $inj$ 时,该函数必须要增加一个``假"分组,并输出一个两分组消息。这对于那些主要以单分组消息为对象的应用来说是很不利的。当使用 CBC 或级联构造时,假分组迫使签名者和验证者对于每条消息都要评估两次底层的 PRF,就算所有的消息本身都只有一个分组。结果就是,所有参与方的计算量都被翻倍了。

因此,我们很自然地想要找到一个不需要添加假分组的单射函数。但不幸的是,这样的函数是不存在的,因为集合 $\{0,1\}^{\leq n\ell}$ 比集合 $\mathcal{X}^{\leq\ell}$ 要大。因此,所有的单射函数都必须要在输出中添加一个``假"分组。

我们将在 \ref{sec:6-10} 节介绍的 CMAC 构造为这个问题提供了一个优雅的解决方案。CMAC 通过使用一个\emph{随机化的}单射函数来避免添加假分组。
\end{snote}
\section{案例研究:ANSI CBC-MAC}\label{sec:6-9}

当使用 PRF 构建 MAC 时,实现者经常会只输出 PRF 输出的 $w$ 个最高有效比特来缩短最终得到的标签。练习 4.4 表明,将一个安全的 PRF 的输出截短并不会影响其作为 PRF 的安全性。但是,这种截短会影响派生的 MAC。定理 \ref{theo:6-2} 表明,$w$ 越小,MAC 的安全性就越低。特别是,该定理在具体的安全边界中包含一个 $1/2^w$ 的误差项。

两个 ANSI 标准(ANSI X9.9 和 ANSI X9.19)和两个 ISO 标准(ISO 8731-1 和 ISO/IEC 9797)指定了 ECBC 的几个变体,它们使用 DES 作为底层的 PRF,提供消息认证的功能。这些标准都截短了 ECBC-DES 的最终 $64$ 比特输出,只使用输出的最左边 $w$ 个比特,其中 $w=32$,$48$ 或 $64$。这种设计以降低安全性为代价,减少了标签的长度。

ANSI 的两个 CBC-MAC 标准都指定了一个填充方案,用于填充长度不是 DES 或者 AES 分组长度整数倍的消息。该填充方案与 \ref{sec:6-8} 节中描述的函数 $inj$ 相同。在签署消息和验证消息-标签对时,使用的填充方案都应当是相同的。
\section{CMAC}\label{sec:6-10}

基于密码的 MAC——CMAC——是美国国家标准研究所 (NIST) 在 2005 年采用的一个 ECBC 的变体。它基于 John Black 和 Phillip Rogaway 的一个提案,以及岩田哲 (Tetsu Iwata) 和黑泽馨 (Kaoru Kurosawa) 对它的一个扩展。CMAC 在两个方面改进了 ANSI 标准中所使用的 ECBC。首先,CMAC 使用随机化的无前缀编码将无前缀安全的 PRF 转换成安全的 PRF。这种设计省略了 ECBC 中所使用的最终加密步骤。其次,CMAC 使用了一种``双密钥"方法,以避免在输入消息长度是底层 PRF 分组长度的整数倍时需要附加一个假分组的问题。

CMAC 是使用 CBC 无前缀安全 PRF 建立按比特安全的 PRF 的最佳方法。它应该被用于代替所有的 ANSI 方法。在练习 \ref{exer:6-14} 中,我们表明,CMAC 构造同样适用于级联的场合。

\begin{snote}[CMAC 按比特 PRF。]
CMAC 算法由两个步骤组成。首先,一个子密钥生成算法被用来从 MAC 密钥 $k$ 派生出三个子密钥 $k_0$,$k_1$ 和 $k_2$。然后,这三个子密钥 $k_0$,$k_1$ 和 $k_2$ 被用来计算 MAC。

令 $F$ 是一个定义在 $(\mathcal{K},\mathcal{X},\mathcal{X})$ 上的PRF,其中 $\mathcal{X}=\{0,1\}^n$。NIST 标准使用 AES 作为 PRF $F$。我们在表 \ref{tab:6-1} 详细介绍了 CMAC 签名算法,并用图 \ref{fig:6-8} 给出了直观的展示。当消息长度是分组长度 $n$ 的整数倍时,我们就使用左图所展示的方法,否则就使用右图。该标准允许将最终输出截短到 $w$ 比特,只需要输出最终值 $t$ 的 $w$ 个最高有效比特。
\end{snote}

\begin{table}
	\hspace*{5pt} 输入:密钥 $k\in\mathcal{K}$ 和 $m\in\{0,1\}^*$\\
	\hspace*{26pt} 输出:标签 $t\in\{0,1\}^w$,其中 $w\leq n$

	\vspace{3pt}

	\hspace*{5pt} \underline{设置}:\\
	\hspace*{26pt} 运行一个子密钥生成算法,由 $k\in\mathcal{K}$ 生成子密钥 $k_0,k_1,k_2\in\mathcal{X}$\\
	\hspace*{26pt} 令 $\ell\leftarrow\mathrm{length}(m)$\\
	\hspace*{26pt} 令 $u\leftarrow\mathrm{max}(1,\lceil\ell/n\rceil)$\\
	\hspace*{26pt} 将 $m$ 切分成长为 $n$ 比特的连续分组,使得 $m=a_1\,\Vert\,a_2\,\Vert\,\cdot\,\Vert\,a_{u-1}\,\Vert\,a_u^*$,其中 $a_1,\dots,a_{u-1}\in\{0,1\}^n$\\
	\hspace*{26pt} 如果 $\mathrm{length}(a_u^*)=n$:\\
	\hspace*{50pt} 则令 $a_u=k_1\oplus a_u^*$\\
	\hspace*{50pt} 否则令 $a_u=k_2\oplus(a_u^*\,\Vert\,1\,\Vert\,0^j)$,其中 $j=nu-\ell-1$
	
	\vspace{3pt}
	
	\hspace*{5pt} \underline{CBC}:\\
	\hspace*{26pt} 令 $t\leftarrow0^n$\\
	\hspace*{26pt} 对于 $i=1,\dots,u$:\\
	\hspace*{50pt} 令 $t\leftarrow F(k_0,\;t\oplus a_i)$\\
	\hspace*{26pt} 输出 $t[0\dots w-1]$ \quad\quad\quad\quad\quad // \emph{输出 $t$ 的 $w$ 个最高有效比特}
  \caption{CMAC 签名算法}
  \label{tab:6-1}
\end{table}

\begin{figure}
  \centering
  \subfigure[当$\mathrm{length}(m)$ 是 $n$的正整数倍时]{
    

\tikzset{every picture/.style={line width=0.75pt}} %set default line width to 0.75pt        

\begin{tikzpicture}[x=0.75pt,y=0.75pt,yscale=-0.9,xscale=0.9]
%uncomment if require: \path (0,225); %set diagram left start at 0, and has height of 225

%Shape: Rectangle [id:dp47323021256135744] 
\draw  [fill={rgb, 255:red, 255; green, 255; blue, 255 }  ,fill opacity=1 ][line width=1.2] [general shadow={fill=black,shadow xshift=2.25pt,shadow yshift=-2.25pt}] (10,90) -- (60,90) -- (60,140) -- (10,140) -- cycle ;
%Shape: Rectangle [id:dp25442292054738114] 
\draw  [fill={rgb, 255:red, 255; green, 255; blue, 255 }  ,fill opacity=1 ][line width=1.2] [general shadow={fill=black,shadow xshift=2.25pt,shadow yshift=-2.25pt}] (0,0) -- (70,0) -- (70,20) -- (0,20) -- cycle ;
%Shape: Rectangle [id:dp9693094194534853] 
\draw  [fill={rgb, 255:red, 255; green, 255; blue, 255 }  ,fill opacity=1 ][line width=1.2] [general shadow={fill=black,shadow xshift=2.25pt,shadow yshift=-2.25pt}] (100,0) -- (170,0) -- (170,20) -- (100,20) -- cycle ;
%Shape: Rectangle [id:dp4113104484344767] 
\draw  [fill={rgb, 255:red, 255; green, 255; blue, 255 }  ,fill opacity=1 ][line width=1.2] [general shadow={fill=black,shadow xshift=2.25pt,shadow yshift=-2.25pt}] (230,0) -- (300,0) -- (300,20) -- (230,20) -- cycle ;
%Shape: Rectangle [id:dp8078258080536631] 
\draw  [fill={rgb, 255:red, 255; green, 255; blue, 255 }  ,fill opacity=1 ][line width=1.2] [general shadow={fill=black,shadow xshift=2.25pt,shadow yshift=-2.25pt}] (110,90) -- (160,90) -- (160,140) -- (110,140) -- cycle ;
%Shape: Rectangle [id:dp23625358032744836] 
\draw  [fill={rgb, 255:red, 255; green, 255; blue, 255 }  ,fill opacity=1 ][line width=1.2] [general shadow={fill=black,shadow xshift=2.25pt,shadow yshift=-2.25pt}] (240,90) -- (290,90) -- (290,140) -- (240,140) -- cycle ;
%Straight Lines [id:da9641444182909755] 
\draw    (35,20) -- (35,87) ;
\draw [shift={(35,90)}, rotate = 270] [fill={rgb, 255:red, 0; green, 0; blue, 0 }  ][line width=0.08]  [draw opacity=0] (7.14,-3.43) -- (0,0) -- (7.14,3.43) -- cycle    ;
%Straight Lines [id:da9243534980546237] 
\draw    (60,115) -- (85,115) -- (85,55) -- (127,55) ;
\draw [shift={(130,55)}, rotate = 180] [fill={rgb, 255:red, 0; green, 0; blue, 0 }  ][line width=0.08]  [draw opacity=0] (7.14,-3.43) -- (0,0) -- (7.14,3.43) -- cycle    ;
%Straight Lines [id:da5802055070589311] 
\draw    (230,55) -- (257,55) ;
\draw [shift={(260,55)}, rotate = 180] [fill={rgb, 255:red, 0; green, 0; blue, 0 }  ][line width=0.08]  [draw opacity=0] (7.14,-3.43) -- (0,0) -- (7.14,3.43) -- cycle    ;
%Straight Lines [id:da1747684846569182] 
\draw    (135,20) -- (135,47) ;
\draw [shift={(135,50)}, rotate = 270] [fill={rgb, 255:red, 0; green, 0; blue, 0 }  ][line width=0.08]  [draw opacity=0] (7.14,-3.43) -- (0,0) -- (7.14,3.43) -- cycle    ;
%Straight Lines [id:da18552822959352055] 
\draw    (135,60) -- (135,87) ;
\draw [shift={(135,90)}, rotate = 270] [fill={rgb, 255:red, 0; green, 0; blue, 0 }  ][line width=0.08]  [draw opacity=0] (7.14,-3.43) -- (0,0) -- (7.14,3.43) -- cycle    ;
%Straight Lines [id:da2622926087607196] 
\draw    (265,20) -- (265,47) ;
\draw [shift={(265,50)}, rotate = 270] [fill={rgb, 255:red, 0; green, 0; blue, 0 }  ][line width=0.08]  [draw opacity=0] (7.14,-3.43) -- (0,0) -- (7.14,3.43) -- cycle    ;
%Straight Lines [id:da5760788212181904] 
\draw    (265,60) -- (265,87) ;
\draw [shift={(265,90)}, rotate = 270] [fill={rgb, 255:red, 0; green, 0; blue, 0 }  ][line width=0.08]  [draw opacity=0] (7.14,-3.43) -- (0,0) -- (7.14,3.43) -- cycle    ;
%Straight Lines [id:da06362440991905571] 
\draw    (160,115) -- (185,115) ;
%Straight Lines [id:da5345180090048072] 
\draw  [dash pattern={on 0.84pt off 2.51pt}]  (185,115) -- (210,115) -- (210,55) -- (230,55) ;
%Straight Lines [id:da6845376750607981] 
\draw    (265,140) -- (265,187) ;
\draw [shift={(265,190)}, rotate = 270] [fill={rgb, 255:red, 0; green, 0; blue, 0 }  ][line width=0.08]  [draw opacity=0] (7.14,-3.43) -- (0,0) -- (7.14,3.43) -- cycle    ;
%Straight Lines [id:da2631191484108897] 
\draw    (273,55) -- (300,55) ;
\draw [shift={(270,55)}, rotate = 0] [fill={rgb, 255:red, 0; green, 0; blue, 0 }  ][line width=0.08]  [draw opacity=0] (7.14,-3.43) -- (0,0) -- (7.14,3.43) -- cycle    ;

% Text Node
\draw (35,115) node  [font=\small]  {$F( k,\cdot )$};
% Text Node
\draw (135,115) node  [font=\small]  {$F( k,\cdot )$};
% Text Node
\draw (265,115) node  [font=\small]  {$F( k,\cdot )$};
% Text Node
\draw (135,55) node  [font=\large]  {$\oplus $};
% Text Node
\draw (265,55) node  [font=\large]  {$\oplus $};
% Text Node
\draw (200,10) node  [font=\small]  {$\cdots $};
% Text Node
\draw (35,10) node  [font=\small]  {$a_{1}$};
% Text Node
\draw (135,10) node  [font=\small]  {$a_{2}$};
% Text Node
\draw (265,10) node  [font=\small]  {$a_{u}$};
% Text Node
\draw (265,193.4) node [anchor=north] [inner sep=0.75pt]  [font=\small]  {$tag$};
% Text Node
\draw (300,51.6) node [anchor=south] [inner sep=0.75pt]  [font=\small]  {$\mathbf{k}_\mathbf{1}$};


\end{tikzpicture}
  }
  \quad
  \subfigure[其他情况]{
    

\tikzset{every picture/.style={line width=0.75pt}} %set default line width to 0.75pt        

\begin{tikzpicture}[x=0.75pt,y=0.75pt,yscale=-0.9,xscale=0.9]
%uncomment if require: \path (0,225); %set diagram left start at 0, and has height of 225

%Shape: Rectangle [id:dp3369166275751636] 
\draw  [fill={rgb, 255:red, 255; green, 255; blue, 255 }  ,fill opacity=1 ][line width=1.2] [general shadow={fill=black,shadow xshift=2.25pt,shadow yshift=-2.25pt}] (10,90) -- (60,90) -- (60,140) -- (10,140) -- cycle ;
%Shape: Rectangle [id:dp8644319992918916] 
\draw  [fill={rgb, 255:red, 255; green, 255; blue, 255 }  ,fill opacity=1 ][line width=1.2] [general shadow={fill=black,shadow xshift=2.25pt,shadow yshift=-2.25pt}] (0,0) -- (70,0) -- (70,20) -- (0,20) -- cycle ;
%Shape: Rectangle [id:dp9581280342770533] 
\draw  [fill={rgb, 255:red, 255; green, 255; blue, 255 }  ,fill opacity=1 ][line width=1.2] [general shadow={fill=black,shadow xshift=2.25pt,shadow yshift=-2.25pt}] (100,0) -- (170,0) -- (170,20) -- (100,20) -- cycle ;
%Shape: Rectangle [id:dp5189349264045813] 
\draw  [fill={rgb, 255:red, 255; green, 255; blue, 255 }  ,fill opacity=1 ][line width=1.2] [general shadow={fill=black,shadow xshift=2.25pt,shadow yshift=-2.25pt}] (230,0) -- (300,0) -- (300,20) -- (230,20) -- cycle ;
%Shape: Rectangle [id:dp9935749246612766] 
\draw  [fill={rgb, 255:red, 255; green, 255; blue, 255 }  ,fill opacity=1 ][line width=1.2] [general shadow={fill=black,shadow xshift=2.25pt,shadow yshift=-2.25pt}] (110,90) -- (160,90) -- (160,140) -- (110,140) -- cycle ;
%Shape: Rectangle [id:dp9080240218726778] 
\draw  [fill={rgb, 255:red, 255; green, 255; blue, 255 }  ,fill opacity=1 ][line width=1.2] [general shadow={fill=black,shadow xshift=2.25pt,shadow yshift=-2.25pt}] (240,90) -- (290,90) -- (290,140) -- (240,140) -- cycle ;
%Straight Lines [id:da4310006464232563] 
\draw    (35,20) -- (35,87) ;
\draw [shift={(35,90)}, rotate = 270] [fill={rgb, 255:red, 0; green, 0; blue, 0 }  ][line width=0.08]  [draw opacity=0] (7.14,-3.43) -- (0,0) -- (7.14,3.43) -- cycle    ;
%Straight Lines [id:da040243810245369716] 
\draw    (60,115) -- (85,115) -- (85,55) -- (127,55) ;
\draw [shift={(130,55)}, rotate = 180] [fill={rgb, 255:red, 0; green, 0; blue, 0 }  ][line width=0.08]  [draw opacity=0] (7.14,-3.43) -- (0,0) -- (7.14,3.43) -- cycle    ;
%Straight Lines [id:da8310478556174594] 
\draw    (230,55) -- (257,55) ;
\draw [shift={(260,55)}, rotate = 180] [fill={rgb, 255:red, 0; green, 0; blue, 0 }  ][line width=0.08]  [draw opacity=0] (7.14,-3.43) -- (0,0) -- (7.14,3.43) -- cycle    ;
%Straight Lines [id:da5512587222870278] 
\draw    (135,20) -- (135,47) ;
\draw [shift={(135,50)}, rotate = 270] [fill={rgb, 255:red, 0; green, 0; blue, 0 }  ][line width=0.08]  [draw opacity=0] (7.14,-3.43) -- (0,0) -- (7.14,3.43) -- cycle    ;
%Straight Lines [id:da34079874242613895] 
\draw    (135,60) -- (135,87) ;
\draw [shift={(135,90)}, rotate = 270] [fill={rgb, 255:red, 0; green, 0; blue, 0 }  ][line width=0.08]  [draw opacity=0] (7.14,-3.43) -- (0,0) -- (7.14,3.43) -- cycle    ;
%Straight Lines [id:da7786742586004287] 
\draw    (265,20) -- (265,47) ;
\draw [shift={(265,50)}, rotate = 270] [fill={rgb, 255:red, 0; green, 0; blue, 0 }  ][line width=0.08]  [draw opacity=0] (7.14,-3.43) -- (0,0) -- (7.14,3.43) -- cycle    ;
%Straight Lines [id:da6649625572955551] 
\draw    (265,60) -- (265,87) ;
\draw [shift={(265,90)}, rotate = 270] [fill={rgb, 255:red, 0; green, 0; blue, 0 }  ][line width=0.08]  [draw opacity=0] (7.14,-3.43) -- (0,0) -- (7.14,3.43) -- cycle    ;
%Straight Lines [id:da4282493595023369] 
\draw    (160,115) -- (185,115) ;
%Straight Lines [id:da17082070379132452] 
\draw  [dash pattern={on 0.84pt off 2.51pt}]  (185,115) -- (210,115) -- (210,55) -- (230,55) ;
%Straight Lines [id:da8367261365795853] 
\draw    (265,140) -- (265,187) ;
\draw [shift={(265,190)}, rotate = 270] [fill={rgb, 255:red, 0; green, 0; blue, 0 }  ][line width=0.08]  [draw opacity=0] (7.14,-3.43) -- (0,0) -- (7.14,3.43) -- cycle    ;
%Straight Lines [id:da9105502462787465] 
\draw    (273,55) -- (300,55) ;
\draw [shift={(270,55)}, rotate = 0] [fill={rgb, 255:red, 0; green, 0; blue, 0 }  ][line width=0.08]  [draw opacity=0] (7.14,-3.43) -- (0,0) -- (7.14,3.43) -- cycle    ;

% Text Node
\draw (35,115) node  [font=\small]  {$F( k,\cdot )$};
% Text Node
\draw (135,115) node  [font=\small]  {$F( k,\cdot )$};
% Text Node
\draw (265,115) node  [font=\small]  {$F( k,\cdot )$};
% Text Node
\draw (135,55) node  [font=\large]  {$\oplus $};
% Text Node
\draw (265,55) node  [font=\large]  {$\oplus $};
% Text Node
\draw (200,10) node  [font=\small]  {$\cdots $};
% Text Node
\draw (35,10) node  [font=\small]  {$a_{1}$};
% Text Node
\draw (135,10) node  [font=\small]  {$a_{2}$};
% Text Node
\draw (265,10) node  [font=\small]  {$a_{u} \ \| \ 100$};
% Text Node
\draw (265,193.4) node [anchor=north] [inner sep=0.75pt]  [font=\small]  {$tag$};
% Text Node
\draw (300,51.6) node [anchor=south] [inner sep=0.75pt]  [font=\small]  {$\mathbf{k}_\mathbf{2}$};


\end{tikzpicture}
  }
  \caption{CMAC签名算法}
  \label{fig:6-8}
\end{figure}

\begin{snote}[安全性。]
图 \ref{fig:6-8} 中所描述的 CMAC 算法可以用随机化无前缀编码的范式来分析。实际上,CMAC 使用一个随机化无前缀编码 $rpf:\mathcal{K}\times\mathcal{M}\to\mathcal{X}^{\leq\ell}_{_{>0}}$ 将 CBC 无前缀安全的 PRF 直接转换为\emph{按比特}安全的 PRF,其中 $\mathcal{K}:=\mathcal{X}^2$,$\mathcal{M}:=\{0,1\}^{\leq n\ell}$。编码 $rpf$ 的定义如下:

\vspace{5pt}

\hspace*{5pt} 输入:$m\in\mathcal{M}$ 和 $(k_1,k_2)\in\mathcal{K}^2$

\vspace{3pt}

\hspace*{5pt} 如果 $|m|$ 不是 $n$ 的正整数倍:\\
\hspace*{50pt} 令 $u\leftarrow |m|\;\mathrm{mod}\;n$\\
\hspace*{26pt} 将 $m$ 切分为一连串的比特序列 $a_1,\dots,a_v\in\mathcal{X}$,\\
\hspace*{50pt} 因此,$m=a_1\,\Vert\,\dots\,\Vert\,a_v$ 和 $a_1,\dots,a_{v-1}$ 都是 $n$ 比特的序列\\
\hspace*{26pt} 如果 $|m|$ 是 $n$ 的正整数倍:\\
\hspace*{50pt} 则输出 $\big(a_1,\,\dots,\,a_{v-1},\,(a_v\oplus k_1)\big)$\\
\hspace*{50pt} 否则输出 $\big(a_1,\,\dots,\,a_{v-1},\,((a_v\,\Vert\,1\,\Vert\,0^{n-v-1})\oplus k_2)\big)$

\vspace{5pt}

\noindent
对 $rpf$ 是一个随机性 $2^{-n}$-无前缀编码的论证与 \ref{sec:6-7} 节中的论证类似。因此,CMAC 符合随机性无前缀编码的范式,其安全性可由定理 \ref{theo:6-9} 推出。密钥 $k_1$ 和 $k_2$ 是用于解决长度为 $n$ 的正整数倍的消息和被填充为 $n$ 的正整数倍的消息之间的碰撞的。这对 CMAC $rpf$ 的分析至关重要。
\end{snote}

\begin{snote}[子密钥生成。]
子密钥生成算法使用密钥 $k$ 生成子密钥 $(k_0,k_1,k_2)$。它使用一个固定的掩码序列 $R_n$,取决于 $F$ 的分组长度。比如说,对于 $128$ 比特的分组长度,标准规定 $R_{128}:=0^{120}10000111$。对于一个比特序列 $X$,我们用 $X\ll1$ 表示将 $X$ 最左侧的比特丢弃,并在最右侧添加一个 $0$ 而产生的新比特序列。于是,子密钥生成算法的工作方式如下:

\vspace{5pt}

\hspace*{5pt} 输入:密钥 $k\in\mathcal{K}$\\
\hspace*{26pt} 输出:子密钥 $k_0,k_1,k_2\in\mathcal{K}$

\vspace{3pt}

\hspace*{5pt} 令 $k_0\leftarrow k$\\
\hspace*{26pt} 令 $L\leftarrow F(k,0^n)$\\
\hspace*{1pt} ($1$)
\hspace*{4.5pt} 如果 $\mathrm{msb}(L)=0$,则令 $k_1\leftarrow (L\ll1)$,否则令 $k_1\leftarrow(L\ll1)\oplus R_n$\\
\hspace*{1pt} ($2$)
\hspace*{4.5pt} 如果 $\mathrm{msb}(k_1)=0$,则令 $k_2\leftarrow (k_1\ll1)$,否则令 $k_2\leftarrow(k_1\ll1)\oplus R_n$\\
\hspace*{26pt} 输出 $k_0,k_1,k_2$

\vspace{5pt}

\noindent
其中,$\mathrm{msb}(L)$ 指的是 $L$ 的最高有效比特。标有 ($1$) 和($2$) 的两行看起来有点神秘,但它们实际上只是在有限域 $\mathrm{GF}(2^n)$ 中(分别)用 $L$ 乘以 $X$ 和 $X^2$。在这里,我们将 $\mathrm{GF}(2^n)$ 中的元素视为用 $\mathbb{F}_2[X]$ 中的多项式模一个固定多项式 $g(X)$ 后的结果。对于 $128$ 比特的分组长度,对应于 $R_{128}$ 的定义多项式$g(X)$ 是 $g(X):=X^{128}+X^7+X^2+X+1$。练习 \ref{exer:6-16} 探讨了子密钥生成的一些不安全的变体。

子密钥生成算法输出的三个密钥 $(k_0,k_1,k_2)$ 可以被用来认证多条消息。因此,它的运行时间可以分摊到多条消息之中。

密钥 $k_0$,$k_1$ 和 $k_2 $ 显然不是独立的。如果它们是独立的,或者如果它们被派生成相互独立的,比如对于常量 $\alpha_0$,$\alpha_1$ 和 $\alpha_2$,令 $k_i:=F(k,\alpha_i)$,CMAC 的安全性就可以直接由这里的论证和我们的通用框架得到。然而,借由更复杂的分析,我们也能证明 CMAC 事实上确实是安全的。
\end{snote}
\section{PMAC:一种并行MAC}
\section{一个有趣的应用:在加密数据上进行搜索}

待写。
\section{案例研究:TLS会话设置}\label{sec:21-10}
\section{一个有趣的应用:建立 Tor 信道}\label{sec:21-13}
\section{练习}

\begin{exercise}\label{exer:6-1}
\end{exercise}

\begin{exercise}\label{exer:6-2}
\end{exercise}

\begin{exercise}\label{exer:6-3}
\end{exercise}

\begin{exercise}\label{exer:6-4}
\end{exercise}

\begin{exercise}\label{exer:6-5}
\end{exercise}

\begin{exercise}\label{exer:6-6}
\end{exercise}

\begin{exercise}\label{exer:6-7}
\end{exercise}

\begin{exercise}\label{exer:6-8}
\end{exercise}

\begin{exercise}\label{exer:6-9}
\end{exercise}

\begin{exercise}\label{exer:6-10}
\end{exercise}

\begin{exercise}\label{exer:6-11}
\end{exercise}

\begin{exercise}\label{exer:6-12}
\end{exercise}

\begin{exercise}\label{exer:6-13}
\end{exercise}

\begin{exercise}\label{exer:6-14}
\end{exercise}

\begin{exercise}\label{exer:6-15}
\end{exercise}

\begin{exercise}\label{exer:6-16}
\end{exercise}

\begin{exercise}\label{exer:6-17}
\end{exercise}

\begin{exercise}\label{exer:6-18}
\end{exercise}

\begin{exercise}\label{exer:6-19}
\end{exercise}
