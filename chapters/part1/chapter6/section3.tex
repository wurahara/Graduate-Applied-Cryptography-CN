\section{使用 PRF 构建 MAC}\label{sec:6-3}

我们下面尝试使用我们已经掌握的工具来构建安全的 MAC。在前几章中,我们使用伪随机函数 (PRF) 来构建各种加密系统。我们举了一些实际的 PRF 的例子,比如 AES(虽然 AES 是一个分组密码,但由于 PRF 切换引理,即定理 \ref{theo:4-4},我们也可将其看作是一个 PRF)。在这里,我们表明,任何安全的 PRF 都可以直接被用来建立一个安全的 MAC。

回顾一下,PRF 是这样的一种算法 $F$,它接受密钥 $k$ 和数据分组 $x$ 作为输入,并输出一个值 $y:=F(k,x)$。通常,如果密钥在 $\mathcal{K}$ 上,输入在 $\mathcal{X}$ 上,输出在 $\mathcal{Y}$ 上,我们就称 $F$ 定义在 $(\mathcal{K},\mathcal{X},\mathcal{Y})$ 上。对于一个 PRF $F$,我们将\textbf{由 $F$ 派生出的确定性 MAC 系统} $\mathcal{I}=(S,V)$ 定义为:

\vspace{5pt}

\hspace*{5pt} $S(k,m)=F(k,m)$;

\vspace{5pt}

\hspace*{5pt} $
V(k,m,t)=\left\{
\begin{array}{ll}
\mathsf{accept}, & F(k,m)=t\\
\mathsf{reject}, & \text{otherwise}
\end{array}
\right.
$

\vspace{8pt}

正如已经讨论过的,任何具有大的(即超多项式的)输出空间的 PRF 都是不可预测的(见 \ref{subsec:4-1-1} 小节),因此,正如 \ref{sec:6-1} 节所讨论的,上述构造产生了一个安全的 MAC。完整起见,我们下面用一个定理来说明这一点。

\begin{theorem}\label{theo:6-2}
令 $F$ 是一个定义在 $(\mathcal{K},\mathcal{X},\mathcal{Y})$ 上的安全的 PRF,其中 $|\mathcal{Y}|$ 是超多项式的。那么,由 $F$ 派生的确定性 MAC 系统 $\mathcal{I}$ 是一个安全的 MAC。
\begin{quote}
特别地,对于每个按照攻击游戏 \ref{game:6-1} 攻击 $\mathcal{I}$ 的 $Q$ 次查询 MAC 对手 $\mathcal{A}$,都存在一个按照攻击游戏 \ref{game:4-2} 攻击 $F$ 的 $(Q+1)$ 次查询 PRF 对手 $\mathcal{B}$,其中 $\mathcal{B}$ 是一个围绕 $\mathcal{A}$ 的基本包装器,满足:
\end{quote}
\[
{\rm MAC}\mathsf{adv}[\mathcal{A},\mathcal{I}]\leq{\rm PRF\mathsf{adv}}[\mathcal{B}, F]+{1}/{|\mathcal{Y}|}
\]
\end{theorem}

\begin{proof}[证明思路]
令 $\mathcal{A}$ 是一个有效 MAC 对手。我们通过约束 $\mathcal{A}$ 产生伪造的信息-标签对的能力,推导出 ${\rm MAC}\mathsf{adv}[\mathcal{A},\mathcal{I}]$ 的上界。和之前一样,用 ${\rm Funs}[\mathcal{X},\mathcal{Y}]$ 中的一个真随机函数 $f$ 替换底层的安全 PRF $F$ 并不会使 $\mathcal{A}$ 的优势发生很大的变化。但是,如果对手 $\mathcal{A}$ 交互的对象是一个真随机函数,它将会面临一个没有希望的任务:它需要在它选定的几个点上获取 $f$ 的值,并发动选择消息攻击。然后,它需要猜测 $f(m)\in\mathcal{Y}$ 在某个新的点 $m$ 处的值。但由于 $f$ 是一个真随机函数,$\mathcal{A}$ 没有办法获得任何关于 $f(m)$ 的信息,因此它猜中 $f(m)$ 的概率微乎其微。
\end{proof}

\begin{proof}
我们通过让 $\mathcal{A}$ 与两个密切相关的挑战者交互来证明这个结论。

\vspace{5pt}

\noindent\textbf{游戏$\mathbf{0}$}。
和之前一样,我们先回顾 MAC 攻击游戏 \ref{game:6-1} 中的挑战者,它此时被应用于 $\mathcal{I}$。在这个游戏中,挑战者的逻辑如下:

\vspace{5pt}

\hspace*{-16.5pt} ($*$)
\hspace*{1pt} 选取 $k\overset{\rm R}\leftarrow\mathcal{K}$,令 $f\leftarrow F(k,\cdot)$\\
\hspace*{26pt} 当收到第 $i$ 个签名查询 $m_i\in\mathcal{M}$ ($i=1,2,\dots$) 时:\\
\hspace*{50pt} 令 $t_i\leftarrow f(m_i)$\\
\hspace*{50pt} 将 $t_i$ 发送给对手

\vspace{5pt}

\noindent
在游戏结束时,对手输出一个消息-标签对 $(m,t)$。我们将 $W_0$ 定义为条件:
\begin{equation}\label{eq:6-5}
t=f(m)
\quad\quad\text{and}\quad\quad
m\notin\{m_1,m_2,\dots\}
\end{equation}
在游戏 $0$ 中成立的事件。显然,$\Pr[W_0]={\rm MAC}\mathsf{adv}[\mathcal{A},\mathcal{I}]$ 成立。

\vspace{5pt}

\noindent\textbf{游戏$\mathbf{1}$}。
接下来,和之前一样,我们打出``PRF牌",用 ${\rm Funs}[\mathcal{X},\mathcal{Y}]$ 中的一个真随机函数 $f$ 替换 $F(k,\cdot)$。直观地说,由于 $F$ 是一个安全的 PRF,对手 $\mathcal{A}$ 应该不会注意到这种差别。游戏 $1$ 中的挑战者与游戏 $0$ 中的挑战者大致相同,只是我们将标有 ($*$) 的那一行改为:

\vspace{5pt}

\hspace*{-16.5pt} ($*$)
\hspace*{1pt} 选取 $f\overset{\rm R}\leftarrow{\rm Funs}[\mathcal{X},\mathcal{Y}]$

\vspace{5pt}

\noindent
令 $W_1$ 为式 \ref{eq:6-5} 在游戏 $1$ 中成立的事件。我们构建一个 $(Q+1)$ 次查询 PRF 对手 $\mathcal{B}$,满足:
\begin{equation}\label{eq:6-6}
|\Pr[W_1]-\Pr[W_0]|={\rm PRF}\mathsf{adv}[\mathcal{B},F]
\end{equation}
对手 $\mathcal{B}$ 通过查询自己的 PRF 挑战者来应答 $\mathcal{A}$ 的选择消息查询。最终,$\mathcal{A}$ 会输出一个候选的 MAC 伪造 $(m,t)$,其中 $m$ 不在 $\mathcal{A}$ 之前的选择信息查询中。现在,$\mathcal{B}$ 在 $m$ 处向它的 PRF 挑战者发起查询,并得到某个 $t'\in\mathcal{Y}$。如果 $t=t'$,$\mathcal{B}$ 就输出 $0$,否则就输出 $1$。不难证明,这样的对手 $\mathcal{B}$ 满足式 \ref{eq:6-6}。

接下来我们直接约束 $\Pr[W_1]$。对手 $\mathcal{A}$ 得到了 $f$ 在 $m_1,m_2,\dots$ 处的值,然后被要求猜测 $f$ 在一个新点 $m$ 处的值。但由于 $f$ 是一个真随机函数,所以值 $f(m)$ 与它在其他点处的值都无关。由于 $m\notin\{m_1,m_2,\dots\}$,所以对手 $\mathcal{A}$ 将以 ${1}/{|\mathcal{Y}|}$ 的概率猜中 $f(m)$。因此,$\Pr[W_1]\leq{1}/{|\mathcal{Y}|}$。将此与式 \ref{eq:6-6} 结合,我们可以得到:
\[
{\rm MAC}\mathsf{adv}[\mathcal{A},\mathcal{I}]
=\Pr[W_0]
\leq\big\lvert\Pr[W_0]−\Pr[W_1]\big\rvert+\Pr[W_1]
\leq{\rm PRF}\mathsf{adv}[\mathcal{B},F]+\frac{1}{|\mathcal{Y}|}
\]
因此本定理得证。
\end{proof}

\begin{snote}[具体的标签长度。]
该定理表明,为确保 ${\rm MAC}\mathsf{adv}[\mathcal{A},\mathcal{I}]<2^{128}$,我们需要这样的一个 PRF,其输出空间 $\mathcal{Y}$ 满足 $|\mathcal{Y}|>2^{128}$。如果输出空间 $\mathcal{Y}$ 形如 $\{0,1\}^n$,其中 $n$ 是某个整数,那么产生的标签至少长 $128$ 比特。
\end{snote}