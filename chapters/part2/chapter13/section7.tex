\section{签名加密:将签名和加密结合起来}\label{sec:13-7}

\begin{definition}\label{def:13-5}
	
\end{definition}

\subsection{安全的签名加密}\label{subsec:13-7-1}

\begin{game}[密文完整性]\label{game:13-5}
	
\end{game}

\begin{definition}\label{def:13-6}
	
\end{definition}

\begin{game}[CCA 安全性]\label{game:13-6}
	
\end{game}

\begin{definition}[CCA 安全性]\label{def:13-7}
	
\end{definition}

\begin{definition}\label{def:13-8}
	
\end{definition}

\subsection{作为抽象接口的签名加密}\label{subsec:13-7-2}

\subsection{两种构造:先加密后签名和先签名后加密}\label{subsec:13-7-3}

\begin{theorem}\label{theo:13-8}
	
\end{theorem}

\begin{theorem}\label{theo:13-9}
	
\end{theorem}

\subsection{一种基于 Diffie-Hellman 密钥交换的构造}\label{subsec:13-7-4}

\begin{game}[双向交互计算性 Diffie-Hellman]\label{game:13-7}
	
\end{game}

\begin{definition}[双向交互计算性 Diffie-Hellman 假设]\label{def:13-9}
	
\end{definition}

\begin{theorem}\label{theo:13-10}
	
\end{theorem}

\subsection{额外的理想属性:前向保密性和不可抵赖性}\label{subsec:13-7-5}

\subsubsection{属性 I:前向保密性(发件人作恶情况下的安全性)}\label{subsubsec:13-7-5-1}

\begin{game}[具备发件人作恶前向保密性的 CCA 安全性]\label{game:13-8}
	
\end{game}

\begin{definition}\label{def:13-10}
	
\end{definition}

\begin{theorem}\label{theo:13-11}
	
\end{theorem}

\subsubsection{属性 II:不可抵赖性(收件人作恶情况下的安全性)}\label{subsubsec:13-7-5-2}

\begin{game}[具备不可抵赖性的密文完整性]\label{game:13-9}
	
\end{game}

\begin{definition}\label{def:13-11}
	
\end{definition}
