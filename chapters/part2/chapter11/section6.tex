\section{门限加密}\label{sec:11-6}

\begin{definition}\label{def:11-6}
	
\end{definition}

\subsection{Shamir 秘密分享方案}\label{subsec:11-6-1}

我们的 ElGamal 加密的门限版本基于一种被称为\textbf{秘密共享(secret sharing)}的技术,它还有许多其他的应用。

假设 Alice 有一个秘密 $\alpha\in Z$,其中 $Z$ 是某个有限集。她希望生成 $s$ 份 $\alpha$ 的共享,其中的每一份都属于某个有限集 $Z'$,记为 $\alpha_1,\dots,\alpha_s\in Z'$,并且满足以下属性:这 $s$ 份中的任意 $t$ 份都足以用来重建 $\alpha$,但任意一组由 $t-1$ 份构成的集合都不会透露任何关于 $\alpha$ 的信息。这种共享使得 Alice 可以向她的 $s$ 个朋友一人发送一部分秘密,这样,任意 $t$ 个朋友都可以帮助她恢复 $\alpha$,但是 $t-1$ 个朋友却什么都学不到。这样的方案被称为\textbf{秘密共享方案}。

\begin{definition}\label{def:11-7}
一个 \textbf{$Z$ 上的秘密共享方案}是一对有效算法 $(G,C)$:
\begin{itemize}
	\item $G$ 是一个概率性算法,用于生成 $\alpha$ 的一个 $s$ 选 $t$ 共享。其调用方式是 $(\alpha_1,\dots,\alpha_s)\overset{\rm R}\leftarrow G(s,t,\alpha)$,其中 $0<t\leq s$,$\alpha\in Z$。它输出 $s$ 份共享 $\mathit{SK}:=\{\alpha_1,\dots,\alpha_s\}$。
	\item $C$ 是一个确定性算法,用于恢复 $\alpha$。其调用方式是 $\alpha\leftarrow C(\alpha_1',\dots,\alpha_t')$。
	\item 正确性:我们要求,对于每个 $\alpha\in Z$,每个由算法 $G(s,t,\alpha)$ 输出的由 $s$ 份共享组成的集合 $SK$,以及每个 $\mathit{SK}$ 的大小为 $t$ 的子集 $\{\alpha_1',\dots,\alpha_t'\}$,我们都有 $C(\alpha_1',\dots,\alpha_t')=\alpha$。
\end{itemize}
\end{definition}

直观地讲,如果每个由算法 $G(s,t,\alpha)$ 输出的 $t-1$ 份共享组成的集合都没有透露任何关于 $\alpha$ 的信息,该秘密共享方案就是安全的。为了正式定义这个概念,使用以下符号是比较方便的:对于一个集合 $S\subseteq\{1,\dots,s\}$,我们记 $G(s,t,\alpha)[S]$ 为 $S$ 所索引的位置上输出的共享份额的集合。举例来说,$G(s,t,\alpha)[\{1,3,4\}]$ 就是集合 $\{\alpha_1,\alpha_3,\alpha_4\}$。

\begin{definition}\label{def:11-8}
如果对于每个 $\alpha,\alpha' \in Z$ 和集合 $\{1,\dots,s\}$ 的每个大小为 $t-1$ 的子集 $S$,分布 $G(s,t,\alpha)[S]$ 和分布 $G(s,t,\alpha')[S]$ 都相同,我们就称集合 $Z$ 上的秘密共享方案 $(G,C)$ 是\textbf{安全的}。
\end{definition}

这个定义意味着,对于 $Z$ 中的所有 $\alpha$ 和 $\alpha'$,仅仅通过观察 $t-1$ 份分享,我们无法判断秘密是 $\alpha$ 还是 $\alpha'$。因此,$t-1$ 份共享不会泄露关于秘密的任何信息。

\begin{snote}[Shamir 秘密共享。]
Shamir 提出了一种优雅的 $\mathbb{Z}_q$ 上的秘密共享方案,其中 $q$ 是素数。该方案利用了下面这个关于\textbf{多项式插值(polynomial interpolation)}的一般事实:一个不超过 $t-1$ 阶的多项式可以由该多项式上的 $t$ 个点完全确定。例如,两点可以确定一条直线,三点可以确定一条抛物线。这个一般事实不仅对实数集和复数集成立,而且对所有非零元素都有乘法逆元的代数领域都成立。我们将这种代数领域称为\emph{域 (field)}。当 $q$ 是素数时,$\mathbb{Z}_q$ 就是一个域,所以这个一般事实在这里也是成立的。

Shamir 的方案 $(G_{\rm sh},C_{\rm sh})$ 是一个 $\mathbb{Z}_q$ 上的 $s$ 选 $t$ 秘密共享方案,它要求 $q>s$,运行如下:
\begin{itemize}
	\item $G_{\rm sh}(s,t,\alpha)$:随机选取 $a_1,\dots,a_{t-1}\overset{\rm R}\leftarrow\mathbb{Z}_q$,并定义多项式:
	\[
	f(x)=a_{t-1}x^{t-1}+a_{t-2}x^{t-2}+\cdots+a_1x+\alpha\quad\in\mathbb{Z}_q[x]
	\]
	注意,多项式 $f$ 最大 $t-1$ 阶,且 $f(0)=\alpha$。
	
	下面,随机从 $\mathbb{Z}_q$ 中选择 $s$ 个非零点 $x_1,\dots,x_s$(比如说,我们可以直接使用 $\mathbb{Z}_q$ 中的点 $1,\dots,s$)。
	
	对于 $i=1,\dots,s$,计算 $y_i\leftarrow f(x_i)\in\mathbb{Z}_q$,并定义 $\alpha_i:=(x_i,y_i)$。
	
	输出 $s$ 份共享 $\alpha_1,\dots,\alpha_s\in\mathbb{Z}_q^2$。
	\item $C_{\rm sh}(\alpha_1',\dots,\alpha_t')$:一个输入包含 $t$ 个有效共享份额,它们对应于多项式 $f$ 上的 $t$ 个点,而这 $t$ 个点能够完全确定 $f$。算法 $C_{\rm sh}$ 对多项式 $f$ 进行插值,并输出 $\alpha:=f(0)$。
\end{itemize}

我们有必要详细解释算法 $C_{\rm sh}$ 的原理。有一种简单的方法可以从 $t$ 个点插值出最高 $t-1$ 阶的多项式,这种方法被称为\textbf{拉格朗日插值 (Lagrange interpolation)}。让我们看看,它是如何工作的。

给定 $t$ 份共享 $\alpha_i'=(x_i',y_i')$,其中 $i=1,\dots,t$,我们定义 $t$ 个多项式:
\[
L_i(x)=\prod^t_{j=1\atop j\neq i}\frac{x-x_j'}{x_i'-x_j'}\quad\in\mathbb{Z}_q[x],
\qquad
i=1,\dots,t
\]
不难验证,对于 $j\in\{1,\dots,t\}$ 中的所有 $j\neq i$,都有 $L_i(x_i')=1$ 且 $L_i(x_j')=0$ 成立。下面,考虑多项式:
\[
g(x):=L_1(x)\cdot y_1'+\dots+L_t(x)\cdot y_t'\quad\in\mathbb{Z}_q[x]
\]
同样,不难看出,$g(x_i')=y_i'=f(x_i')$ 对于所有 $i=1,\dots,t$ 都成立。由于 $f$ 和 $g$ 都是最高 $t-1$ 阶的多项式,而且它们在 $t$ 个点上均能匹配,所以它们一定是同一个多项式(这里,我们使用了关于多项式插值的一般事实)。因此,我们有 $\alpha=f(0)=g(0)$。特别地:
\begin{equation}\label{eq:11-20}
\alpha=g(0)=\sum^t_{i=1}\lambda_i\cdot y_i',
\qquad\text{其中}\qquad
\lambda_i:=L_i(0)=\prod^t_{j=1\atop j\neq i}\frac{-x_j'}{x_i'-x_j'}\in\mathbb{Z}_q
\end{equation}
其中,标量 $\lambda_1,\dots,\lambda_t\in\mathbb{Z}_q$ 被称为\textbf{拉格朗日系数 (Lagrange coefficients)}。

利用式 \ref{eq:11-20},我们现在可以更详细地描述算法 $C_{\rm sh}$。给定一个由 $t-1$ 份共享构成的集合,算法首先计算拉格朗日系数 $\lambda_1,\dots,\lambda_t\in\mathbb{Z}_q$。计算这些值需要使用除法,但由于 $q$ 是素数,这种计算一般是很简单的。然后,算法使用式 \ref{eq:11-20} 左手边的线性组合来计算 $\alpha$。

请注意,拉个朗日系数 $\lambda_1,\dots,\lambda_t$ 并不依赖于秘密 $\alpha$,而且,如果我们提前知道哪些共享份额会被用于重构 $\alpha$,我们就可以预先将其计算出来。
\end{snote}

\begin{snote}[安全性。]
剩下的工作就是证明这个秘密共享方案是安全的,如定义 \ref{def:11-8}。
\end{snote}

\begin{theorem}\label{theo:11-6}
Shamir 秘密共享方案 $(G_{\rm sh},C_{\rm sh})$ 是安全的。
\end{theorem}

\begin{proof}
为了证明该定理,我们将表明,对于每个 $\alpha\in\mathbb{Z}_q$,任何一个由 $t-1$ 份共享组成的集合 $(x_1',y_1'),\dots,\allowbreak (x_{t-1}',y_{t-1}')$ 都具备这样的特性:纵坐标 $y_1',\dots,y_{t-1}'$ 都均匀分布在 $\mathbb{Z}_q$ 上,且相互独立。因此,我们令 $\alpha$ 和 $x_1',\dots,x_{t-1}'$ 都是固定的。 

\emph{声称。}考虑从 $(a_1,\dots,a_{t-1})\in\mathbb{Z}_q^{t-1}$(由 $G_{\rm sh}(s,t,\alpha)$ 选出)到 $(y_1',\dots,y_{t-1}')\in\mathbb{Z}_q^{t-1}$ 的映射,后者是 $t-1$ 份共享的纵坐标,与之对应的横坐标是 $x_1',\dots,x_{t-1}'$。那么,该映射是一个单射。

于是,定理可以由该声称直接得出,因为如果 $(a_1,\dots,a_{t-1})$ 是从 $\mathbb{Z}_q^{t-1}$ 中均匀随机选出的,那么 $(y_1',\dots,y_{t-1}')$ 也一定均匀分布在 $\mathbb{Z}_q^{t-1}$ 上。

最后,为了证明上述声称,我们使用反证法,假设这个映射并不是一个单射。这意味着,存在两个互不相同的,最高 $t-2$ 阶的多项式 $g(x),h(x)\in\mathbb{Z}_q[x]$,使得多项式 $\alpha+xg(x)$ 和 $\alpha+xh(x)$ 在 $t-1$ 个\emph{非零}点 $x_1',\dots,x_{t-1}'$ 处一致。但这意味着,多项式 $g(x)$ 和 $h(x)$ 本身在这 $t-1$ 个点上也都一致,这与我们关于多项式插值的基本事实相矛盾。
\end{proof}

\subsection{ElGamal 门限加密}\label{subsec:11-6-2}

\begin{game}[门限加密的语义安全性]\label{game:11-4}
	
\end{game}

\begin{definition}[门限加密的语义安全性]\label{def:11-9}
	
\end{definition}

\begin{theorem}\label{theo:11-7}
	
\end{theorem}