\chapter{公钥基本工具}\label{chap:10}

在正式开始讨论公钥密码学之前,我们首先在本章介绍一些基本工具,它们将会在本书之后的部分中使用。接下来的几个章节将会陆续出现对这些工具的具体应用,具体地说,我们会将它们应用于公钥加密、数字签名和密钥交换中。由于我们将在本章中使用一些基本代数和基本数论,所以建议读者先简单浏览一下附录 \ref{chap:A}。

我们从一个简单的玩具问题开始:在两方之间生成一个共享密钥,并使得被动窃听对手无法有效地猜出这个共享密钥。对手可以监听网络通信,但不能篡改途中的消息或者注入自己的消息。在后面的章节中,我们将讨论,当存在可能篡改网络通信内容的主动攻击者时,如何开发进行密钥交换的完整机制。

首先我们强调,对于现实世界来说,针对窃听的安全性仍然是远远不够的,因为有能力监听网络通信的攻击者通常也能篡改网络通信;尽管如此,这个玩具性质的窃听模型仍然是一个介绍新公钥工具的好方法。

\section{一个玩具问题:匿名密钥交换}\label{sec:10-1}
\section{单向陷门函数}\label{sec:10-2}

\subsection{使用单向陷门函数方案的密钥交换}\label{subsec:10-2-1}

\subsection{数学细节}\label{subsec:10-2-2}
\section{一种基于 RSA 的陷门置换方案}\label{sec:10-3}
\section{Diffie-Hellman 密钥交换}\label{sec:10-4}
\section{离散对数及相关假设}\label{sec:10-5}

\subsection{随机自归约性}\label{subsec:10-5-1}

\subsection{数学细节}\label{subsec:10-5-2}
\section{来自数论原语的抗碰撞哈希函数}\label{sec:10-6}
\section{针对匿名 Diffie-Hellman 协议的攻击}\label{sec:10-7}
\section{Merkle谜题:一种使用分组密码进行密钥交换的部分解决方案}\label{sec:10-8}
\section{一个有趣的应用:RSA累加器}\label{sec:10-9}
\section{笔记}\label{sec:10-10}

对文献的引用有待补充。
\section{练习}\label{sec:10-11}

\begin{exercise}\label{exer:10-1}
\end{exercise}

\begin{exercise}\label{exer:10-2}
\end{exercise}

\begin{exercise}\label{exer:10-3}
\end{exercise}

\begin{exercise}\label{exer:10-4}
\end{exercise}

\begin{exercise}\label{exer:10-5}
\end{exercise}

\begin{exercise}\label{exer:10-6}
\end{exercise}

\begin{exercise}\label{exer:10-7}
\end{exercise}

\begin{exercise}\label{exer:10-8}
\end{exercise}

\begin{exercise}\label{exer:10-9}
\end{exercise}

\begin{exercise}\label{exer:10-10}
\end{exercise}

\begin{exercise}\label{exer:10-11}
\end{exercise}

\begin{exercise}\label{exer:10-12}
\end{exercise}

\begin{exercise}\label{exer:10-13}
\end{exercise}

\begin{exercise}\label{exer:10-14}
\end{exercise}

\begin{exercise}\label{exer:10-15}
\end{exercise}

\begin{exercise}\label{exer:10-16}
\end{exercise}

\begin{exercise}\label{exer:10-17}
\end{exercise}

\begin{exercise}\label{exer:10-18}
\end{exercise}

\begin{exercise}\label{exer:10-19}
\end{exercise}

\begin{exercise}\label{exer:10-20}
\end{exercise}

\begin{exercise}\label{exer:10-21}
\end{exercise}

\begin{exercise}\label{exer:10-22}
\end{exercise}

\begin{exercise}\label{exer:10-23}
\end{exercise}

\begin{exercise}\label{exer:10-24}
\end{exercise}

\begin{exercise}\label{exer:10-25}
\end{exercise}

\begin{exercise}\label{exer:10-26}
\end{exercise}

\begin{exercise}\label{exer:10-27}
\end{exercise}

\begin{exercise}\label{exer:10-28}
\end{exercise}