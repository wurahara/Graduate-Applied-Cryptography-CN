\section{语言与存在健全性}\label{sec:20-1}

让我们从一个定义开始。

\begin{definition}[真实陈述的语言]
令 $\mathcal{R}\subseteq\mathcal{X}\times\mathcal{Y}$ 是一个有效关系。如果对于某个 $x\in\mathcal{X}$ 有 $(x,y)\in\mathcal{R}$,那么我们就称陈述 $y\in\mathcal{Y}$ 是一个\textbf{真实陈述 (true statement)},否则称其为\textbf{虚假陈述 (false statement)}。我们称 $L_{\mathcal{R}}$ 为\textbf{由 $\mathcal{R}$ 定义的语言 (language defined by $\mathcal{R}$)},即定义在 $\mathcal{R}$ 上的所有真实陈述,也就是说,$L_{\mathcal{R}}:=\{y\in\mathcal{Y}:(x,y)\in\mathcal{R}\;\text{for some}\;x\in\mathcal{X}\}$。
\end{definition}

上述定义中的``语言 (language)" 一词来自复杂性理论。在本章中,我们将研究一些有趣的关系 $\mathcal{R}$ 以及由它们所定义的语言集合 $L_{\mathcal{R}}$。举一个上一章的例子,回顾一下,Chaum-Pedersen 协议是一个定义在关系
\[
\mathcal{R}:=\bigg\lbrace
\big(\beta, (u,v,w)\big)\in\mathbb{Z}_q\times\mathbb{G}^3:v=g^\beta, w=u^\beta
\bigg\rbrace
\]
上的 Sigma 协议。那么,由 $\mathcal{R}$ 定义的语言 $L_{\mathcal{R}}$ 就是所有 DH 三元组 $(u,v,w)\in\mathbb{G}^3$ 的集合。

于是,我们就可以用下面的攻击游戏来定义存在健全性的概念:

\begin{game}[存在健全性]\label{game:20-1}
令 $\Pi=(P,V)$ 是关系 $\mathcal{R}\subseteq\mathcal{X}\times\mathcal{Y}$ 上的一个 Sigma 协议。对于一个给定对手 $\mathcal{A}$,攻击游戏按照如下方式运行:
\begin{itemize}
	\item 对手 $\mathcal{A}$ 选择一个陈述 $y\in\mathcal{Y}$,并将其发送给挑战者。
	\item 对手 $\mathcal{A}$ 与验证者 $V(y)$ 进行交互,其中挑战者扮演验证者的角色,对手 $\mathcal{A}$ 扮演``作弊的"证明者的角色。
\end{itemize}
如果 $V(y)$ 在交互结束时输出 $\mathsf{accept}$ 但 $y\notin L_{\mathcal{R}}$,我们就称对手 $\mathcal{A}$ 赢得了该游戏。我们定义 $\mathrm{ES}\mathsf{adv}[\mathcal{A},{\Pi}]$ 为对手 $\mathcal{A}$ 对于 $\Pi$ 的优势,其值为对手 $\mathcal{A}$ 赢得该游戏的概率。
\end{game}

\begin{definition}
如果对于所有的有效对手 $\mathcal{A}$,$\mathrm{ES}\mathsf{adv}[\mathcal{A},{\Pi}]$ 的值都可以忽略不计,我们就称 Sigma 协议 $\Pi$ 是\textbf{存在健全 (existentially sound)} 的。
\end{definition}

\begin{theorem}
令 $\Pi$ 是一个有大挑战空间的 Sigma 协议,如果 $\Pi$ 提供知识健全性,那么 $\Pi$ 一定是存在健全的。
\begin{quote}
特别地,对于所有对手 $\mathcal{A}$,我们都有:
\end{quote}
\begin{equation}
\mathrm{ES}\mathsf{adv}[\mathcal{A},{\Pi}]
\leq
\frac{1}{N}
\end{equation}
\begin{quote}
其中 $N$ 是挑战空间的大小。
\end{quote}
\end{theorem}

\begin{proof}
我们只需证明,如果 $\mathcal{A}$ 选择了一个虚假陈述 $y$ 和一个承诺 $t$,那么最多只能有一个挑战 $c$ 使得存在一个应答 $z$ 能够产生一个 $y$ 的接受对话 $(t,c,z)$。观察到,如果有两个这样的挑战,那么 $y$ 将有两个接受对话 $(t,c,z)$ 和 $(t,c',z')$,其中 $c\neq c'$,而知识健全性意味着仅存在一个 $y$ 的见证,这与事实相矛盾。
\end{proof}

我们指出,对于任意强大的对手,上述定理都\emph{无条件}成立。我们在下一节将这些想法付诸实施。