\chapter{基本数论}\label{chap:A}

\section{循环群}\label{sec:A-1}

符号:对于一个有限循环群 $\mathbb{G}$,我们用 $\mathbb{G}^*$ 表示 $\mathbb{G}$ 的生成元集合。

\section{素数算术模运算}\label{sec:A-2}

\subsection{基本概念}\label{subsec:A-2-1}

我们使用字母 $p$ 和 $q$ 来表示素数。我们将要使用的素数都是大素数,比如 $300$ 字符($1024$ 比特)数量级的素数。
\begin{enumerate}
	\item 对于一个素数 $p>2$,令 $\mathbb{Z}_p=\{0,1,2,\dots,p-1\}$。\\
	$\mathbb{Z}_p$ 中的元素可以进行模 $p$ 加法和模 $p$ 乘法。对于 $x,y\in\mathbb{Z}_p$,我们用 $x+y$ 和 $x\cdot y$ 来分别表示 $x$ 和 $y$ 的模 $p$ 和与积。
	\item 费马定理:对于所有的 $0\neq g\in\mathbb{Z}_p$,都有 $g^{p-1}=1$。\\
	例:$3^4=81\equiv 1\,(\bmod\;5)$。
	\item $x\in\mathbb{Z}_p$ 的\emph{逆(inverse)}是一个 $\mathbb{Z}_p$ 上的满足 $a\cdot x=1$ 的元素 $a\in\mathbb{Z}_p$。\\
	$x\in\mathbb{Z}_p$ 的逆用 $x^{-1}$ 表示。\\
	例:1).\quad $\mathbb{Z}_5$ 上的 $3^{-1}$ 是 $2$,因为 $2\cdot 3\equiv1\,(\bmod\;5)$。\\
	\hspace*{18pt} 2).\quad $\mathbb{Z}_p$ 上的 $2^{-1}$ 是 $(p+1)/2$。
	\item 除 $x=0$ 外,$\mathbb{Z}_p$ 中所有的元素都是可逆的。\\
	一个简单(但不高效)的倒数算法:对于 $\mathbb{Z}_p$,$x^{-1}=x^{p-2}$。\\
	事实上,在 $\mathbb{Z}_p$ 上,我们有 $x^{p-2}\cdot x=x^{p-1}=1$。
	\item 我们用 $\mathbb{Z}_p^*$ 表示 $\mathbb{Z}_p$ 上所有可逆元素的集合。于是 $\mathbb{Z}_p^*=\{1,2,\dots,p-1\}$。
	\item 现在,我们有了解决 $\mathbb{Z}_p$ 上线性方程 $a\cdot x = b$ 的算法。\\
	通解:$x=b\cdot a^{-1}=b\cdot a^{p-2}$。\\
	那么解一元二次方程的算法呢?
\end{enumerate}

\subsection{$\mathbb{Z}_p^*$的结构}\label{subsec:A-2-2}

\begin{enumerate}
	\item $\mathbb{Z}_p^*$ 是一个\emph{循环群(cyclic group)}。\\
	换言之,存在一个 $g\in\mathbb{Z}_p^*$ 使得 $\mathbb{Z}_p^*=\{1,g,g^2,g^3,\dots, g^{p-2}\}$。\\
	这样的 $g$ 被称为 $\mathbb{Z}_p^*$ 的\emph{生成元(generator)}。\\
    例:在 $\mathbb{Z}_7^*$ 中:$\left\langle3\right\rangle=\{1,3,3^2,3^3,3^ 4,3^5,3^6\} \equiv\{1,3,2,6,4,5\}\,(\bmod\;7)=\mathbb{Z}_7^*$。
    \item 并非 $\mathbb{Z}_7^*$ 中的所有元素都可以作为生成元。\\
    例:在 $\mathbb{Z}_7^*$ 中,我们有 $\left\langle2\right\rangle=\{1,2,4\}\neq\mathbb{Z}_7^*$。
	\item $g\in\mathbb{Z}_p^*$ 的\emph{阶(order)}是使得 $g^a=1$ 成立的最小正整数 $a$。\\
    $g\in\mathbb{Z}_p^*$ 的阶用 ${\rm order}_p(g)$ 表示。\\
    例:${\rm order}_7(3)=6$,${\rm order}_7(2)=3$。
	\item 拉格朗日定理:对于所有的 $g\in\mathbb{Z}_p^*$,我们都有 ${\rm order}_p(g)$ 整除 $p-1$。\\
    注意到,费马定理是本定理的一个简单推论:\\
    \hspace*{30pt} 对于 $g\in\mathbb{Z}_p^*$,我们有 $g^{p-1}=(g^{{\rm order}(g)})^{(p-1)/{\rm order}(g)}=1^{(p-1)/{\rm order}(g)}=1$。
	\item 对于任意的 $g\in\mathbb{Z}_p^*$,如果已知 $p-1$ 的一种因式分解,则存在一种简单且有效的计算 ${\rm order}_p(g)$ 的算法。
\end{enumerate}

\subsection{二次剩余}\label{subsec:A-2-3}

\begin{enumerate}
	\item $x\in\mathbb{Z}_p$ 的\emph{平方根(square root)}是一个满足 $y^2=x \bmod p$ 的值 $y\in\mathbb{Z}_p$。\\
    例:1).\quad $\mathbb{Z}_7$ 上的 $\sqrt{2}$ 为 $3$,因为 $3^2=2 \bmod 7$。\\
    \hspace*{18pt} 2).\quad $\mathbb{Z}_7$ 上的 $\sqrt{3}$ 不存在。
	\item 如果一个元素 $x\in\mathbb{Z}_p^*$ 在 $\mathbb{Z}_p$ 上有一个平方根,我们就称 $x$ 是一个\emph{二次剩余(Quadratic Residue, QR)}。
	\item $x\in\mathbb{Z}_p$ 有多少个平方根?\\
    如果在 $\mathbb{Z}_p$ 上有 $x^2=y^2$,则 $0=x^2-y^2=(x+y)(x-y)$。\\
    $\mathbb{Z}_p$ 是一个``整数领域",这意味着 $x+y=0$ 或者 $x-y=0$,即 $x=\pm y$。\\
    因此,$\mathbb{Z}_p$ 上的元素要么没有平方根,要么就有 $2$ 个平方根。\\
    如果 $a$ 是 $x$ 在 $\mathbb{Z}_p$ 上的一个平方根,则 $-a$ 必然也是 $x$ 在 $\mathbb{Z}_p$ 上的平方根。
	\item 欧拉定理:当且仅当 $x^{(p-1)/2}=1$ 时,$x\in\mathbb{Z}_p$ 才是一个二次剩余。\\
    例:在 $\mathbb{Z}_7$ 上,我们有 $2^{(7-1)/2}=1$,但 $3^{(7-1)/2}=-1$。
	\item 令 $g\in\mathbb{Z}_p^*$。那么 $a=g^{(p-1)/2}$ 是 $1$ 的平方根。事实上,在 $\mathbb{Z}_p$ 上,我们有 $a^2=g^{p-1}=1$。\\
    $1$ 在 $\mathbb{Z}_p$ 上的平方根是 $1$ 或 $-1$。\\
    因此,对于 $g\in\mathbb{Z}_p^*$,我们知道 $g^{(p-1)/2}$ 是 $1$ 或 $-1$。
	\item 勒让德符号:对于 $x\in\mathbb{Z}_p$,我们定义:
    \[
    \left(\frac{x}{p}\right):=
    \left\{
    	\begin{array}{rl}
    	1,  &x\;\text{是}\;\mathbb{Z}_p\;\text{上的一个二次剩余},\\
    	-1, &x\;\text{不是}\;\mathbb{Z}_p\;\text{上的一个二次剩余},\\
    	0,  & x=0 \bmod p
    \end{array}
    \right.
    \]
	\item 根据欧拉定理,我们知道,在  $\mathbb{Z}_p$ 上,有 $\left(\frac{x}{p}\right)=x^{(p-1)/2}$。\\
	$\Longrightarrow$\quad 勒让德符号可以有效计算。
	\item 简单的事实:令 $g\in\mathbb{Z}_p^*$ 是一个生成元。令 $x=g^r$,其中 $r$ 是某个整数。\\
    那么,当且仅当 $r$ 是偶数时,$x$ 是 $\mathbb{Z}_p$ 的一个二次剩余。\\
    $\Longrightarrow$\quad \textbf{勒让德符号会透露 $r$ 的奇偶性。}
	\item 由于当且仅当 $r$ 是偶数时,$x$ 才是 $\mathbb{Z}_p$ 的一个二次剩余,易知 $\mathbb{Z}_p$ 中恰好半数元素是二次剩余。
	\item 当 $p=3 \bmod 4$ 时,计算 $x\in\mathbb{Z}_p$ 的平方根是容易的。\\
    只需要在 $\mathbb{Z}_p$ 上简单计算 $a=x^{(p+1)/4}$。\\
    $a=\sqrt{x}$,因为在 $\mathbb{Z}_p$ 上,我们有 $a^2=x^{(p+1)/2}=x\cdot x^{(p-1)/2}=x\cdot 1=x$。
	\item 当 $p=1 \bmod 4$ 时,计算 $\mathbb{Z}_p$ 上的平方根是可行的,但是有点复杂;我们需要一个随机化算法。
	\item 我们现在已经有一个解决  $\mathbb{Z}_p$ 上的二次方程的算法。\\
    我们知道,如果方程 $ax^2+bx+c=0 \bmod p$ 的解存在,它必能由下面的 $\mathbb{Z}_p$ 上的计算给出:
    \[
    x_{1,2}=\frac{-b\pm\sqrt{b^2-4ac}}{2a}
    \]
    因此,当且仅当 $\Delta=b^2-4ac$ 是 $\mathbb{Z}_p$ 上的一个二次剩余时,方程在 $\mathbb{Z}_p$ 上才有根。使用我们在 $\mathbb{Z}_p$ 上获取平方根的算法,我们可以找到 $\sqrt{\Delta} \bmod p$,进而恢复 $x_1$ 和 $x_2$。
	\item 那么,$\mathbb{Z}_p$ 上的三次方程呢?存在一个有效的随机化算法能够在 $d$ 阶多项式时间内解出 $\mathbb{Z}_p$ 上的任意 $d$ 次方程。
\end{enumerate}

\subsection{$\mathbb{Z}_p$上的计算}\label{subsec:A-2-4}

\begin{enumerate}
	\item 由于 $p$ 是一个大素数(比如 $1024$ 比特长),它无法被存储在一个单一的寄存器中。
	\item $\mathbb{Z}_p$ 中的元素会被存储在桶中,根据处理器芯片的位宽,桶的大小可能是 $32$ 或 $64$ 比特。
	\item 计算 $\mathbb{Z}_p$ 上任意两个元素 $x,y$ 的加法可以在线性时间内完成,正比于 $p$ 的\emph{长度}。
	\item 计算 $\mathbb{Z}_p$ 上任意两个元素 $x,y$ 的乘法可以在平方时间内完成,二次正比于 $p$ 的\emph{长度}。如果 $p$ 有 $n$ 比特长,设计较好的算法能够在 $O(n^{1.7})$ 时间内完成(优于 $O(n^2)$)。
	\item 计算 $\mathbb{Z}_p$ 上任意元素 $x$ 的逆可以在平方时间内完成,二次正比于 $p$ 的长度。
	\item 使用重复平方算法,$x^r\in\mathbb{Z}_p$ 可以通过不超过 $2\log_2r$ 次 $\mathbb{Z}_p$ 上的乘法操作完成。
	\item 当指数运算的底 $x$ 固定时,我们可以预先计算一张 $x$ 的幂表,包含 $(2^w/w)\cdot\log_2r$ 个群元素,将计算 $x^r\in\mathbb{Z}_p$ 的时间减小到 $(\log_2r/w)$ 次 $\mathbb{Z}_p$ 上乘法的耗时。这里,$w$ 是一个小常数,比如 $w=5$。
\end{enumerate}

\subsection{总结:素数算术模运算}\label{subsec:A-2-5}

令 $p$ 是一个 $1024$ 比特的素数。$\mathbb{Z}_p$ 上的简单问题:
\begin{enumerate}
	\item 生成一个随机数。计算元素的加法和乘法。
	\item 计算 $g^r \bmod p$ 是容易的,即使 $r$ 很大。
	\item 计算一个元素的逆。求解线性方程。
	\item 检验一个元素是否是二次剩余。如果一个元素是二次剩余,计算它的平方根。
	\item 计算 $d$ 阶多项式方程可以在 $d$ 阶多项式时间内完成。
\end{enumerate}

在 $\mathbb{Z}_p$ 上被认为是困难的问题:
\begin{enumerate}
	\item 令 $g$ 是 $\mathbb{Z}_p^*$ 的一个生成元。给定 $x\in\mathbb{Z}_p^*$,找到一个 $r$ 使得 $x=g^r \bmod p$。该问题被称为\emph{离散对数问题(discrete log problem)}。
	\item 令 $g$ 是 $\mathbb{Z}_p^*$ 的一个生成元。给定 $x,y\in\mathbb{Z}_p^*$,其中 $x=g^{r_1}$,$y=g^{r_2}$。求解 $z=g^{r_1r_2}$。该问题被称为 \emph{Diffie-Hellman 问题}。
\end{enumerate}

\section{合数算术模运算}\label{sec:A-3}

\subsection{基本概念}\label{subsec:A-3-1}

我们要处理的是长为 $300$ 个字符($1024$ 比特)的整数 $n$。除非另有说明,我们假设 $n$ 是两个长度相同的素数的积,例如,每个素数都有 $150$ 个字符($512$ 比特)。
\begin{enumerate}
	\item 对于一个合数 $n$,令 $\mathbb{Z}_n=\{0,1,2,\dots,n-1\}$。\\    
    $\mathbb{Z}_n$ 中的元素可以进行模 $n$ 加法和乘法。
	\item $x\in\mathbb{Z}_n$ 的逆是一个元素 $y\in\mathbb{Z}_n$,$y$ 满足 $x\cdot y=1 \bmod n$。\\    
    当且仅当 $x$ 和 $n$ 互素时,元素 $x\in\mathbb{Z}_n$ 有逆。换言之,$\gcd(x,n)=1$。    
	\item 利用欧几里得算法,可以有效计算 $\mathbb{Z}_n$ 上元素的逆。如果 $\gcd(x,n)=1$ 成立,使用欧几里得算法,我们就可以有效地构造两个整数 $a,b\in\mathbb{Z}_n$,使得 $ax+bn=1$。将该关系模 $n$,可以导出 $ax=1 \bmod n$。因此,$a=x^{-1} \bmod n$。\\
    注意:这一求逆算法也可用于 $\mathbb{Z}_p$ 中,其中 $p$ 是素数。使用该算法求逆比计算 $x^{p-2} \bmod p$ 还要更有效。
	\item 令 $\mathbb{Z}_n^*$ 为 $\mathbb{Z}_n$ 中所有可逆元的集合。
	\item 我们现在有一个解线性方程 $a\cdot x = b \bmod n$ 的算法。    
    通解:$x=b\cdot a^{-1}$,其中 $a^{-1}$ 可以使用欧几里得算法计算。    
	\item $\mathbb{Z}_n^*$ 中包含多少个元素?我们用 $\varphi(n)$ 表示 $\mathbb{Z}_n^*$ 中元素的个数。我们已经知道,对于素数 $p$,有 $\varphi(p)=p-1$。    
	\item 我们可以证明,如果 $n=p_1^{e_1}\cdots p_m^{e_m}$,则有 $\varphi(n)=n\cdot\prod_{i=1}^m(1-1/p_i)$。\\
    特别地,当 $n=pq$ 时,我们有 $\varphi(n)=(p-1)(q-1)=n-p-q+1$。\\
    例:$\varphi(15)=\big\lvert\{1,2,4,7,8,11,13,14\}\big\rvert=8=2\times4$。
	\item 欧拉定理:所有的 $a\in\mathbb{Z}_n^*$ 在 $\mathbb{Z}_n$ 上都满足 $a^{\varphi(n)}=1$。\\
    注意:对于素数 $p$,欧拉定理意味着,对于所有的 $a\in\mathbb{Z}_p^*$,都有 $a^{\varphi(p)}=a^{p-1}=1$。因此,欧拉定理是费马定理的一个推广。
\end{enumerate}

\subsection{$\mathbb{Z}_n^*$ 的结构}\label{subsec:A-3-2}

\begin{theorem}[中国剩余定理]\label{theo:A-1}
陈述定理。
\end{theorem}

\subsection{总结:合数算术模运算}\label{subsec:A-3-3}

令 $n$ 是一个 $1024$ 比特的整数,它是两个 $512$ 比特的素数的积。$\mathbb{Z}_n$ 上的容易问题:
\begin{enumerate}
	\item 生成一个随机元素。计算元素的加法和乘法。
	\item 计算 $g^r \bmod n$ 是容易的,即使 $r$ 很大。
	\item 计算一个元素的逆。求解线性方程。
\end{enumerate}

如果 $n$ 的因式分解未知,下面的问题被认为是困难的,但如果 $n$ 的因式分解已知,它们就会变得容易:
\begin{enumerate}
	\item 寻找 $n$ 的素因子。
	\item 检验一个元素是否是 $\mathbb{Z}_n$ 上的二次剩余。
	\item 计算 $\mathbb{Z}_n$ 上的一个二次剩余的平方根。可以证明,这与 $n$ 的因式分解的难度相当。当 $n=pq$ 的因式分解已知时,我们可以计算 $x\in\mathbb{Z}_n^*$ 的平方根,方法是首先计算 $x \bmod p$ 在 $\mathbb{Z}_p$ 上的平方根,再计算 $x \bmod q$ 在 $\mathbb{Z}_q$ 上的平方根,最后使用中国剩余定理得到 $x$ 在 $\mathbb{Z}_n$ 上的平方根。
	\item 当 $\gcd(e,\varphi(n))=1$ 时,计算模 $n$ 的第 $e$ 个根。
	\item 更一般地,求解 $d>1$ 阶多项式方程。如果 $n$ 的因式分解是已知的,这个问题就很容易:我们可以首先找到模 $n$ 的素因子的多项式方程的根,然后用中国剩余定理得到 $\mathbb{Z}_n$ 上的根。
\end{enumerate}

在 $\mathbb{Z}_n$ 上被认为是困难的问题:
\begin{enumerate}
	\item 令 $g$ 是 $\mathbb{Z}_n^*$ 的一个生成元。给定 $x\in\mathbb{Z}_n^*$,找到一个 $r$ 使得 $x=g^r \bmod n$。该问题被称为\emph{离散对数问题}。
	\item 令 $g$ 是 $\mathbb{Z}_n^*$ 的一个生成元。给定 $x,y\in\mathbb{Z}_n^*$,其中 $x=g^{r_1}$,$y=g^{r_2}$。求解 $z=g^{r_1r_2}$。该问题被称为 \emph{Diffie-Hellman 问题}。
\end{enumerate}