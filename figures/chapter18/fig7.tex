\tikzset{every picture/.style={line width=0.75pt}}

\begin{tikzpicture}[x=0.75pt,y=0.75pt,yscale=-0.9,xscale=0.9]

\draw  [fill={rgb, 255:red, 255; green, 255; blue, 255 }  ,fill opacity=1 ][line width=1.2] [general shadow={fill=black,shadow xshift=2.25pt,shadow yshift=-2.25pt}] (5,80) -- (115,80) -- (115,300) -- (5,300) -- cycle ;
\draw  [fill={rgb, 255:red, 255; green, 255; blue, 255 }  ,fill opacity=1 ][line width=1.2] [general shadow={fill=black,shadow xshift=2.25pt,shadow yshift=-2.25pt}] (340,80) -- (450,80) -- (450,300) -- (340,300) -- cycle ;

\draw   (25,250) -- (95,250) -- (95,290) -- (25,290) -- cycle ;

\draw  [->]  (60,290) -- (60,340) ;

\draw  [->]  (115,110) -- (338,110) ;
\draw  [->]  (115,150) -- (338,150) ;
\draw  [->]  (115,230) -- (338,230) ;
\draw  [<-]  (96,270) -- (340,270) ;

\draw (60,65) node   [align=left] {挑战者};
\draw (60,100) node    {$( vk,sk)\overset{\mathrm{R}}{\leftarrow } G()$};
\draw (60,270) node    {$V( vk)$};
\draw (225,175) node    {$\vdots $};
\draw (225,106.6) node [anchor=south] [inner sep=0.75pt]    {$vk$};
\draw (395,65) node   [align=left] {对手 $\mathcal{A}$};
\draw (65,337) node [anchor=west] [inner sep=0.75pt]   [align=left] {$\mathsf{accept}$ 或 $\mathsf{reject}$};

\draw (225,147) node [anchor=south] [inner sep=0.75pt]   [align=left] {$\text{对话}_1$};
\draw (225,227) node [anchor=south] [inner sep=0.75pt]   [align=left] {$\text{对话}_Q$};
\draw (225,267) node [anchor=south] [inner sep=0.75pt]   [align=left] {冒充尝试};


\end{tikzpicture}